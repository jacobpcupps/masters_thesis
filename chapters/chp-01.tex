\onehalfspacing

\epigraph{{``Listening correctly can cause introspecting''}}{R.A.P. Ferreira}

\section{Open Mike Eagle and Nocando's ``Unapologetic''}

\begin{itemize}
    \item This song represents the advent of a subgenre of rap/hip-hop with which my research is concerned
    \item I'm particularly interested in how the emcees identify themselves in their lyrics with respect to some 
    overlapping complexities of identity:
          \begin{itemize}
                \item They're small-time enough that they identify with an underground (Nocando: ``I'm underground man / Like 
                I'm  \emph{beneath} the streets'') as a point of pride, but also big-time enough that they don't identify with 
                ``scenesters,'' as a local act might (Nocando: ``My name's bigger on the flyer / There's a reason I'm performing
                after him'')
                \item They have some fans that they're very adamant about reaching (OME: ``all the adolescent Negro lads, / Who make 
                sketches and daydream the whole class'') and some who seem to be in the way (OME: ``Descendants of the folks that 
                shackled my ancestors / Come to watch us battle and cackle like Fran Drescher'')
                \item They have some need for financial success (OME: ``Scene's built on the backs of black rappers / (somehow) when
                I'm asking for cash you can't answer''), but conceive of a level of success where they'd be selling out (OME: ``The 
                only [rappers his brother] can discover / are the one's who please Viacom's executive n---- lovers'')
          \end{itemize}
\end{itemize}

``Unapologetic'' creates a complex picture of the (black, masculine) identities of the emcees who rap on it, in a way 
that demands some consideration of the lyrics on their own terms. At the same time, the lyrics exist in a musical context,
itself working to sound some version of their identities. In the following sections, I want to examine how these identities
might be contextualized: between music and text, what styles of hip-hop does the song invoke, and what histories might
we position it within? Although (as my title suggests) I will ultimately contextualize ``Unapologetic'' within a genre 
designation I call ``Underground'', I first want to consider it in relationship to a few genre categorizations which
have previously been theorized by hip-hop scholars: namely, Amanda Sewell's genre designation ``nerdcore'' and Adam Krims'
proposition of a ``jazz/bohemian'' rap (rearticulated by Justin Williams as ``jazz rap as a high art'').

\section{Nerdcore, Jazz-Rap, and the ``Golden Age''}

For me as a listener, ``Unapologetic'' engenders two connections to the nerdcore subgenre. First, its beat 
layer is primarily constructed from two synthesizer loops. Although synthesizer does not explicitly
correlate to ``nerdiness'' in most contexts, when deployed within a hip-hop beat, it draws my attention
towards the technical prowess of the producer. This hearing also helps contextualize concerns of intelligence
made by both emcees. Nocando puts his own intelligence on display moving through a wide variety of allusions, 
including the Hanna-Barbera cartoon \textit{Tom and Jerry}, Joshua Marston's spanish-language film 
\textit{María, llena eres de gracia}, and even a rewritten version of Psalm 23.\footnote{
    According to Rory Ferreira, Nocando's re-contextualizaiton of Biblical language, here, might also 
    allude to Friedrich Nietzsche's \textit{Twilight of the Idols}, where he also plays with language
    from the same Psalm (Genius Annotation?).} 
Though less topically diffuse, Eagle's verses focus on intelligence by painting a picture of the listener he
attempts to reach: the ``adolescent Negro lads / who make sketches and daydream the whole class.''

The emcees' focus on intelligence through these contrasting approaches relates to the generic construction of
nerdcore. According to Amanda Sewell, ``Nerdcore hip-hop happens when technologically savvy, verbally precocious,
and social marginalized people begin to make hip-hop using their skill sets and experiences.''\footnote{
    \autocite[223]{amandasewellNerdcoreHiphop2015}.}
   \begin{itemize}
    \item Sewell's conception of nerdcore:
            \begin{itemize}
                \item MC Frontalot ``[As a nerd, you] don't fit into the traditional expectations of 
                masculinity, especially as a teenager'' (226). Compare to OME's ``adolescent negro lads who 
                daydream the whole class''
            \end{itemize}
        \item OME's ``Art Rap Party'' liner notes video:
            \begin{itemize}
                \item ``My idea of an art rap party, which is somewhat imaginary and hasn't all the way
                happened yet, is the kind of function where, like, intelligent folks go to dig on some
                intelligent music, \textellipsis it's not a dance party, you know what I mean?''
                \item ``  It's the kind of party where information's exchanged between people[;] 
                information's exchanged between performer and audience, and back and forth'' 
                \item ``Of course it's gonna be a little bit nerdy[.] \textellipsis We could talk about 
                anime, \textellipsis political stuff, exploring realms conversationally that aren't in the 
                typical rap song''
            \end{itemize}
    \end{itemize}

Adam Krims constructs a variety of subgeneric designations for rap that he argues started to concretize
around 1994, each harnessing distinct musical styles to address particular topics. In particular, he 
notes that the genres he designates ``jazz/bohemian'' rap and ``reality'' rap articulated themselves 
as alternatives  to ``gangsta'' rap based on the perception that it was the genre most aligned with
a commercial mainstream.\footnote{
    \autocite[64--65]{adamkrimsRapMusicPoetics2000}. Krims offers two distinct case studies\textemdash
    The Roots' ``What They Do'' and RZA's introduction on Wu-Tang Clan's \textit{Wu-Tang Forever}\textemdash 
    as instances of each genre directly addressing what it conceived to be the commercial mainstream 
    at the time.}
In much the same way, Eagle's criticism of the types of artists that please Viacom's executives and his 
estimation of his own work as a ``Black man's art'' allow him to construct the same dichotomy; he and
other rappers like him are bohemians within the music industry, and you, the listener who can appreciate
his style, are a connoisseur of ``fine art'' rap music.

    \begin{itemize}
        \item OME's definitions (and genre designation) construct art rap as a type of high art taken with 
        respect to the gamut of hip-hop in the 2010s
        \item Another way in which hip-hop musicians have constructed their sound to be heard as a high art
        within  hip-hop communities is through sonic links to jazz and related Black sonic profiles (Table 1.1)
            \begin{itemize}
                \item As early as 2010, artists including Kendrick Lamar and billy woods are sampling 
                from  mainstream jazz in the seventies and eighties
                \item As the decade progressed, other artists including R.A.P. Ferreira (fka Milo, aka 
                Scallops Hotel)  and Armand Hammer (the rap duo consisting of woods and ELUCID) begin to 
                draw on more eclectic styles of jazz, especially Sun Ra
                \item Soul and blues samples proliferate throughout the decade as well
            \end{itemize}
        \item Williams (2015) argues that, during the golden era of hip-hop, artists used jazz and other 
        Black sonic profiles in order to construct their work as a ``high art'' substyle within the genre
            \begin{itemize}
                \item Jazz rap came into formation as an ``alernative'' to gansta and pop rap, and jazz in 
                the decade  preceding had become an institutional and ``serious'' art form (48--52).
                \item Certain musical elements function as ``jazz codes'' which, through performance and 
                timbral qualities, reach audience members as having a jazz \emph{feel}, even if they are 
                not drawn from jazz records, strictly speaking (55).
                \item Groups like A Tribe Called Quest and Digable Planets drew on these codes to (explictly 
                or subtextually) reach listeners as artists with an identity that was different than those of 
                their contemporaries working in pop/gangsta rap styles. (58)
            \end{itemize}        
        \item While I'm not convinced that jazz rap's renaissance in the 2010s was explicitly motivated to construct 
        jazz rap as a type of high art uniformly, I think the same urge to \emph{differentiate} one's self to the 
        listener, to articulate identity in a way that goes against some construction of identity in the mainstream, 
        still exists as a driving force in the 2010s.
    \end{itemize}
    
The sonic profile of ``Unapologetic'' does not fit neatly into the generic category Krims and Williams
construct. While both acknowledge the possibility for a stylistic eclecticism beyond that reaches beyond
jazz sonically, the groups that receive their focus (A Tribe Called Quest, De La Soul, Digable Planets, 
Jungle Brothers, and The Roots) all rely to \emph{some} degree on the sound of jazz in their music. 
Without the sonic trappings of some jazz codes signify some relationship to jazz in sound, 
``Unapologetic'' cannot reach listeners as ``bohemian''\textemdash at least not in the same way 
the groups connected to the ``Golden Age'' did. I reject that ``Unapologetic'' can neatly
fit in within one sound or style, so I will instead look for some sort of practice in the music
which can be emulated.f

\begin{sidewaystable}[p]
    \centering
    \small
    \begin{tabular}{|c|c|c|c|}
         \hline
        Year & Track & Referent & Type \\ \hline
        2018 & Armmand Hammer -- ``VX'' & Prince Far I -- ``Throw Away Your Gun'' & Sample \\ \hline
        2018 & Armmand Hammer -- ``No Days Off'' & Sun Ra Arkestra -- ``The All of Everything'' & Sample \\ \hline
        2012 & billy woods (ft. Elucid) -- ``Sour Grapes'' & Miles Davis -- ``Pharoah's Dance'' & Sample \\ \hline
        2012 & billy woods -- ``Body of Work'' & Nina Simone -- ``Work Song'' & Sample \\ \hline
        2012 & billy woods -- ``Crocodile Tears'' & Muddy Waters -- ``Champagne \& Reefer'' & Interpolation \\ \hline
        2012 & billy woods -- ``DCMA'' & Junior Murvin -- ``Police and Thieves'' & Interpolation \\ \hline
        2019 & billy woods -- ``Fnu Lnu'' & Hank Crawford -- ``Wildflower'' & Sample \\ \hline
        2010 & Kendrick Lamar -- ``Rigamortus'' & Willie Jones III - ``The Thorn'' & Sample \\ \hline 
        2012 & Kendrick Lamar -- ``Sing About Me, I'm Dying of Thirst'' & Grant Green -- ``Maybe Tomorrow'' & Sample \\ \hline
        2015 & Kendrick Lamar -- ``King Kunta'' & James Brown -- ``The Payback'' & Interpolation \\ \hline
        2017 & Kendrick Lamar ``XXX.'' & James Brown -- ``Get Up Offa That Thing'' & Sample \\ \hline
        2013 & Milo (ft. Busdriver) -- ``The Gus Haynes Cribbage League'' & Quincy Jones (ft. James Ingram) -- ``Just Once'' & Sample \\ \hline
        2015 & Milo (ft. Hemlock Ernst) -- ``Souvenir'' & Shuggie Otis -- ``Rainy Day'' & Sample \\ \hline
        2017 & Milo -- ``Call + Form (Picture)'' & Eddie Munji III -- ``Doon Po Sa Amin'' & Sample \\ \hline
        2017 & Milo (ft. Elucid) -- ``Landscaping'' & Sun Ra -- ``Quiet Ecstasy'' & Sample \\ \hline
        2018 & Milo -- ``Tiptoe'' & Hank Crawford -- ``Teach Me Tonight'' & Sample \\ \hline
        2016 & Scallops Hotel -- ``Niopo Tree Stipend'' & Ella Jenkins -- ``Moon Don't Go'' & Sample \\ \hline
        2016 & Scallops Hotel (ft. SB the Moor) -- ``Lanquidity'' & Sun Ra -- ``Lanqudiity'' & Sample \\ \hline
        2017 & Scallops Hotel -- ``Ain't No Hustle Where I Live'' & Stanley Cowell -- ``Here I Am'' & Sample \\ \hline
        2017 & Scallops Hotel -- ``A Beat for My Lil Boy'' & Sun Ra -- ``Where There Is No Sun'' & Sample \\ \hline
    \end{tabular}
    \caption{References to jazz, soul, and funk pieces in 2010s underground hip-hop.}
    \label{tab:jazz_references}
\end{sidewaystable}


\newpage
\section{Hip-Hop Practice and Listening as Meditation}

In the previous section, I argue that the underground is characterized not by one style such as
nerdcore or jazz rap but rather by a musical practice. This practice is one of listening: in 
particular, a type of sonic engagement I will term \emph{listening as meditation}. At every step 
of the making of underground hip-hop\textemdash its conception in the mind of the producer and 
emcee, its performance live or in studio, and its reception by the listener in either space\textemdash 
the style of listening being promoted is one that engenders a deep immersion inwith hip-hop as 
a musical form.

The phrase listening as meditation serves as an inversion of the mind-body dualism invoked by 
earlier popular music discourses.\footnote{
    Early examples of popular music criticism such as \textit{Crawdaddy} magazine used 
    listening-forward models for understanding and canonizing the music of The Beatles and Bob
    Dylan, stressing that listening can and should be abstracted from textual interpretation.
    According to Norma Coates, these early rock critics imported a vocabulary based upon a
    binary construction of high culture as masculine/intellectualized and mass culture
    as feminine/of the body. (see \autocite[66\textit{ff}.]{coatesTeenyboppers2009})}
To be sure, by invoking ``listening'' as a unifying thread amongst a wide variety of stylistic 
manifestations, I am inviting a tacit definition of the underground as ``hip-hop to be 
\emph{listened} to'' as distinct from ``to be danced to'' or ``merely heard.'' Part of the 
trouble with such a definition is potential its use as a justification  for excluding certain
musics from being objects of critical inquiry. Moreover, my choice to prioritize discussing 
this music as an engagement of one's musical intellect does not exclude its use as ``dance'' 
music, nor as a music that does not engender embodied responses. The metaphor of listening 
as meditation, then, proposes a unification of cognitive and embodied responses that characterize
engagement with underground hip-hop. In short, it is an argument that will, in the coming pages,
put this chapter's epigraph to the test (that ``Listening correctly can cause introspecting.'')

At each point in the process of making hip-hop, and indeed throughout different points in
its history as a music form, several practitioners have evinced that their listening exhibits
a meditative engagement with music. The rapper-producer Count Bass D suggests that this style
of listening guides his process for sample selection and composition particularly for the
introduction to his 2002 track ``Truth to Light'': 
\begin{quote}
    Songs that really I like a whole lot, that I've liked over the years, kind of run through my head all
    the time and so they kind of creep into songs. \textellipsis Unless you know [Nice \& Smooth's ``Funky
    for You''], you don't know who I'm talking about or what I'm talking about, but to people who are in
    the know, it strengthens their faith that the things I'm talking about that they don't understand may
    have some relevance to them in time.\footnote{\cite{mickeyhessHipHopDead2007}: 100.}
\end{quote}
Count particularly focuses on familiarity, being ``in the know'' with the source material, as
a requirement for parsing his dialogue with the Nice \& Smooth at the beginning of the track, but
his discussion also reveals listening norms within the hip-hop community. First, Count as a producer
engaged with ``Funky for You'' regularly enough over a long period of time that the decision to
dialogue with the artists ad-libs on the original recording felt subliminal and ``[crept] into''
the track. Second, he expects that his audience will have engaged with the material on a similar
level, enough to ``know what [he's] talking about.'' Finally, and perhaps most interestingly,
Count conceives of a \emph{type} of listener, who may not know the source material but as a
result of their familiarity with the listening norms in the genre trusts will come to them
``in time'' as they listen further and more broadly beyond ``Truth to Light.'' Such a
listener is the \emph{meditative} listener and is the type of listener Count considers
his primary audience.

The meditative listening process for the emcee is similar to the producer's in that the emcee
listens with enough regularity and focus that the imitation of style can become subconscious,
to the point where it can prove problematic for their on writing process. Open Mike explains
that because of his listening interactions with the emcee MF DOOM's flow, ``he has to be 
careful with [DOOM's] flow, because [he] can almost get into [DOOM's] mind in terms of how he 
writes.''\footnote{\cite{estellecaswellRappingDeconstructedBest2016}.} Open Mike characterizes
his depth of engagement with DOOM as something that could turn into a problem because it would
be easy for him to be write verse in the same exact manner that DOOM does and thus turn into
something more derivative than he would like. Again the practitioner positions their compositional
choices as a subconscious choice based on deep engagement with models for composing that show 
up in their listening. For Open Mike as an emcee, this listening has occurred with enough
frequency, depth, and regularity that he subsumed DOOM's style of flow into his own and needs
to consciously choose to offset his writing from this practice at times.

Finally, hip-hop practitioners show that listening as meditation informs their interpretation of
performance by other practitioners; they envision the artists they interact with as meditative
listeners, who then serve as guides for further meditative listening. The Bronx DJ Grand Wizzard
Theodore describes, when hearing a record through DJ Kool Herc's soundsystem, ``[i]t made you
listen to a record and made you appreciate the record even more. He would play a record that 
you listened to every day and you would be like `Wow, that record has \emph{bells} in it?' It's 
like you heard instruments in the record that you never thought the record even had.''\footnote{
\autocite[139]{christabronGlassHiphopProduction2015}.} Theodore's description shows that even
within the context of a block party, the type of listening hip-hop engenders brings about new
depths to the music beyond the standard engagement.

%\newpage
\section{Definitions and Limitations of Underground}

Mike Eagle's ``Unapologetic''---indeed, his whole conception of Art Rap as a moment at the start
of the 2010s---articulates something that is at once new in hip-hop while relating to ideas, sounds,
and genres from throughout hip-hop's history. Although the track's ``tech-y'' sound and construction 
of ``intelligence'' intimates some connection to nerdcore, the designation falls short because of 
nerdcore's historically white positionality and lack of concern with its own construction as art. To
situate ``Unapologetic'' within the lineage of jazz/bohemian rap does not adequately account for the 
timbral qualities or ``codes'' of the track but gets closer to the topical focus and conception how
Art Rap conceives of itself with respect to some commercial mainstream. 


\begin{itemize}
    \item I'm choosing to call this style \emph{underground} as opposed to art rap, indie rap, or the like
    for a few reasons:
        \begin{itemize}
            \item I don't want to import discourses of art/beauty or of authenticity (as much as that can 
            be avoided with how musicians talk about the underground)
            \item I don't want to limit what stylistic or timbral elements ``count'' as underground
            \item The artists I cover seem to have moved away from Art Rap as a moment (OME on Milo's
            ``Otherground Pizza Party'' and Rory himself on ``Twenty on Five''), but at least conceive of
            themselves as related to the underground 
        \end{itemize}
    \item Add these points to ideas from Prospectus:
        \begin{itemize}
            \item If the underground exists, it manifests itself in sonic and semantic (musical and lyrical?)
            dimensions. Principally, I am interested in the former, though obviously acknowledge the latter
            where I can
            \item Close readings via transcription is my primary methodology (I expand later on the
            difficulties of capturing certain musical elements in Standard Western Notation)
        \end{itemize}
\end{itemize}

This thesis focuses on transcribing and analyzing a repertory of rap songs that sound
the quality of ``undergroundness'' within their beats and flows. While the term underground 
conjures up meanings predicated on music discourses of gatekeeping and authenticity, I do 
not intend to explore it as a way of forming some alternative canon within the genre; instead, 
I wish to explore undergroundness as mode in which listeners access the music. I follow Loren 
Kajikawa in arguing that the rap song is a medium for transmitting musical meaning between 
hip-hop musicians and listeners.\footnote{\cite{lorenkajikawaSoundingRaceRap2015}, 2.} Where 
he notes the rap song's ability to code racial and gender identities, I observe a subsequent 
level of coding: the extent to which the song (and thus its creators) accept mainstream narratives 
about those identities.

While rappers can communicate underground identity at the textual level, I contend that undergroundness
permeates the rap song beyond text. Thus, in my repertory, I use transcription to examine methods by which
rappers and producers distinguish their musical performance as underground. In general, I note an attitude
towards music making that is anti-commodification, and aligns this belief with a narrative of `getting 
back' to a more authentic form of hip-hop. Fredrick Brathwaite (aka Fab 5 Freddy) summarized this mentality
when discussing his role as a writer for the 1982 film \textit{Wild Style}. The writers set the story 
before the 1979 release of Sugarhill Gang's single ``Rapper's Delight'' because ``[they] wanted to go back
a few years earlier\textellipsis \emph{when hip-hop was completely underground, when the form was raw and
pure}.''\footnote{Quoted in \cite{justinawilliamsRhyminStealinMusical2013}, 23. Emphasis my own. The release
of ``Rapper's Delight'' is often thought of as the moment when hip-hop went commercial.} Although I am not
interested in assessing undergroundness along the evaluative lines Brathwaite implies, I have observed an
unspoken consensus within the underground that the sound aesthetic of mid-1970s of hip-hop is one to be
replicated.

I believe hip-hop musicians mimic this era not only because of their evaluative judgements about it, but
because doing so positions their music in a lineage of black American music forms; this aesthetic link is
manifest in what Olly Wilson terms the ``heterogeneous sound ideal'' of African American
music.\footnote{\cite{ollywilsonHeterogeneousSoundIdeal1992}: 329.} In particular, underground hip-hop
champions an aesthetic of heterogeneity through its sampling of disparate sources, as well as  through
composing in ways that eschew textural and timbral homogeneity. This compositional style distinguishes the
work as underground, in part because the ability to create more fully-polished styles of hip-hop is available
to anyone with a Digital Audio Workstation. 

\section{Chapter Breakdown}
\begin{itemize}
    \item Chapter 2: 
        \begin{itemize}
            \item Form \& Snapshot Transcriptions
            \item Variety within repetition through 4 techniques I identify
            \item Heterogeneity, diversity within the texture punctuates alternative identity
        \end{itemize}
    \item Chapter 3:
        \begin{itemize}
            \item ??? Transcriptions of Flow
            \item Emcees manipulate elements of structure as lyricists, performance as vocalists
            \item Olly Wilson / Heterogeneity? Tricia Rose? Mitchell Ohriner?
        \end{itemize}
\end{itemize}

The two primary chapters in this project will interrogate the overlapping yet distinct methods
by which hip-hop musicians sound the underground. My first chapter focuses on beatmaking, deconstructing 
the notion of a fixed-loop hip-hop beat.\footnote{Justin A. Williams argues that all of hip-hop's 
substyles deconstruct this notion (see \cite{justinawilliamsBeatsFlowsResponse2009}). I am interested 
in Williams' higher order critique of music theoretical intrigue with repetition in hip-hop, but at 
the same time, I am concerned with seemingly deliberate choice to introduce variety within a musical 
texture constrained to repetition.} In particular, I examine how producers introduce variety primarily
through digital editing techniques that mimic the live improvisatory roots of the genre within a broadly
repetitive musical texture. I do so by building upon two distinct styles of transcription\textemdash
Kajikawa's breakbeat  transcriptions\footnote{\textit{Cf.} \cite{lorenkajikawaSoundingRaceRap2015}, 29-30 
and 36-37.} and Williams' basic beat transcriptions.\footnote{\textit{Cf.}
\cite{justinawilliamsRhyminStealinMusical2013}, 61ff.} Based on these transcriptions, I assign four 
terms to methods producers use to affect variety: recomposing, choking, glitching, and slipping. While 
this list of methods does not exhaust the ways in which producers sound the underground, it demonstrates 
the hip-hop beat as a space for co-creation of variety with the rapper and thus alternative identity 
within hip-hop.

Building off this notion, my second chapter examines the role of the emcee as a composer 
adding heterogeneity to the rap song. Although lyrics offer the clearest method for transmitting 
identity, this chapter focuses on emcees' non-textual methods of ``[amplifying] and in some cases
[transforming] the information that listeners receive'' through texted and visual
communication.\footnote{\cite{lorenkajikawaSoundingRaceRap2015}, 12.} As in my previous chapter, I 
employ two kinds of transcription to examine distinct musical qualities. With standard notation, I 
note emcees' uses of pitched and rhythmic motives, in addition to the text's interplay with elements 
of the beat. I also use poetic scansion to investigate meter, rhyme scheme, and verse form within 
rap verses. Compared to transcription in score form, this method of analysis reflects how rappers 
compose verses, allowing me to consider phrase and syntax from a perspective that traditional 
music-theoretical models cannot.\footnote{In particular, I am interested in comparing the score-forward
methodologies used by Kyle Adams, Robert Komaniecki, and Ben Duinker with more lyric-abstracted 
modes of accessing the texted elements in hip-hop by Kajikawa and Tricia Rose.}