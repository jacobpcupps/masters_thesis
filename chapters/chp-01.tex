\onehalfspacing
\label{chapter1}
\epigraph{{``Listening correctly can cause introspecting.''}}{R.A.P. Ferreira}

\section{Open Mike Eagle and Nocando's ``Unapologetic''}

Before landing several projects on best-of-the-decade lists from outlets like \textit{Pitchfork},
\textit{Brooklyn Vegan}, and \textit{Hip-Hop Golden Age}, the Los Angeles based rapper Open Mike
Eagle started off the decade with a declaration. His first studio LP, \textit{Unapologetic Art Rap},
coined the subgenre art rap, which Eagle calls ``a shorthand for leftfield and avant-garde rap music''
that he ``has spent his career redefining and expanding.''\footnote{
    \cite{openmikeeagle}.}
And not just Eagle himself, but so too American hip-hop fandom and criticism around him: in 2020,
the former Editor-in-Chief of \textit{Rap Genius} Shawn Setaro claimed that: ``[art rap] is a 
movement that has become one of the most vital things happening in hip-hop today. Boundaries are
stretched constantly, in a way that recalls the innovations that took place in rap's golden age, 
without aping that era's  sonics.''\footnote{
    \cite{shawnsetaroWhyYouShould2020}.}

If Setaro's claim holds merit, we should interrogate how the music being received as art rap comes
to be known in that way\textemdash not just in its idiomatic ``sonics'' but also its themes,
techniques, and approaches. Max Bell attempted this by digging through Bandcamp's art rap genre
tag for \textit{Bandcamp  Daily}. After purporting that it ``is antithetical to terrestrial radio 
station playlists,'' he notes a few unifying stylistic traits ``including but not limited to 
left-field, forward thinking production, unconventional song structures and cadences, songs written
from the perspectives of fictional characters, explicit and protracted engagement with social and 
political issues, and absurdist metaphors and similes.''\footnote{
    \cite{maxbellWalkAvantGardeWorld2017}.} 
Bell's attempt at a definition is somewhat helpful, but its breadth offers only a partial picture, 
especially due to the qualifications it must make.\footnote{
    Bell mentions that Kendrick Lamar's work, both commercially successful and radio friendly,
    still ``pushes the boundary of rap.'' After listing the sonic qualities he identified, he
    also states that ``labelling a song/album `art rap' doesn't mean that it's \textit{only}
    that.''} 
It seems, more than a decade later, the hip-hop community is still asking the question posed by
the seventh track on Eagle's \textit{Unapologetic Art Rap}: ``WTF is Art Rap?''

Eagle and the guest emcee Nocando offer a glimpse at a definition within the lyrical content of
the eighth track on the LP, ``Unapologetic.'' As art-rappers, Nocando claims that they are 
small-time enough to identify with an underground scene:
\settowidth{\versewidth}{I dress nice but I ain't no goddamned sneaker freak.}
    \begin{verse}[\versewidth]
        \small I'm underground, man, like I'm beneath the streets, \\ 
        \small I dress nice but I ain't no goddamned sneaker freak.\footnote{
            \cite{openmikeeagle2010}.}
    \end{verse}
At the same time, the two have more caché than the ``rapper friends''  of the local  
``scenesters'':
\settowidth{\versewidth}{There's a reason I'm performing after him.}
    \begin{verse}[\versewidth]
        \small Now everybody's got a rapper friend, \\ 
        \small But my name's higher on the flyer,\\
        \small There's a reason I'm performing after him.
    \end{verse}
On Eagle's end, he envisions art rap as dedicated to a particular audience:
\settowidth{\versewidth}{Who make sketches and daydream the whole class.}
    \begin{verse}[\versewidth]
        \small This to all the adolescent negro lads, \\
        \small Who make sketches and daydream the whole class. \\
        \small In `96, they would have been De La Soul Fans, \\
        \small (In 2010), it's My Chemical Romance.
    \end{verse}
However, he is troubled by a whiter (and perhaps less-initiated) audience his art rap 
parties have attracted:
\settowidth{\versewidth}{Came to watch us battle and cackle like Fran Drescher.}
    \begin{verse}[\versewidth]
        \small Scene's built on the backs of black rappers, \\
        \small (Somehow) when I'm asking for cash you can't answer. \\ 
        \small Descendants of the folks that shackled my ancestors, \\ 
        \small Came to watch us battle and cackle like Fran Drescher.
    \end{verse}
Although keeping his own financial success in mind, Eagle condemns another lucrative type of rap
music that sells itself over to the entertainment industry:
\settowidth{\versewidth}{Cause my little brother never heard of Little Brother \textellipsis}
    \begin{verse}[\versewidth]
        \small Cause my little brother never heard of Little Brother \textellipsis \\ 
        \small The only [rappers] he can discover \\
        \small Are the ones who please Viacom's executive n----- lovers.
    \end{verse}

Even from a brief lyrical analysis, ``Unapologetic'' creates a complex picture of the black,
masculine identities of the emcees on the track.\footnote{
    To say \emph{complex} here is not to deny that the lyrics of ``Unapologetic'' contain
    oversights. One that comes to mind is its misogynistic, hand-waving attitude toward
    ``drunk party girls.'' Part of the complexity I note is that Eagle is comfortable holding
    this attitude while, in the song's outro, condemning misogyny in the lyrics of Soulja
    Boy\textemdash no doubt one of the rappers he has envisioning a Viacom-friendly rap music.} 
The two communicate these lyrically, but text alone does not construct identity in rap: musical
content plays a role as well. This chapter considers ``Unapologetic''  as an early manifestation
of the music the thesis overall intends to address. Although (as my title suggests) I will eventually
come to contextualize the track within a subgenre I call underground, I first consider its relationship
to some categories in rap which have received attention in recent music scholarship: namely, Amanda
Sewell's conception of the ``nerdcore'' subgenre, and Adam Krims  and Justin Williams ``jazz/bohemian'' 
rap  constructions. I reject these designations for  ``Unapologetic'' because the song's themes and 
sounds cannot be contextualized within one style, so, instead, I look towards listening practices 
in hip-hop to create definition of underground that will serve as a guide for the music that this
thesis explores.

\section{``Unapologetic''\textemdash Nerdcore or Jazz Rap?} \label{unapologeticnerdcorejazz}

The lyrical content, themes, and sonic palette of ``Unapologetic'' suggest a number of connections 
to genres which have been previously theorized in music studies. Its beat and lyrical focus on 
intelligence place it within a lineage of nerdcore hip-hop, a less commercial type of hip-hop from
the 1990s and early 2000s. Additionally, its thematic focus on the artistry of hip-hop suggests a
connection to jazz rap of the early-to-mid nineties, which also dealt  with similar topics. In this
section, I consider  whether ``Unapologetic'' satisfactorily fits within one generic category or 
the other.

Nerdcore, according to Amanda Sewell, ``happens when technologically savvy, verbally precocious, and \sloppy
socially marginalized people begin to make hip-hop using their skill sets and 
experiences.''\footnote{
    \autocite[223]{amandasewellNerdcoreHiphop2015}.} 
Drawing on discussions with MC Frontalot, a pioneer of nerdcore, she describes a few core elements
of the  the genre. One, more superficially, is lyrical reference: nerdcore rappers often discuss
``nerdy'' media  (sci-fi, fantasy, comics, anime) or ``nerdy'' career paths (astrophysics, computer
engineering, or genetics.) More substantially, nerdcore emcees also rap about the hardships shared
by young men who can not ``fit into the traditional expectations of masculinity, especially as a 
teenager.''\footnote{
    \autocite[225]{amandasewellNerdcoreHiphop2015}.} 

``Unapologetic'' thus presents itself as nerdcore by drawing on these topics. Both the track and the
LP it is housed on share a grab-bag of references with nerdcore. On Eagle's liner notes video for ``Art
Rap Party,'' he describes such a party as a space where ``of course, it's gonna be a little nerdy[.]
\textellipsis We could talk about anime, \textellipsis political stuff, exploring realms conversationally
that aren't in the typical rap song.''\footnote{
    \cite{openmikeeagleVideoLinerNotes2010}.}
Eagle also positions art rap at the periphery of the genre\textemdash in a space where ``intelligent
folks go to dig on intelligent music.''\footnote{
    \cite{openmikeeagleVideoLinerNotes2010}.} 
Eagle includes young black men in this narrative, those who may not identify with scripts for masculinity
that are highlighted in more commercial rap. The ``adolescent negro lads'' Eagle dedicates ``Unapologetic''
to  are inclined  to ``rebellin' since it's the first sign of intelligence,'' and his similarly intelligent
music may speak to that.

One final element of ``Unapologetic'' reaches my ear as nerdcore although it is not articulated in Sewell's
definition. The primary harmonic content of its beat is constructed from a bitcrushed synthesizer, outlining
diatonic harmony in G major. Along with the two-bar drum loop, it does not create a complex musical texture,
but it does create one that carries meaning for the listener. More superficially, the timbre evokes some 
connection to video game (a stereotypical nerdy past time) soundtracks from the 1980s, 1990s, and early 
2000s, which frequently feature eight-bit soundtracks.\footnote{
    To my ear, the synthesizer in this track plays a bit on \emph{chiptune}, ``a collection of related
    music production and performance practice sharing a history with video game soundtracks'' (see
    \cite{kevindriscollEndlessLoopBrief2009}.}
Additionally, I believe the changes in timbre between  the two loops (e.g., 0:00--0:04 and 0:51--1:01) ask
listeners to engage with the producer as ``technologically savvy''\textemdash perhaps then, too, the emcees.

Nerdcore helps contextualize notions of intelligence and types of sounds on ``Unapologetic,'' but it does
not account for the track in totality, especially because nerdcore is ``almost all white, middle-class, 
and [suburban].''\footnote{
    \autocite[223]{amandasewellNerdcoreHiphop2015}.} 
Although ``Unapologetic'' may be rapped by emcees with nerdy proclivities, the track communicates more
about their identities than this. Taking them at their work, Eagle and Nocando are not just outsiders 
due to their nerdinesss but also because they are \emph{artists}, working on the outskirts of the rap
landscape.  In this way, ``Unapologetic'' reflects ideologies that helped to construct the ethos of 
jazz rap (or bohemian rap) that emerged during hip-hop's golden age.

Krims developed his genre system for rap music by examining a number of distinct subcategories present
in the rap music industry by 1994. In particular, he cites two\textemdash in his parlance, jazz/bohemian
rap and reality rap\textemdash that articulated themselves as alternatives to gangsta rap, a style that 
was closely aligned with the commercial mainstream.\footnote{
    \autocite[64--65]{adamkrimsRapMusicPoetics2000}. Krims offers two case studies\textemdash The Roots' 
    ``What They Do'' and RZA's introduction on Wu-Tang Clan's \textit{Wu-Tang Forever}\textemdash as 
    instances of each genre directly addressing what it conceived to be the commercial mainstream at
    the time.}
In the case of the former, jazz became a sonic profile closely associated with alternative identity:
A Tribe Called Quest, Digable Planets, and The Roots all used jazz, either via sampling or performed
live, as a sonic marker of this style.

Williams heightens the connection between jazz and artistry in mid-nineties hip-hop. He argues that 
the above groups used jazz and other black sonic profiles as a method of constructing their music as
a high art taken with respect to the industry. They drew on certain musical elements that functioned
as \emph{jazz codes}: signifiers in the instrumentation that reach audiences as having a jazz feel,
even if their sound source is not a jazz record specifically.\footnote{
    \autocite[55]{justinawilliamsRhyminStealinMusical2013}.}
That contemporaneous audiences appreciated these sounds as art, he argues, was precipitated by two 
elements: (1) through the efforts of musicians like Wynton Marsalis during the 1980s, jazz had become
an institutional and  ``serious'' art form,\footnote{
    \autocite[48--52]{justinawilliamsRhyminStealinMusical2013}. The directionality of Williams' claim 
    is important to note here. Jazz-rappers did not use these sounds because thought of jazz as a serious
    art form; more likely, jazz was music that permeated family record collections. Rather, because of 
    the position jazz held within mainstream American culture and these rappers articulated a 
    mainstream/underground divide textually, he claims jazz \emph{sounded as} art to the hip-hop 
    community.} 
and (2) jazz-rappers coupled these sonic borrowings with lyrical and visual significations of their 
own alternative identity.\footnote{
    \autocite[55--64]{justinawilliamsRhyminStealinMusical2013}.}
Audiences were therefore poised to experience the music as a sort of artistic rap substyle.

Eagle's art rap operates using a similar discourse, one that suggests ``Unapologetic'' may be a new
type of jazz rap. Although the track does not draw on the same jazz codes mentioned above, many tracks 
by artists working in a similar vein do; jazz and other black sonic profiles function as a sonic 
shorthand for alternative identity in the 2010s like they did in the 1990s.\footnote{
    In the appendix, I have collected a few tracks by artists whose music I engage with throughout 
    this project that either sample or interpolate lyrical content from pieces that exhibit jazz codes 
    (see Table~\ref{tab:jazz_references}.)}
Eagle also makes reference to jazz rap when rapping about his audience: he explicitly says, if born
a  bit earlier, ``they would have been De La Soul Fans.'' In his estimation, the ideal demographic
for art  rap overlaps with the fan-base of jazz rap, separated by a generation. Finally, indeed most
importantly, art rap reinscribes the same divisions within the rap industry that served as the thematic
focus in jazz rap. Eagle's criticism of the types of artists that ``please Viacom's [executives]'' and
his  estimation of his own work as a ``black man's art'' allow him to construct a similar dichotomy between
him and popular rappers in his time that A Tribe Called Quest's Q-Tip did between himself and popular
rappers in the 1990s.\footnote{
    Williams cites the ``new jack swing'' movement in R\&B as the object of jazz rap's scorn, and the
    lyrics Tribe's emcees devote to discussing it bear some resemblance to Eagle and Nocando's rapping
    (see \autocite[56]{justinawilliamsRhyminStealinMusical2013}.)}
Both art and jazz rappers make themselves out to be bohemians within the music industry; thus, according
to these artists, the listeners who can appreciate each style are connoisseurs amongst the listening 
audience of rap music.

As with nerdcore, jazz rap speaks to matters of genre in ``Unapologetic'' but likewise does not completely
explain the track. Even if it shares an ideology with jazz rap, the sonic profile of ``Unapologetic'' does
not fit neatly into the generic category Krims and Williams construct. Granted, each author acknowledges the
possibility for a stylistic eclecticism that reaches beyond jazz sonically; however, the groups that receive
their focus (A Tribe Called Quest, De La Soul, Digable Planets, Jungle Brothers, and The Roots) all rely to
\emph{some} degree on the sound of jazz in their music. Without jazz codes to signify generic construction,
``Unapologetic'' cannot reach listeners as jazz rap\textemdash at least not by the same methods that 
golden-age jazz rappers did.

Eagle's ``Unapologetic'' and his conception of art rap as a moment at the start of the 2010s articulate something
that is at once new in hip-hop while relating to ideas, sounds, and genres from throughout hip-hop's history. 
Although the track's ``tech-y'' sound and ideas about intelligence intimate some connections to nerdcore, the
designation falls short because of nerdcore's historically white practitioners and lack of concern with being 
received as art. On the other hand, to situate ``Unapologetic'' within the lineage of jazz rap does not adequately
account for the track's genre codes but more closely approximates the topical focus of art rap, as well as the 
positionality it affords itself with respect to the mainstream.

\section{Hip-Hop Practice and Listening as Meditation} \label{listeningasmeditation}

In the previous section, I argue that ``Unapologetic'' is characterized not by one sonic style like
nerdcore or jazz rap. Instead, I argue that the track is characterized by a practice of listening,
a type of sonic engagement I will term \emph{listening as meditation}. Throughout this thesis, I aim
to demonstrate that contemporary underground hip-hop is closely associated with this style of engagement,
so I will define and historicize it in this section. Listening as meditation is abundant in all historical 
stages of the making of any kind of hip-hop\textemdash from its conception in the mind of the producer
and emcee, to its performance live or in studio, and eventually its reception by the listener. 
Practitioners throughout hip-hop's history (including Eagle himself) promote this style of listening,
one that engenders a deep immersion in hip-hop as a musical form.

The phrase listening as meditation inverts the mind-body dualism invoked by earlier popular music
discourses. To be sure, by privileging listening as a unifying thread amongst a wide variety of
stylistic manifestations, I am inviting a tacit definition of the underground as \emph{hip-hop to
be listened to} as distinct from \emph{to be danced to} or \emph{merely heard.}\footnote{
    My focus on listening runs the risk of positioning underground hip-hop as a ``transmission of 
    mind--mind messages'' between composer and listener (see 
    \autocite[20]{suzanneg.cusickFeministTheoryMusic1994}.) 
    While this style of inquiry still characterizes a great deal of music-theoretical scholarship,
    my reservations arise from the potential for erasing the embodied experience of performers 
    (who, in this repertory, tend to be the composers as well) and listeners within the exchange 
    that happens in underground hip-hop. I therefore intend with my term for this mode of listening
    (\emph{as meditation}) to be interpreted as a unified mind-body practice.}
Part of the trouble with such a definition is that similar justifications have been levied for excluding
music traditions from being objects of critical inquiry. My choice to prioritize discussing this music 
as an engagement of one's musical intellect does not exclude its use as dance music, nor as a music that
invites other embodied responses. The metaphor of listening as meditation, then, proposes a unification
of cognitive and embodied responses that characterize engagement with underground hip-hop. In short, it is
an argument that will, in the coming pages, put this chapter's epigraph to the test.

At each point in hip-hop's creative processes, and indeed throughout its history as a musical form, several 
practitioners have evinced that their listening exhibits a meditative engagement with music. The rapper-producer
Count Bass D suggests that this style of listening guided his sample selection in the introduction to his 2002
track ``Truth to Light'': 
    \begin{quote}
        \small Songs that really I like a whole lot, that I've liked over the years, kind of run through my head all
        the time and so they kind of creep into songs. \textellipsis Unless you know [Nice \& Smooth's ``Funky
        for You''], you don't know who I'm talking about or what I'm talking about, but to people who are in
        the know, it strengthens their faith that the things I'm talking about that they don't understand may
        have some relevance to them in time.\footnote{
            \autocite[100]{mickeyhessHipHopDead2007}.}
    \end{quote}
Count focuses on familiarity, being ``in the know'' with the source material, as a requirement for parsing 
his dialogue with a previous recording of Nice \& Smooth at the beginning of the track, but his discussion also
reveals listening norms within hip-hop. First, Count engaged with ``Funky for You'' regularly enough over a long
period of time that his decision to dialogue with the artists' ad-libs on the original recording felt subliminal
and ``[crept] into'' the track. Second, he expects that his audience will have engaged with the material on a 
similar level, enough to ``know what [he's] talking about.'' Finally, Count conceives of a \emph{type} of listener,
who may not know the source material but, as a result of their familiarity with the listening norms, trusts its 
meaning will come to them ``in time'' as they listen further and more broadly beyond ``Truth to Light.'' Such a 
listener is the \emph{meditative} listener, the type of listener Count is and that makes up his listenership.

For the emcee, meditative listening parallels a producer's engagement; they listen with enough regularity and
focus that imitating other artists can become subconscious and in some cases problematic for their on writing
process. Eagle explains that, because of his listening interactions with the emcee MF DOOM's flow, ``he has to
be careful \textellipsis because [he] can almost get into [DOOM's] mind in terms of how he writes.''\footnote{
    \cite{estellecaswellRappingDeconstructedBest2016}.} 
Eagle characterizes listening to DOOM with such frequency, regularity, and depth that it would be easy for him
to write a verse in DOOM's style that would be more derivative than he would like. Again, the hip-hop practitioner
positions their compositional choices as subconscious, based on deep engagement with models for composing that
show up in their own listening. For Eagle as an emcee, this listening occurred so often that he subsumed DOOM's
style of flow into his own.

Finally, hip-hop practitioners show that listening as meditation informs their interpretation of performances 
by other practitioners. They envision the artists they interact with as meditative listeners, who then serve
as guides for further meditative listening. The Bronx DJ Grand Wizzard Theodore describes, when hearing a 
record through DJ Kool Herc's soundsystem, it, ``made you appreciate the record even more. He would play 
[something] that you listened to every day and you would be like `Wow, that record has \emph{bells} in it?' 
It's like you heard instruments in the record that you never thought [it] even had.''\footnote{
    \autocite[139]{christabronGlassHiphopProduction2015}.} 
Theodore's description shows that, even within the context of a block party, hip-hop can be listened to
in a way that brings about new depths to the music beyond the standard forms of engagement.

\section{Definitions and Limitations of Underground} \label{undergrounddeflims}

If Eagle's listening is meditative, and no other style or subgenre accounts for all of the lyrical, thematic,
and musical elements in ``Unapologetic,'' then another term is required to describe the track's subgeneric
construction. Rather than his own art rap, I argue the designation \emph{underground} is appropriate, for
underground hip-hop is typified by this style of meditative listening. Fans of underground hip-hop are 
encouraged to engage with the subgenre  as meditative listeners because underground hip-hop musicians exhibit
meditative listening in their music (re-articulating the practice, too, in interviews). As a result of the 
perceived closeness to the artist that characterizes the underground, listeners choose to engage with the music
in a similarly meditative fashion.

Before I arrive at this definition, I need to address three conceptions of the underground which I believe to
be faulty and therefore do not wish to import in it. Specifically, the concept of underground cannot rely on 
discourses of artistry, authenticity, or commerciality. These three metrics permeate discussions of what makes
music underground, so addressing them out front will prove useful in elucidating what I wish to focus on: namely,
listening practice and methods of musical construction.

Defining underground as an art genre invokes metrics of a work's beauty and requires asking
questions at once too broad to answer and too based in subjective experience to communicate
effectively. Although rappers like Eagle have constructed their personas around artistry, and 
fans buy into such constructions, I find it telling that such discourses are often short-lived.
As early as 2013, Eagle raps:
\settowidth{\versewidth}{I used to throw these sensitive parties for art rap,}
    \begin{verse}[\versewidth]
        \small I used to throw these sensitive parties for art rap, \\
        \small No regrets, but I was foolish to start that. \\
        \small Sophisticated fuckers left a bitch of a bar tab, \\
        \small And now we just throw pizza parties.\footnote{
        \cite{milo2013}.}
    \end{verse}
Eagle's indictment of the art rap scene followed quickly on the heels of his self-identifying
as an art-rapper, during a feature verse only a few projects later than he unapologetically
codified art rap. Though his lyrics focus on the pretension and bourgeois tastes of the fans
the term garnered him, I also  wonder if writing ``artful'' rap lost its meaning. As a listener,
I find it hard to describe whether something reaches me artfully; for instance, I do not, as
Justin Williams suggests, hear a piece based around a jazz sample as more artistic than, say,
something constructed from DAW-based synthesizers simply because jazz's standing in contemporary
American culture. Rap music's artistry seems less quantifiable (and more personally defined) 
than its undergroundness.

Defining underground as an authentic music proves to be slippery as well. Like narratives of artistry, authenticity
finds purchase in popular discourse. This is  because artists tie music they deem authentic closely to the early,
pre-recorded era of hip-hop. Fab 5 Freddy summarized this mentality when discussing  his role as the writer for 
the 1982 film \textit{Wild Style}. The writers set the story before the 1979 release of Sugarhill Gang's single
``Rapper's Delight'' because ``[they] wanted to go back to a few years earlier\textellipsis \emph{when hip-hop was 
completely underground, when the form was raw and pure.}''\footnote{
    \autocite[23]{justinawilliamsRhyminStealinMusical2013}. Emphasis my own. The release
    of ``Rapper's Delight'' is often thought of as the moment when hip-hop went commercial.}
His association of undergroundness with rawness or purity illustrates an attitude that listeners, critics, and fans
alike share about of hip-hop, yet, once again, I wonder if the terms have any bearing on the sound of a commercialized
music form decades later. As I will show in oncoming chapters, underground artists use the recorded medium to communicate
their identity to audiences, a necessarily commercialized endeavor. Discourses of authenticity therefore seem to be based
on subjective, evaluative judgements a listener makes.

Defining underground strictly as an anti-commercial music may seem valuable intuitively, but 
modern methods for discovering and participating in cultures of listening problematize such a
conception. A lack of commerciality may very well tie to the perceived authenticity of a rap
song. However, commercial success is not only metric considered in undergroundness. As Anthony
Kwame Harrison argues in his ethnographic foray into the San Francisco underground hip-hop scene, 
``the label `underground' can be applied to everything from Grammy-nominated artists like Common 
and the Roots to groups like The Latter, who once boasted(!) of having sold only two copies of one
of their CDs.''\footnote{
    \autocite[9]{anthonykwameharrisonHipHopUnderground2009}.}
Harrison's insight gels with observations by Kegs One, the proprietor of the San  Francisco 
hip-hop shop Below the Surface. Kegs speaks to degrees of commerciality that exist within 
underground as a record-bin category:
    \begin{quote}
        \small I have a few different people that are into the more `commercial underground' stuff, 
        but then 95 percent of my customers are here for the  literal four-track tapes. You 
        know,  dirty-sounding, low-budget, in-the-room, in-the-closet recorded 
        tapes.\footnote{
            Quoted in \autocite[10]{anthonykwameharrisonHipHopUnderground2009}.}
    \end{quote}
Each point illustrates that several levels of success can all maintain underground as a quality:
all the way from The Latter's two records sold, to the more successful Bay-area artists who are
``just below the surface,'' on up to  musicians like The Roots, who signed to Def-Jam in 2006 
and  have been Jimmy Fallon's in-house band since 2009. For acts to have varying degrees of success 
and still be considered underground points to some metric beyond commerciality being invoked to 
define the quality of undergroundness.

Harrison's definition of underground comes from an ethos he observes in the Bay-area scene, and his 
observations support my own definition based in listening practices. In particular, he notes
``the idea of a blurring, thinning, almost imperceptible line that separates artists and fans'' and
``an intimacy of the fan bases [underground artists] tend to attract.''\footnote{
    \autocite[10--11]{anthonykwameharrisonHipHopUnderground2009}.}
Although, like Harrison, I believe this ethos decreases in palpability as an artist's commercial 
success and notoriety increases, these artists are still stylistic participants in underground 
hip-hop because this ethos sounds in the music. I connect this intimacy to the way an underground
artist models meditative listening for fans; in effect, the music shows the artist acting as a 
listener (as a \emph{fan} themselves), and fans experience a type of intimacy in that recognition.

This perception of intimacy in acts of meditative listening is core to understanding how underground
is sounded in hip-hop music. Underground is not codified by one sonic practice, nor by one thematic
focus, but rather by a multiplicity of both, guided by musicians demonstrating their relation to a 
network of practices and focuses. Underground hip-hop artists that bring together a multiplicity of
experiences in their work model a listening practice for the fan, who may at first intentionally re-navigate
these listening networks for themselves or may\textemdash in time, as Count Bass D suggests\textemdash come
to trust that these networks will re-manifest in their own paths of musical experience.


\section{Outline of the Thesis}

Defining the subgenre underground as a practice of meditative listening shared by fans and artists,
this thesis focuses on transcribing and analyzing a repertory of underground rap songs, examining
how the quality of undergroundness manifests within beats and flows. Avoiding limiting underground
to one sonic style allows me to examine how, despite the stylistic setting, certain techniques
communicate meaning to a listener; this proves to be helpful when the lyrics in a song speak less
explicitly to identity than ``Unapologetic.'' In short, I investigate in how musical texts sound
the underground, agnostic of any stylistic approach to music making.

My reading of musical texts agentially is bound up in Loren Kajikawa's notion of \emph{sounding} 
in rap music. Kajikawa argues that the rap song is a medium for transmitting identities and meaning 
between the artist and the listener.\footnote{
    \autocite[2]{lorenkajikawaSoundingRaceRap2015}.}
Where his scholarship focuses on the encoding of racial and gender identities, mine will focus on
another matter: how an artist navigates mainstream expectations about these identities, and 
in doing so invites a more meditative engagement their navigation.

The two primary chapters in this project interrogate the overlapping yet distinct methods by which
hip-hop musicians sound the underground. Chapter~\ref{chapter2} focuses on beatmaking, arguing that
producers create underground hip-hop beats by avoiding fixed samples/loops and normative forms. I 
examine how producers introduce variety primarily through digital editing techniques that mimic the
live, improvisatory roots of the genre within a broadly repetitive musical texture. My beat 
transcriptions come in two forms: tablular notation of larger formal sections and standard notation
of snapshots of the musical texture. From these transcriptions, I identify four methods of alteration
producers use to introduce musical variety: resampling/recomposing, choking, glitching, and slipping.
Although the list does not exhaust the methods producers use sound the underground, it demonstrates
how a hip-hop beat functions as a space for co-creation of alternative identity with an emcee.

Chapter~\ref{chapter3} examines the underground emcee's role in identity construction by studying 
techniques used in flows, or rapped verses. Lyrics occasionally offer information about identity,
but this chaper focuses on emcees' non-textual methods of ``[amplifying] and in some cases [transforming]
the information that listeners receive'' through textual and visual communication.\footnote{
    \autocite[12]{lorenkajikawaSoundingRaceRap2015}.} 
As in my previous chapter, I employ a few styles of transcription to examine different musical
elements. When examining flow's structure\textemdash its metrical patterns, rhyme schemes, and 
linguistic-syntactic units\textemdash I use Mitchell Ohriner's system for visualizing flow, which
avoids standard notation.\footnote{
    For a sample and detailed overview of Ohriner's notational method, see 
    \autocite[xxvii--xl]{mitchellohrinerFlowRhythmicVoice2019}.}
By constrast, when I examine flow's performance\textemdash its manifestation as a musical object\textemdash
I opt to transcribe using standard notation, mindful of its limited ability to denote the complexities
of rap perfromance. Across this performative-structural divide, I draw out four more methods underground
emcees use in structuring and performing flow: pivot rhyme, closing fragmentation, mimesis, and processing.
Each of these techniques indicate an underground emcees flow exists in dialogue with listener expectations 
and challenges the notion that the rapping voice is hierarchically most significant in the musical texture
of a rap song.