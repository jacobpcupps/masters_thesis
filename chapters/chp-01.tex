%\onehalfspacing

\epigraph{{``Listening correctly can cause introspecting''}}{R.A.P. Ferreira}

\section{Open Mike Eagle and Nocando's ``Unapologetic''}

\begin{itemize}
    \item Declaration of a place within a ``scene'' ; declaration against a ``status quo''
    \item OME focuses on theorizing Art Rap -- who the audience is, why it's needed
    \item Why \emph{he} is rapping: ``Cause my little brother never heard of Little Brother 
    /\textellipsis~the only [rappers] he can discover / are the one's who please Viacom's [executives]''
\end{itemize}

\section{Nerdcore, Jazz-Rap, and the ``Golden Age''}
    \begin{itemize}
        \item The art rap moment seems to re-narrate a lot of the discourse of nerdcore hip-hop (Sewell, 2015):
            \begin{itemize}
                \item ``Nerdcore hip-hop happens when technologically savvy, verbally precocious, and 
                socially marginalized people begin to make hip-hop using their skill sets and experiences.'' 
                (223)
                \item Historically ``white, middle-class, and [suburban]'' (ibid); OME seems interested in 
                re-narrating this for young Black men
                \item MC Frontalot ``[As a nerd, you] don't fit into the traditional expectations of 
                masculinity, especially as a teenager'' (226). Compare to OME's ``adolescent negro lads who 
                daydream the whole class''
            \end{itemize}
        \item OME's ``Art Rap Party'' liner notes video:
            \begin{itemize}
                \item ``My idea of an art rap party, which is somewhat imaginary and hasn't all the way
                happened yet, is the kind of function where, like, intelligent folks go to dig on some intelligent music, \textellipsis it's not a dance party, you know what I mean?''
                
                \item ``  It's the kind of party where information's exchanged between people[;] 
                information's exchanged between performer and audience, and back and forth'' 
                \item ``Of course it's gonna be a little bit nerdy[.] \textellipsis We could talk about 
                anime, \textellipsis political stuff, exploring realms conversationally that aren't in the 
                typical rap song''
            \end{itemize}
        \item As the 2010s advanced, art and indie rap differentiated itself from nerdcore  turning towards 
        jazz and related Black sonic profiles:
            \begin{itemize}
                \item Table of artists/songs that make reference to Black sonic traditions
                \item (Williams 2015) speaks about another period in time where jazz and other Black sonic 
                traditions were constructed as ``High Art''\textemdash the mid-90s golden era of hip-hop
            \end{itemize}
    \end{itemize}
    
\begin{sidewaystable}[p]
    \centering
    \small
    \begin{tabular}{|c|c|c|c|}
         \hline
        Year & Track & Referent & Type \\ \hline
        2018 & Armmand Hammer -- ``VX'' & Prince Far I -- ``Throw Away Your Gun'' & Sample \\ \hline
        2018 & Armmand Hammer -- ``No Days Off'' & Sun Ra Arkestra -- ``The All of Everything'' & Sample \\ \hline
        2012 & billy woods (ft. Elucid) -- ``Sour Grapes'' & Miles Davis -- ``Pharoah's Dance'' & Sample \\ \hline
        2012 & billy woods -- ``Body of Work'' & Nina Simone -- ``Work Song'' & Sample \\ \hline
        2012 & billy woods -- ``Crocodile Tears'' & Muddy Waters -- ``Champagne \& Reefer'' & Interpolation \\ \hline
        2012 & billy woods -- ``DCMA'' & Junior Murvin -- ``Police and Thieves'' & Interpolation \\ \hline
        2019 & billy woods -- ``Fnu Lnu'' & Hank Crawford -- ``Wildflower'' & Sample \\ \hline
        2010 & Kendrick Lamar -- ``Rigamortus'' & Willie Jones III - ``The Thorn'' & Sample \\ \hline 
        2012 & Kendrick Lamar -- ``Sing About Me, I'm Dying of Thirst'' & Grant Green -- ``Maybe Tomorrow'' & Sample \\ \hline
        2015 & Kendrick Lamar -- ``King Kunta'' & James Brown -- ``The Payback'' & Interpolation \\ \hline
        2017 & Kendrick Lamar ``XXX.'' & James Brown -- ``Get Up Offa That Thing'' & Sample \\ \hline
        2013 & Milo (ft. Busdriver) -- ``The Gus Haynes Cribbage League'' & Quincy Jones (ft. James Ingram) -- ``Just Once'' & Sample \\ \hline
        2015 & Milo (ft. Hemlock Ernst) -- ``Souvenir'' & Shuggie Otis -- ``Rainy Day'' & Sample \\ \hline
        2017 & Milo -- ``Call + Form (Picture)'' & Eddie Munji III -- ``Doon Po Sa Amin'' & Sample \\ \hline
        2017 & Milo (ft. Elucid) -- ``Landscaping'' & Sun Ra -- ``Quiet Ecstasy'' & Sample \\ \hline
        2018 & Milo -- ``Tiptoe'' & Hank Crawford -- ``Teach Me Tonight'' & Sample \\ \hline
        2016 & Scallops Hotel -- ``Niopo Tree Stipend'' & Ella Jenkins -- ``Moon Don't Go'' & Sample \\ \hline
        2016 & Scallops Hotel (ft. SB the Moor) -- ``Lanquidity'' & Sun Ra -- ``Lanqudiity'' & Sample \\ \hline
        2017 & Scallops Hotel -- ``Ain't No Hustle Where I Live'' & Stanley Cowell -- ``Here I Am'' & Sample \\ \hline
        2017 & Scallops Hotel -- ``A Beat for My Lil Boy'' & Sun Ra -- ``Where There Is No Sun'' & Sample \\ \hline
    \end{tabular}
    \caption{References to jazz, soul, and funk pieces in 2010s underground hip-hop.}
    \label{tab:jazz_references}
\end{sidewaystable}

\section{Listening as Meditation}

\begin{itemize}
    \item Count Bass D, on ``Truth to Light'': ``Songs that really I like a whole lot, that I've liked over 
    the years, kind of run through my head all the time and so they kind of creep into songs.\textellipsis 
    [T]o the people who are in the know, I think it strengthens their faith that the things I'm talking about
    that they don't understand may have some relevance to them in time'' (Hess 2006, 287).
    \item OME, on DOOM: ``His flow\textemdash I have to be careful with his flow because his flow lives in 
    my mind and in my heart. I can almost get into his mind in terms of how he writes, you know?'' 
    (Caswell 2016)
    \item DJ GW Theodore, on listening through Herc's soundsystem: ``It made you listen to a record and made
    you appreciate the record even more. He would play a record that you listened to every day and you would
    be like `Wow, that record has \emph{bells} in it?' It's like you heard instruments in the record that you
    never thought the record even had.'' (Tabron 2015, 139).
\end{itemize}

\section{Thesis Statement}

\begin{itemize}
    \item I'm choosing to call this style \emph{underground} as opposed to art rap, indie rap, or the like
    for a few reasons:
        \begin{itemize}
            \item I don't want to import discourses of art/beauty or of authenticity (as much as that can 
            be avoided with how musicians talk about the underground)
            \item I don't want to limit what stylistic or timbral elements ``count'' as underground
            \item The artists I cover seem to have moved away from Art Rap as a moment (OME on Milo's
            ``Otherground Pizza Party'' and Rory himself on ``Twenty on Five''), but at least conceive of
            themselves as related to the underground 
        \end{itemize}
    \item Add these points to ideas from Prospectus:
        \begin{itemize}
            \item If the underground exists, it manifests itself in sonic and semantic (musical and lyrical?)
            dimensions. Principally, I am interested in the former, though obviously acknowledge the latter
            where I can
            \item Close readings via transcription is my primary methodology (I expand later on the
            difficulties of capturing certain musical elements in Standard Western Notation)
        \end{itemize}
\end{itemize}

This thesis focuses on transcribing and analyzing a repertory of rap songs that sound
the quality of ``undergroundness'' within their beats and flows. While the term underground 
conjures up meanings predicated on music discourses of gatekeeping and authenticity, I do 
not intend to explore it as a way of forming some alternative canon within the genre; instead, 
I wish to explore undergroundness as mode in which listeners access the music. I follow Loren 
Kajikawa in arguing that the rap song is a medium for transmitting musical meaning between 
hip-hop musicians and listeners.\footnote{\cite{lorenkajikawaSoundingRaceRap2015}, 2.} Where 
he notes the rap song's ability to code racial and gender identities, I observe a subsequent 
level of coding: the extent to which the song (and thus its creators) accept mainstream narratives 
about those identities.

While rappers can communicate underground identity at the textual level, I contend that undergroundness
permeates the rap song beyond text. Thus, in my repertory, I use transcription to examine methods by which
rappers and producers distinguish their musical performance as underground. In general, I note an attitude
towards music making that is anti-commodification, and aligns this belief with a narrative of `getting 
back' to a more authentic form of hip-hop. Fredrick Brathwaite (aka Fab 5 Freddy) summarized this mentality
when discussing his role as a writer for the 1982 film \textit{Wild Style}. The writers set the story 
before the 1979 release of Sugarhill Gang's single ``Rapper's Delight'' because ``[they] wanted to go back
a few years earlier\textellipsis \emph{when hip-hop was completely underground, when the form was raw and
pure}.''\footnote{Quoted in \cite{justinawilliamsRhyminStealinMusical2013}, 23. Emphasis my own. The release
of ``Rapper's Delight'' is often thought of as the moment when hip-hop went commercial.} Although I am not
interested in assessing undergroundness along the evaluative lines Brathwaite implies, I have observed an
unspoken consensus within the underground that the sound aesthetic of mid-1970s of hip-hop is one to be
replicated.

I believe hip-hop musicians mimic this era not only because of their evaluative judgements about it, but
because doing so positions their music in a lineage of black American music forms; this aesthetic link is
manifest in what Olly Wilson terms the ``heterogeneous sound ideal'' of African American
music.\footnote{\cite{ollywilsonHeterogeneousSoundIdeal1992}: 329.} In particular, underground hip-hop
champions an aesthetic of heterogeneity through its sampling of disparate sources, as well as  through
composing in ways that eschew textural and timbral homogeneity. This compositional style distinguishes the
work as underground, in part because the ability to create more fully-polished styles of hip-hop is available
to anyone with a Digital Audio Workstation. 

\section{Chapter Breakdown}
\begin{itemize}
    \item Chapter 2: 
        \begin{itemize}
            \item Form \& Snapshot Transcriptions
            \item Variety within repetition through 4 techniques I identify
            \item Heterogeneity, diversity within the texture punctuates alternative identity
        \end{itemize}
    \item Chapter 3:
        \begin{itemize}
            \item ??? Transcriptions of Flow
            \item Emcees manipulate elements of structure as lyricists, performance as vocalists
            \item Olly Wilson / Heterogeneity? Tricia Rose? Mitchell Ohriner?
        \end{itemize}
\end{itemize}

The two primary chapters in this project will interrogate the overlapping yet distinct methods
by which hip-hop musicians sound the underground. My first chapter focuses on beatmaking, deconstructing 
the notion of a fixed-loop hip-hop beat.\footnote{Justin A. Williams argues that all of hip-hop's 
substyles deconstruct this notion (see \cite{justinawilliamsBeatsFlowsResponse2009}). I am interested 
in Williams' higher order critique of music theoretical intrigue with repetition in hip-hop, but at 
the same time, I am concerned with seemingly deliberate choice to introduce variety within a musical 
texture constrained to repetition.} In particular, I examine how producers introduce variety primarily
through digital editing techniques that mimic the live improvisatory roots of the genre within a broadly
repetitive musical texture. I do so by building upon two distinct styles of transcription\textemdash
Kajikawa's breakbeat  transcriptions\footnote{\textit{Cf.} \cite{lorenkajikawaSoundingRaceRap2015}, 29-30 
and 36-37.} and Williams' basic beat transcriptions.\footnote{\textit{Cf.} \cite{justinawilliamsRhyminStealinMusical2013}, 61ff.} Based on these transcriptions, I assign four 
terms to methods producers use to affect variety: recomposing, choking, glitching, and slipping. While 
this list of methods does not exhaust the ways in which producers sound the underground, it demonstrates 
the hip-hop beat as a space for co-creation of variety with the rapper and thus alternative identity 
within hip-hop.

Building off this notion, my second chapter examines the role of the emcee as a composer 
adding heterogeneity to the rap song. Although lyrics offer the clearest method for transmitting 
identity, this chapter focuses on emcees' non-textual methods of ``[amplifying] and in some cases
[transforming] the information that listeners receive'' through texted and visual
communication.\footnote{\cite{lorenkajikawaSoundingRaceRap2015}, 12.} As in my previous chapter, I 
employ two kinds of transcription to examine distinct musical qualities. With standard notation, I 
note emcees' uses of pitched and rhythmic motives, in addition to the text's interplay with elements 
of the beat. I also use poetic scansion to investigate meter, rhyme scheme, and verse form within 
rap verses. Compared to transcription in score form, this method of analysis reflects how rappers 
compose verses, allowing me to consider phrase and syntax from a perspective that traditional 
music-theoretical models cannot.\footnote{In particular, I am interested in comparing the score-forward
methodologies used by Kyle Adams, Robert Komaniecki, and Ben Duinker with more lyric-abstracted 
modes of accessing the texted elements in hip-hop by Kajikawa and Tricia Rose.}