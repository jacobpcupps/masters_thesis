\epigraph{{``Really I'm just glad that I got the rights to my \\ masters / and \emph{y'all} called it art rap\textellipsis''}}{Scallops Hotel, ``Twenty on Five''}

%Rap music is not a monolith.

%\lipsum[1-3]

%\section{Thesis Statement}
This thesis focuses on transcribing and analyzing a repertory of rap songs that sound the quality of ``undergroundness'' within their beats and flows. While the term underground conjures up meanings predicated on music discourses of gatekeeping and authenticity, I do not intend to explore it as a way of forming some alternative canon within the genre; instead, I wish to explore undergroundness as mode in which listeners access the music. I follow Loren Kajikawa in arguing that the rap song is a medium for transmitting musical meaning between hip-hop musicians and listeners.\footnote{\cite{lorenkajikawaSoundingRaceRap2015}, 2.} Where he notes the rap song's ability to code racial and gender identities, I observe a subsequent level of coding: the extent to which the song (and thus its creators) accept mainstream narratives about those identities.

While rappers can communicate underground identity at the textual level, I contend that undergroundness permeates the rap song beyond text. Thus, in my repertory, I use transcription to examine methods by which rappers and producers distinguish their musical performance as underground. In general, I note an attitude towards music making that is anti-commodification, and aligns this belief with a narrative of `getting back' to a more authentic form of hip-hop. Fredrick Brathwaite (aka Fab 5 Freddy) summarized this mentality when discussing his role as a writer for the 1982 film \textit{Wild Style}. The writers set the story before the 1979 release of Sugarhill Gang's single ``Rapper's Delight'' because ``[they] wanted to go back a few years earlier\textellipsis \emph{when hip-hop was completely underground, when the form was raw and pure}.''\footnote{Quoted in \cite{justinawilliamsRhyminStealinMusical2013}, 23. Emphasis my own. The release of ``Rapper's Delight'' is often thought of as the moment when hip-hop went commercial.} Although I am not interested in assessing undergroundness along the evaluative lines Brathwaite implies, I have observed an unspoken consensus within the underground that the sound aesthetic of mid-1970s of hip-hop is one to be replicated.

I believe hip-hop musicians mimic this era not only because of their evaluative judgements about it, but because doing so positions their music in a lineage of black American music forms; this aesthetic link is manifest in what Olly Wilson terms the ``heterogeneous sound ideal'' of African American music.\footnote{\cite{ollywilsonHeterogeneousSoundIdeal1992}: 329.} In particular, underground hip-hop champions an aesthetic of heterogeneity through its sampling of disparate sources, as well as  through composing in ways that eschew textural and timbral homogeneity. This compositional style distinguishes the work as underground, in part because the ability to create more fully-polished styles of hip-hop is available to anyone with a Digital Audio Workstation. 

\section{Chapter Breakdown}
The two primary chapters in this project will interrogate the overlapping yet distinct methods by which hip-hop musicians sound the underground. My first chapter focuses on beatmaking, deconstructing the notion of a fixed-loop hip-hop beat.\footnote{Justin A. Williams argues that all of hip-hop's substyles deconstruct this notion (see \cite{justinawilliamsBeatsFlowsResponse2009}). I am interested in Williams' higher order critique of music theoretical intrigue with repetition in hip-hop, but at the same time, I am concerned with seemingly deliberate choice to introduce variety within a musical texture constrained to repetition.} In particular, I examine how producers introduce variety primarily through digital editing techniques that mimic the live improvisatory roots of the genre within a broadly repetitive musical texture. I do so by building upon two distinct styles of transcription\textemdash Kajikawa's breakbeat  transcriptions\footnote{\textit{Cf.} \cite{lorenkajikawaSoundingRaceRap2015}, 29-30 and 36-37.} and Williams' basic beat transcriptions.\footnote{\textit{Cf.} \cite{justinawilliamsRhyminStealinMusical2013}, 61ff.} Based on these transcriptions, I assign four terms to methods producers use to affect variety: recomposing, choking, glitching, and slipping. While this list of methods does not exhaust the ways in which producers sound the underground, it demonstrates the hip-hop beat as a space for co-creation of variety with the rapper and thus alternative identity within hip-hop.

Building off this notion, my second chapter examines the role of the emcee as a composer adding heterogeneity to the rap song. Although lyrics offer the clearest method for transmitting identity, this chapter focuses on emcees' non-textual methods of ``[amplifying] and in some cases [transforming] the information that listeners receive'' through texted and visual communication.\footnote{\cite{lorenkajikawaSoundingRaceRap2015}, 12.} As in my previous chapter, I employ two kinds of transcription to examine distinct musical qualities. With standard notation, I note emcees' uses of pitched and rhythmic motives, in addition to the text's interplay with elements of the beat. I also use poetic scansion to investigate meter, rhyme scheme, and verse form within rap verses. Compared to transcription in score form, this method of analysis reflects how rappers compose verses, allowing me to consider phrase and syntax from a perspective that traditional music-theoretical models cannot.\footnote{In particular, I am interested in comparing the score-forward methodologies used by Kyle Adams, Robert Komaniecki, and Ben Duinker with more lyric-abstracted modes of accessing the texted elements in hip-hop by Kajikawa and Tricia Rose.}