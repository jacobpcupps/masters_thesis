\onehalfspacing

\epigraph{{``Listening correctly can cause introspecting.''}}{R.A.P. Ferreira}

\section{Open Mike Eagle and Nocando's ``Unapologetic''}

Before landing several projects on best-of-the-decade lists from outlets like \textit{Pitchfork},
\textit{Brooklyn Vegan}, and \textit{Hip-Hop Golden Age}, the L.A.-based rapper Open Mike Eagle 
started off the decade with a declaration. His first studio LP, \textit{Unapologetic Art Rap}, coined
the subgeneric term ``art rap,'' which Eagle defines as ``a shorthand for leftfield and avant-garde rap 
music'' that he ``has spent his career redefining and expanding.''\footnote{
    \cite{openmikeeagle}.}
And not just Eagle himself, but so too the world of American hip-hop fandom around him: in 2020, the 
former Editor-in-Chief of \textit{Rap Genius} Shawn Setaro claimed that: ``[Art Rap] is a movement that
has become one of the most vital things happening in hip-hop today. Boundaries are stretched constantly,
in a way that recalls the innovations that took place in rap's Golden Age, without aping that era's 
sonics.''\footnote{\cite{shawnsetaroWhyYouShould2020}.}

If Setaro's claim holds merit, then we should interrogate how the music being received as art rap comes
to be known in that way: not just in its idiomatic ``sonics'' but also its themes, techniques, and 
approaches. Max Bell attempted this by digging through Bandcamp's ``art rap'' genre tag for 
\textit{Bandcamp  Daily.} After purporting then qualifying that art rap ``is antithetical to 
terrestrial radio station playlists,'' he notes a few unifying stylistic traits ``including but not limited
to left-field, forward thinking production, unconventional song structures and cadences, songs written 
from the perspectives of fictional characters, explicit and protracted engagement with social and political
issues, and absurdist metaphors and similes.''\footnote{
    \cite{maxbellWalkAvantGardeWorld2017}.} 
Bell's attempt at a definition is somewhat helpful, but its breadth offers only a partial picture, especially
due to the qualifications it must make. It seems, more than a decade later, the hip-hop community is still 
asking the question posed in Eagle's seventh track: ``WTF is Art Rap?''

Eagle and the guest emcee Nocando offer a glimpse at a definition within the lyrical content of the following
track, ``Unapologetic.'' As art-rappers, Nocando claims that they are small-time enough to identify with 
an underground scene:
\settowidth{\versewidth}{I dress nice but I ain't no goddamned sneaker freak.}
    \begin{verse}[\versewidth]
        I'm underground, man, like I'm beneath the streets, \\ 
        I dress nice but I ain't no goddamned sneaker freak.\footnote{
            \cite{openmikeeagle2010}.}
    \end{verse}
At the same time, the two are working at a level above the rappers who are friends with the local 
``scenesters'':
\settowidth{\versewidth}{There's a reason I'm performing after him.}
    \begin{verse}[\versewidth]
        Now everybody's got a rapper friend, \\ 
        But my name's higher on the flyer,\\
        There's a reason I'm performing after him.
    \end{verse}
On Eagle's end, he envisions art rap as dedicated to a particular audience:
\settowidth{\versewidth}{Who make sketches and daydream the whole class.}
    \begin{verse}[\versewidth]
        This to all the adolescent negro lads, \\
        Who make sketches and daydream the whole class. \\
        In '96, they would have been De La Soul Fans, \\
        (In 2010), it's My Chemical Romance.
    \end{verse}
At the same time, he is troubled by a whiter (and perhaps less-initiated) audience his art rap 
parties have attracted:
\settowidth{\versewidth}{Came to watch us battle and cackle like Fran Drescher.}
    \begin{verse}[\versewidth]
        Scene's built on the backs of black rappers, \\
        (Somehow) when I'm asking for cash you can't answer. \\ 
        Descendants of the folks that shackled my ancestors, \\ 
        Came to watch us battle and cackle like Fran Drescher.
    \end{verse}
The above quatrain also shows Eagle's concern with financial success. Notably, however, Eagle has a clear 
conception of a type of rap music that has sold itself over to the entertainment industry:
\settowidth{\versewidth}{Cause my little brother never heard of Little Brother \textellipsis}
    \begin{verse}[\versewidth]
        Cause my little brother never heard of Little Brother \textellipsis \\ 
        The only [rappers] he can discover \\
        Are the ones who please Viacom's executive n----- lovers.
    \end{verse}

Even from a brief lyrical analysis, ``Unapologetic'' creates a complex picture of the (black, masculine) identities
of the emcees on the track. They clearly work to textually communicate this, but text alone does not construct identity
in rap: musical content plays a role in this. This chapter works considers ``Unapologetic'' as a case study for the music
the thesis overall intends to address. Although (as my title suggests) I will eventually come to contextualize the track
within a subgenre I call underground, I first consider its relationship to some categories in rap which have received
attention in music scholarship in the past: namely, Amanda Sewell's conception of the ``nerdcore'' sugenre, and Adam Krims
and Justin Williams ``jazz/bohemian'' rap constructions. I come to reject these designations for ``Unapologetic'' because
of the limitations of contextualizing it within a single sound or style, instead looking towards listening practices in
hip-hop to come to a definition of underground that will serve as a guide for what this thesis explores.


\section{Nerdcore, Jazz-Rap, and the ``Golden Age''} \label{nerdcorejazzrapgoldenage}

For me as a listener, ``Unapologetic'' engenders two connections to the nerdcore subgenre. First, its beat 
layer is primarily constructed from two synthesizer loops. Although synthesizer does not explicitly
correlate to ``nerdiness'' in most contexts, when deployed within a hip-hop beat, it draws my attention
towards the technical prowess of the producer. This hearing also helps contextualize concerns of intelligence
made by both emcees. Nocando puts his own intelligence on display moving through a wide variety of allusions, 
including the Hanna-Barbera cartoon \textit{Tom and Jerry}, Joshua Marston's spanish-language film 
\textit{María, llena eres de gracia}, and even a rewritten version of Psalm 23.\footnote{
    According to Rory Ferreira, Nocando's re-contextualizaiton of Biblical language, here, might also 
    allude to Friedrich Nietzsche's \textit{Twilight of the Idols}, where he also plays with language
    from the same Psalm (Genius Annotation?).} 
Though less topically diffuse, Eagle's verses focus on intelligence by painting a picture of the listener he
attempts to reach: the ``adolescent Negro lads / who make sketches and daydream the whole class.''

The emcees' focus on intelligence through these contrasting approaches relates to the generic construction of
nerdcore. According to Amanda Sewell, ``[n]erdcore hip-hop happens when technologically savvy, verbally precocious,
and social marginalized people begin to make hip-hop using their skill sets and experiences.''\footnote{
    \autocite[223]{amandasewellNerdcoreHiphop2015}.}
It's also notable that
   
   \begin{itemize}
    \item Sewell's conception of nerdcore:
            \begin{itemize}
                \item MC Frontalot ``[As a nerd, you] don't fit into the traditional expectations of 
                masculinity, especially as a teenager'' (226). Compare to OME's ``adolescent negro lads who 
                daydream the whole class''
            \end{itemize}
        \item OME's ``Art Rap Party'' liner notes video:
            \begin{itemize}
                \item ``My idea of an art rap party, which is somewhat imaginary and hasn't all the way
                happened yet, is the kind of function where, like, intelligent folks go to dig on some
                intelligent music, \textellipsis it's not a dance party, you know what I mean?''
                \item ``  It's the kind of party where information's exchanged between people[;] 
                information's exchanged between performer and audience, and back and forth'' 
                \item ``Of course it's gonna be a little bit nerdy[.] \textellipsis We could talk about 
                anime, \textellipsis political stuff, exploring realms conversationally that aren't in the 
                typical rap song''
            \end{itemize}
    \end{itemize}

Adam Krims constructs a variety of subgeneric designations for rap that he argues started to concretize
around 1994, each harnessing distinct musical styles to address particular topics. In particular, he 
notes that the genres he designates ``jazz/bohemian'' rap and ``reality'' rap articulated themselves 
as alternatives  to ``gangsta'' rap based on the perception that it was the genre most aligned with
a commercial mainstream.\footnote{
    \autocite[64--65]{adamkrimsRapMusicPoetics2000}. Krims offers two case studies\textemdash The Roots' 
    ``What They Do'' and RZA's introduction on Wu-Tang Clan's \textit{Wu-Tang Forever}\textemdash as 
    instances of each genre directly addressing what it conceived to be the commercial mainstream at
    the time.}
In much the same way, Eagle's criticism of the types of artists that please Viacom's executives and his 
estimation of his own work as a ``Black man's art'' allow him to construct the same dichotomy; he and
other rappers like him are bohemians within the music industry, and you, the listener who can appreciate
his style, are a connoisseur of ``fine art'' rap music.

    \begin{itemize}
        \item OME's definitions (and genre designation) construct art rap as a type of high art taken with 
        respect to the gamut of hip-hop in the 2010s
        \item Another way in which hip-hop musicians have constructed their sound to be heard as a high art
        within  hip-hop communities is through sonic links to jazz and related Black sonic profiles (Table 1.1)
            \begin{itemize}
                \item As early as 2010, artists including Kendrick Lamar and billy woods are sampling 
                from  mainstream jazz in the seventies and eighties
                \item As the decade progressed, other artists including R.A.P. Ferreira (fka Milo, aka 
                Scallops Hotel)  and Armand Hammer (the rap duo consisting of woods and ELUCID) begin to 
                draw on more eclectic styles of jazz, especially Sun Ra
                \item Soul and blues samples proliferate throughout the decade as well
            \end{itemize}
        \item Williams (2015) argues that, during the golden era of hip-hop, artists used jazz and other 
        Black sonic profiles in order to construct their work as a ``high art'' substyle within the genre
            \begin{itemize}
                \item Jazz rap came into formation as an ``alernative'' to gansta and pop rap, and jazz in 
                the decade  preceding had become an institutional and ``serious'' art form (48--52).
                \item Certain musical elements function as ``jazz codes'' which, through performance and 
                timbral qualities, reach audience members as having a jazz \emph{feel}, even if they are 
                not drawn from jazz records, strictly speaking (55).
                \item Groups like A Tribe Called Quest and Digable Planets drew on these codes to (explictly 
                or subtextually) reach listeners as artists with an identity that was different than those of 
                their contemporaries working in pop/gangsta rap styles. (58)
            \end{itemize}        
        \item While I'm not convinced that jazz rap's renaissance in the 2010s was explicitly motivated to construct 
        jazz rap as a type of high art uniformly, I think the same urge to \emph{differentiate} one's self to the 
        listener, to articulate identity in a way that goes against some construction of identity in the mainstream, 
        still exists as a driving force in the 2010s.
    \end{itemize}
    
The sonic profile of ``Unapologetic'' does not fit neatly into the generic category Krims and Williams
construct. While both acknowledge the possibility for a stylistic eclecticism beyond that reaches beyond
jazz sonically, the groups that receive their focus (A Tribe Called Quest, De La Soul, Digable Planets, 
Jungle Brothers, and The Roots) all rely to \emph{some} degree on the sound of jazz in their music. 
Without the sonic trappings of some jazz codes signify some relationship to jazz in sound, 
``Unapologetic'' cannot reach listeners as ``bohemian''\textemdash at least not in the same way 
the groups connected to the ``Golden Age'' did. I reject that ``Unapologetic'' can neatly
fit in within one sound or style, so I will instead look for some sort of practice in the music
which can be emulated.

Mike Eagle's ``Unapologetic''\textemdash his whole conception of art rap as a moment at the start
of the 2010s\textemdash articulates something that is at once new in hip-hop while relating to ideas,
sounds, and genres from throughout hip-hop's history. Although the track's ``tech-y'' sound and 
construction of ``intelligence'' intimates some connection to nerdcore, the designation falls short 
because of nerdcore's historically white positionality and lack of concern with its own construction 
as art. On the other hand, to situate ``Unapologetic'' within the lineage of jazz/bohemian rap does 
not adequately account for the  timbral qualities or ``codes'' of the track but gets closer to the 
topical focus and  positionality art rap affords itself with respect to the ``mainstream.''

\section{Hip-Hop Practice and Listening as Meditation} \label{listeningasmeditation}

In the previous section, I argue that ``Unapologetic'' is characterized not by one style such as
nerdcore or jazz rap but rather by a musical practice. This practice is one of listening: in 
particular, a type of sonic engagement I will term \emph{listening as meditation}. Historically,
this listening practice is abundant in all stages of the making of hip-hop\textemdash from its 
conception in the mind of the producer and emcee, to its performance live or in studio, and 
eventually its reception by the listener. In this section, I show that practitioners throughout
hip-hop's history and Eagle himself promote this style of listening, one that engenders a deep 
immersion in hip-hop as a  musical form.

The phrase listening as meditation serves as an inversion of the mind-body dualism invoked by 
earlier popular music discourses. To be sure, by invoking ``listening'' as a unifying thread 
amongst a wide variety of stylistic manifestations, I am inviting a tacit definition of the 
underground as ``hip-hop to be \emph{listened} to'' as distinct from ``to be danced to'' or 
``merely heard.''\footnote{
    My focus on listening runs the risk of positioning underground hip-hop as a ``transmission of mind--mind messages'' 
    between composer and listener (see~\autocite[20]{suzanneg.cusickFeministTheoryMusic1994}.) While this style of
    inquiry still characterizes a great deal of music-theoretical scholarship, my reservations arise from the potential
    for erasing the embodied experience of performers (who tend to be the composers in my repertory as well) and the
    listeners within the exchange that happens in underground hip-hop. I therefore intend with my term for this mode
    of listening (``as meditation'') to be interpreted as a unified mind-body practice.}
Part of the trouble with such a definition is potential its use as a justification  for excluding certain
musics from being objects of critical inquiry. Moreover, my choice to prioritize discussing 
this music as an engagement of one's musical intellect does not exclude its use as ``dance'' 
music, nor as a music that does not engender embodied responses. The metaphor of listening 
as meditation, then, proposes a unification of cognitive and embodied responses that characterize
engagement with underground hip-hop. In short, it is an argument that will, in the coming pages,
put this chapter's epigraph to the test (that ``Listening correctly can cause introspecting.'')

At each point in the process of making hip-hop, and indeed throughout different points in
its history as a music form, several practitioners have evinced that their listening exhibits
a meditative engagement with music. The rapper-producer Count Bass D suggests that this style
of listening guides his process for sample selection and composition particularly for the
introduction to his 2002 track ``Truth to Light'': 
    \begin{quote}
        Songs that really I like a whole lot, that I've liked over the years, kind of run through my head all
        the time and so they kind of creep into songs. \textellipsis Unless you know [Nice \& Smooth's ``Funky
        for You''], you don't know who I'm talking about or what I'm talking about, but to people who are in
        the know, it strengthens their faith that the things I'm talking about that they don't understand may
        have some relevance to them in time.\footnote{\autocite[100]{mickeyhessHipHopDead2007}.}
    \end{quote}
Count particularly focuses on familiarity, being ``in the know'' with the source material, as
a requirement for parsing his dialogue with the Nice \& Smooth at the beginning of the track, but
his discussion also reveals listening norms within the hip-hop community. First, Count as a producer
engaged with ``Funky for You'' regularly enough over a long period of time that the decision to
dialogue with the artists ad-libs on the original recording felt subliminal and ``[crept] into''
the track. Second, he expects that his audience will have engaged with the material on a similar
level, enough to ``know what [he's] talking about.'' Finally, and perhaps most interestingly,
Count conceives of a \emph{type} of listener, who may not know the source material but as a
result of their familiarity with the listening norms in the genre trusts will come to them
``in time'' as they listen further and more broadly beyond ``Truth to Light.'' Such a
listener is the \emph{meditative} listener and is the type of listener Count considers
his primary audience.

The meditative listening process for the emcee is similar to the producer's in that the emcee
listens with enough regularity and focus that the imitation of style can become subconscious,
to the point where it can prove problematic for their on writing process. Open Mike explains
that because of his listening interactions with the emcee MF DOOM's flow, ``he has to be 
careful with [DOOM's] flow, because [he] can almost get into [DOOM's] mind in terms of how he 
writes.''\footnote{\cite{estellecaswellRappingDeconstructedBest2016}.} Open Mike characterizes
his depth of engagement with DOOM as something that could turn into a problem because it would
be easy for him to be write verse in the same exact manner that DOOM does and thus turn into
something more derivative than he would like. Again the practitioner positions their compositional
choices as a subconscious choice based on deep engagement with models for composing that show 
up in their listening. For Open Mike as an emcee, this listening has occurred with enough
frequency, depth, and regularity that he subsumed DOOM's style of flow into his own and needs
to consciously choose to offset his writing from this practice at times.

Finally, hip-hop practitioners show that listening as meditation informs their interpretation of
performance by other practitioners; they envision the artists they interact with as meditative
listeners, who then serve as guides for further meditative listening. The Bronx DJ Grand Wizzard
Theodore describes, when hearing a record through DJ Kool Herc's soundsystem, ``[i]t made you
listen to a record and made you appreciate the record even more. He would play a record that 
you listened to every day and you would be like `Wow, that record has \emph{bells} in it?' It's 
like you heard instruments in the record that you never thought the record even had.''\footnote{
\autocite[139]{christabronGlassHiphopProduction2015}.} Theodore's description shows that even
within the context of a block party, the type of listening hip-hop engenders brings about new
depths to the music beyond the standard engagement.

\section{Definitions and Limitations of Underground}

In the above sections, I have examined stylistic and thematic traits in two musicological 
constructions of subgenre\textemdash nerdcore and jazz rap\textemdash considering Open Mike
Eagle's ``Unapologetic'' as an example that does not neatly fit in either category as previous
scholars have defined them. Additionally, I considered how Eagle participates in a historical
practice shared by a number of hip-hop musicians: a practice where musicians are listeners first,
listeners who engage with the music at a meditative level. In this section, I argue that the 
designation \emph{underground} in hip-hop is defined by meditative listening. Fans of underground
hip-hop are encouraged to engage with the subgenre as meditative listeners because they see 
underground artists perform meditative listening. Then, as a result of the perceived closeness to
the artist that characterizes the underground, listeners choose to engage with the music in a 
similarly meditative fashion.

Before I arrive at this definition, I need to address three conceptions of the underground which I
believe to be faulty and therefore do not wish to import in my definition. In particular, a definition
of underground cannot meaningfully be constructed from discourses of artistry, authenticity, or 
commerciality. These three metrics permeate popular discussions of what makes music ``underground,'' 
so addressing them out front will prove useful in elucidating what I wish to focus on: namely, 
listening practice and methods of constructing the music.

Defining underground hip-hop as an ``art'' genre invokes discourses around what makes a piece of
music beautiful, a metric that is at once too broad to define and too based in subjective experience
to communicate effectively. Although rappers like Eagle have constructed their personas in this way,
and in fact fans buy into such constructions, I find it telling that such discourses are often
short-lived. As early as 2013, Eagle raps:

\settowidth{\versewidth}{I used to throw these sensitive parties for art rap,}
    \begin{verse}[\versewidth]
        I used to throw these sensitive parties for art rap, \\
        No regrets, but I was foolish to start that. \\
        Sophisticated fuckers left a bitch with a bar tab, \\
        And now we just throw pizza parties.\footnote{
        \cite{milo2013}.}
    \end{verse}
Eagle's indictment of the ``art rap'' scene came relatively quickly on the heels of his construction of
identity as an art-rapper, in a feature only a few projects later than he unapologetically declared
it. Though his lyrics focus on the pretension and bourgeois tastes of the fans it garnered him, I also 
wonder if writing ``artful'' rap lost its meaning. As a listener, I find it hard to describe whether
something reaches me artful; for instance, I do not, as Justin Williams suggests, hear a piece based around
a jazz sample as more ``artistic'' than, say, something constructed from DAW-based synthesizers simply
because of the importance jazz is given in contemporary American culture.\footnote{
    For my discussion on Williams' argument that jazz rap reaches listeners as a ``high art,'' see
    section~\ref{nerdcorejazzrapgoldenage}.} 
Rap music's artistry seems less quantifiable (and more personally defined) than its ``undergroundness.''

Invoking authenticity to define underground also proves to be a slippery metric, in perhaps a similar way
to artistry. This definition finds purchase in popular discourse as well because of artists who tie it
closely to the early, pre-recorded era of hip-hop. Fab 5 Freddy summarized this mentality when discussing 
his role as the writer for the 1982 film \textit{Wild Style}. The writers set the story before the 1979
release of Sugarhill Gang's single ``Rapper's Delight'' because ``[they] wanted to go back to a few years 
earlier\textellipsis \emph{when hip-hop was completely underground, when the form was raw and pure.}''\footnote{
    \autocite[23]{justinawilliamsRhyminStealinMusical2013}. Emphasis my own. The release
    of ``Rapper's Delight'' is often thought of as the moment when hip-hop went commercial.}
Fab 5 Freddy's association of ``undergroundness'' with ``rawness'' or ``purity'' illustrates one type
of a way that listeners, critics, and fans alike make sense of hip-hop, but once again, I wonder if the
term has any bearing on the sound of underground hip-hop. To me, discourses of authenticity seem to be
based on subjective, evaluative judgements a listener makes.

Finally, a definition of underground based in commerciality may seem valuable at a surface level, but modern
methods for discovering and participating in cultures of listening problematize such a conception. Commerciality
may very well be the metric for judging the authenticity of a rap song: an artists' commercial success can be tied to
whether or not a listener receives them as underground. However, commercial success is not the sole quality that
constructs ``undergroundness.'' As Anthony Kwame Harrison argues in his ethnographic foray into the San Francisco
underground hip-hop scene, ``the label `underground' can be applied to everything from Grammy-nominated artists like
Common and the Roots to groups like The Latter, who once boasted(!) of having sold only two copies of one of their
CDs.''\footnote{
    \autocite[9]{anthonykwameharrisonHipHopUnderground2009}.}
His quotation about the designation from Kegs One, the artist and proprietor of the hip-hop shop Below the Surface, 
elucidates how gradations of commerciality manifest within underground hip-hop:

    \begin{quote}
        I have a few different people that are into the more `commercial underground' stuff, but then 95 percent of my
        customers are here for the  literal four-track tapes. You know, dirty-sounding, low-budget, in-the-room, in-the-closet
        recorded tapes.\footnote{
        Quoted in \autocite[10]{anthonykwameharrisonHipHopUnderground2009}.}
    \end{quote}
Harrison assesses a variety of commercial levels in hip-hop, all the way from The Latter's two records sold, to 
the Bay-area artists who are about to touch the surface underground, on up musicians like The Roots, who have been 
Jimmy Fallon's in-house band since 2009. For acts to have varying degrees of success and still be considered 
underground points to some metric beyond commerciality being invoked to define the quality of ``undergroundness.''

The definition Harrison constructs is based on an ethos he observes in the Bay-area underground, and his observations
support serve as a basis for my own definition based in listening practices. In particular, he notes ``the idea of a
blurring, thinning, almost imperceptible line that separates artists and fans'' and ``an intimacy of the fan bases
[underground artists] tend to attract.''\footnote{
    \autocite[10--11]{anthonykwameharrisonHipHopUnderground2009}.}
Although, like Harrison, I believe this ethos decreases in palpability as an artists commercial success and notoriety
increases, these artists are still stylistic participants in underground hip-hop because this ethos can be perceived
in the music. In particular, I argue, this intimacy can be perceived in the way an artists models meditative listening
for fans; in effect, they see the artist acting as a listener\textemdash as a fan themselves\textemdash and experience
a type of intimacy in that recognition.

This perception of intimacy and the acts of meditative listening it engenders are core to understanding how a listener
experiences underground hip-hop. Underground is not a style of hip-hop codified by one sonic practice, nor by one thematic
focus, but rather by a multiplicity of both, guided by the ways in which the musicians demonstrate their relation to a
network of practices and focuses. Underground hip-hop artists that bring together a multiplicity of experiences in their
work model a listening practice for the fan, who may at first intentionally re-navigate these listening networks for themselves
or may in time, as Chuck D suggests, come to trust that these networks exist and will come to them through their own paths of
musical experience.


\section{Outline of the Thesis}

Working from the definition of underground as a practice of meditative listening shared between fans and artists 
participating in the subgenre, this thesis focuses on transcribing and analyzing a repertory of underground rap songs, 
attempting to uncover how the quality of ``undergroundness'' manifests within beats and flows. Even though ``underground''
cannot be clarify how  the music is received as underground, especially in instances where ``undergroundness'' is less 
textually explicit than in Eagle's ``Unapologetic.'' In short, I am interested in how musical text in my repertory 
``sound'' the underground.

My approach to reading musical texts agentially is bound up in Loren Kajikawa's notion of ``sounding'' in rap music. 
Particularly, Kajikawa argues that the rap song is a medium for transmitting identities and meaning between the artist
and the listener.\footnote{
    \autocite[2]{lorenkajikawaSoundingRaceRap2015}.}
Where his scholarship focuses on the encoding of racial and gender identities, mine will focus on a subsequent matter:
how an artist navigates mainstream expectations about these identities, and in so doing asks you to listen in a new, more
meditative way as a result of their navigation.

The two primary chapters in this project will interrogate the overlapping yet distinct methods by which hip-hop
musicians sound the underground. My first chapter focuses on beatmaking, arguing that producers create underground
hip-hop beats by avoiding fixed loops and regular sample lengths.\footnote{
    Justin A. Williams argues that all of rap's subgenres deconstruct this notion (see \cite{justinawilliamsBeatsFlowsResponse2009}).
    Williams' critique of the musical-theoretical focus on repetition in hip-hop is foundational to my approach; however, my work
    in this chapter deals more with underground producers' deliberate choice to disrupt repetition beyond layering elements in and
    out of the music.} 
In particular, I examine how producers introduce variety primarily through digital editing techniques that mimic 
the live, improvisatory roots of the genre within a broadly repetitive musical texture. I employ two distinct methods of
transcription based on other scholars methods for beat transcription\textemdash namely, Kajikawa's breakbeat 
transcriptions\footnote{
    \textit{Cf.} \autocite[29--30 and 36--37]{lorenkajikawaSoundingRaceRap2015}, 29-30  and 36-37.} 
and Williams' basic beat transcriptions.\footnote{
    \textit{Cf.} \autocite[61ff]{justinawilliamsRhyminStealinMusical2013}.} 
From my transcriptions, I work out four terms to methods producers use to affect variety in the musical texture: 
resampling/recomposing, choking, glitching, and slipping. While this list of methods does not exhaust the ways in which
producers sound the underground, it demonstrates the hip-hop beat as a space for co-creation of variety with the rapper
and thus alternative identity within hip-hop.

Building off this notion, my second chapter the comparable role underground emcees play in composing their verses in rap
songs. Although lyrics offer the clearest method for transmitting identity, this chapter focuses on emcees' non-textual 
methods of ``[amplifying] and in some cases [transforming] the information that listeners receive'' through texted and visual
communication.\footnote{
    \autocite[12]{lorenkajikawaSoundingRaceRap2015}.} 
As in my previous chapter, I employ a few methods of transcription to examine different musical elements. When working primarily
the way text creates metrical patterns, rhyme schemes, and linguistic-syntactic structures, I use Mitchell Ohriner's system for
visualizing rap flows, which is intentionally abstracted from Western musical notation.\footnote{
    For a sample and detailed overview of Ohriner's notational method, see \autocite[xxvii--xl]{mitchellohrinerFlowRhythmicVoice2019}.}
When dealing with an emcees' approach to performance rather than structure, I opt for staff notation, mindful of the its limited
ability to denote the complexities of a vocal rap performance. 
    