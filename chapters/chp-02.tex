In response to Kyle Adams' seminal \textit{Music Theory Online} article ``Aspects of the Music/Text Relationship in Rap,'' Justin Williams litigated an important phenomenon in rap music production that changed alongside the move from the turntable to the recording studio. He writes:

\begin{quote}
    \small ``In terms of rap music recordings, the idea of a completely fixed loop is largely fictitious. There may be a set of layers which we could term the `basic beat' which repeats intact for certain durations of time, but one would be hard-pressed to find an entire musical complement that stays the same throughout. Rap music’s layers will more often than not fluctuate throughout a given song, with sonic additions and subtractions, manipulations of digital samples, and even sharp changes in aspects of the basic beat.''\footnote{\cite{justinawilliamsBeatsFlowsResponse2009}. It is worth noting that Adams does not necessarily dispute the occurrence of development within an accompaniment as a phenomenon, but he does maintain that the unchanging elements of the beat function as ``primary accompanimental layers'' (See \cite{kyleadamsPeopleInstinctiveAssumptions2009}.)}
\end{quote}

\noindent \normalsize As both a fan of and critical listener to rap music, I experience repetition and variation significantly within musical lexicon of the rap instrumental. Adams notes that even as the locus of rap music-making moved into the studio, producers remained adherent to the break-beat based origins of the genre, and so structural elements of the hip-hop beat tend to repeat within a four to eight bar, often simple quadruple metrical space. At the same time, Williams' observations point towards producers' observable preference for variation within repetition, made feasible by newer technologies for sampling, manipulating, and composing with pre-recorded digital materials.

The answer to whether the hip-hop beat is primarily repetitive or primarily developmental is indeterminate because both dimensions exert influence over the process of creating it. Instead, I want to interrogate how producers use these dimensions to aesthetic or rhetorical ends: to invoke Loren Kajikawa's notion of sounded phenomena in hip-hop, how ``rap artists produce (and listeners interpret) musical meanings at the level of the song.''\footnote{\cite{lorenkajikawaSoundingRaceRap2015}, 2.} While Kajikawa's focus on racial identity unpacks the most important element of a producer's  curatorial work, another conscious choice a producer makes is to identify with or against the hip-hop mainstream. Producers use the beat as a means of coding themes and identities complementary to the rapper who declares allegiance to the underground within their flow. Such a reading, I believe, resonates with the central argument of Kajikawa's \textit{Sounding Race} because ``rap has cultivated a mainstream audience\textellipsis by promoting highly visible (and often controversial) representations of black masculine identity''\footnote{\cite{lorenkajikawaSoundingRaceRap2015}, 5.}.

As with any sub-generic distinction predicated on narratives of authenticity, I use the term \emph{underground} trepidatiously, knowing full well it has as many meanings as it has users. What I call the underground signifies a space where deviation from the mainstream is expected. This is not a value judgement, nor an all-encompassing definition of either mainstream or underground. My definition is necessarily vague because these are elusive terms, but they hold importance because they represent \textit{de facto} ``imagined communities'' with which rappers, producers, and listeners identify.\footnote{The concept of the imagined community of hip-hop has been broadly traced by Williams via Joseph G. Schloss: the term comes from Benedict Anderson's conceptualization of how a community 
Williams argues for the importance of recognizing hip-hop as an imagined community, a term coined by Benedict Anderson for conceptualizing nationalism, because such a community must exist to provide a cohesive set of strategies by which producers and listeners interpret musical objects (see \cite{justinawilliamsRhyminStealinMusical2013}: 13-19.) 

Joseph G. Schloss also develops the notion of an imagined hip-hop community to categorize the network of hip-hop producers who serve as his ethnographic subjects, and to explain their conception of the audience for their music (see \cite{josephgschlossMakingBeatsArt2004}: 4-5.)}

In this chapter, I argue that producers work with rappers to sound the hip-hop underground by using variety within their beat architecture, punctuating alternative identity in sampling; I base this argument in transcriptions that illustrate developmental and repeating elements within the musical texture. My case studies show that underground producers tend to deviate from expectations concerning form in their sectional divisions of the beat. Highlighting sample-based and non-sample-based approaches to beat-making, I trace methods of sample and loop manipulation such as choking, glitching, and slippage as producers' means of variation. Lastly, I contend that underground hip-hop is sounded by an overarching aesthetic of disunity, linking the tradition to Olly Wilson's heterogeneous sound ideal of African-American music. I briefly conclude by discussing the importance of the rapped text as the most variable element within the texture and consider its importance in future projects I will undertake concerning the hip-hop underground.
\clearpage

\doublespacing
\section{Methods of Transcription and Analysis}
I use transcription in this chapter\textemdash indeed, overall in this project\textemdash despite knowing that it introduces a level of abstraction from both the musical practice and perceptual experience of my hip-hop repertory. Scholars such as Joseph G. Schloss have meaningfully analyzed hip-hop production while eschewing transcription altogether on ethical and aesthetic grounds.\footnote{\cite{josephgschlossMakingBeatsArt2004}, 13-15.} Others, like Kajikawa and Adam Krims, have employed methods of transcription that move away from standard notation in Western Classical traditions.\footnote{\textit{Cf.} \cite{lorenkajikawaSoundingRaceRap2015}, 29-30 and 36-37; \cite{adamkrimsRapMusicPoetics2000}: 105-110.} Still others, including Adams and Robert Komaniecki, rely primarily on standard notation in order present their arguments within traditional spheres of music-theoretical discourse.\footnote{\textit{Cf. }\cite{kyleadamsMetricalTechniquesFlow2009}; \cite{robertkomanieckiAnalyzingCollaborativeFlow2017}.} Each of these approaches holds its own merit, and each privileges a different audience: the creator, the listener, and the academic.

Although I understand Schloss and others' reticence to transcribe rap music, my choice to do so reflects a process of listening that I have undergone to deepen my understanding of the repertory I use in this thesis. In the following pages, I aim to build into traditions of popular music scholarship that do not apply ``the tools of notation and analysis developed for the study of Western Classical music\textellipsis uncritically to rap music.''\footnote{\cite{lorenkajikawaSoundingRaceRap2015}, 12.}  In doing so, I do not treat my transcriptions as objective, nor do I believe should they serve as a replacement for listening to musical material; instead, I offer them as a subjective realization of my ``living inside'' the musical object for a time.\footnote{\cite{peterwinklerWritingGhostNotes1997}: 200.} The value of these transcriptions primarily within the process of creating them.

The two styles of transcription I employ in this chapter each represent the tendency to create variety within reptition in distinct ways. First, I use a tabular “roadmap” style of transcription that overviews musical texture and form. These ``roadmaps'' call to attention that which a producer might hear as significant within a track and also mark the addition, subtraction, or other alteration of layers as the beat progresses. In an attmept to center praxis, I create my roadmap tables using Ableton. Second, I use standard notation to create a musical ``snapshot'' of the beat at distinct points in the musical texture. These ``snapshots'' allow me to discuss the function of particular musical participants, in addition to showing how repetitive textures might shift over the course of the beat. I use staff notation not as a reflection of producers' methods for visualizing the beat, but because staff notation is more widely legible in academic settings than tabulature and other methods of transcription.

Each style of transcription also allows me to draw analytical conclusions about compositional tendencies within the underground.

My roadmap and basic beat transcriptions offer contrasting perspectives on the architecture of four underground hip-hop beats; they help me analyze the nature of underground hip-hop production in several distinct ways. Assuming the normativity of sixteen-bar verses and four-to-eight-bar hooks within the hip-hop mainstream,\footnote{To my knowledge, no large-scale corpus study exists to corroborate this assumption; however, it does reflect the practice of verse writing championed by emcees such as Rakim, MF DOOM, and Open Mike Eagle (See \cite{estellecaswellRappingDeconstructedBest2016}).} I read deviation from formal expectations as a mode of alterity. I also focus on the rhetoric of sample choice and the aesthetic of heterogeneity as distinct manifestations of the underground within beats.

I trace four methods by which producers affect variety: recomposing, choking, glitching, and slipping. 
\clearpage

\singlespacing
\section{Sample-Based Case Studies}
My first two case studies are MF DOOM's 2003 ``One Beer, '' produced by Madlib, and Kendrick Lamar's 2011 ``Rigamortis,'' produced by Willie B. These are both sample-based on account of their construction around one ``lead'' sample. ``One Beer'' samples primarily from Cortex's 1975 ``Huit Octobre 1971,'' and Willie B samples Willie Jones III's 2010 ``The Thorn.'' Both producers introduce variety to their limited compositional palettes through techniques of sample manipulation and layering while also circumventing formal expectations for mainstream hip-hop.

\subsection*{\centering MF DOOM's ``One Beer''}
\begin{table}[h!]
    \tiny
    \centering
\begin{tabular}{c|c|c|c|c|c}
     Timecode & DOOM & Lead Sample & Bass & Drums & Vocal Sample \\
     \toprule 
     0:00-0:14 & ``I get no kick\textellipsis'' & ||: ``Huit'' I :|| x6 & ||: ``Huit'' I :|| x6 & ||: ``Huit'' I :|| x6 &  \\
     \midrule
     0:15-0:17 & ``I get a kick outta\textellipsis'' & •//• x1 & •//• x1 & •//• x1 & \\
     \midrule 
     0:18-0:19 & ``brew!'' & & & & \\
     \midrule
     0:20-0:40 & ``There's only one beer\textellipsis'' & ||: ``Huit'' II :|| & ||: ``Huit'' II :|| & ||: ``Huit'' II :|| &  \\
     \midrule
     0:41-1:01 & ``Told him tell 'em\textellipsis'' & •//• x2 & •//• x2 & •//• x2 & \\
     \midrule
     1:02-1:21 & ``He went to go laugh\textellipsis'' & •//• x2 & •//• x2 & •//• x2 & \\ 
     \midrule
     1:22-1:42 & ``(skeezer) eye, and squeeze\textellipsis'' & •//• x2 & •//• x2 & •//• x2 & \\
     \midrule
     1:43-1:57 & ``Looser than a pair\textellipsis'' & ||: ``Huit'' I :|| x6 & ||: ``Huit'' I :|| x6 & ||: ``Huit'' I :|| x6 & \\
     \midrule
     1:58-1:59 & ``Few could do it\textellipsis'' & •//• x1 & •//• x1 & •//• x1 & \\
     \midrule
      2:00-2:02 & ``Take it from the dude\textellipsis'' & •//• x1* & •//• x1* & •//• x1* & \\
     \midrule
     2:03-2:23 & ``He plot shows like...\textellipsis'' & ||: ``Huit'' II :|| & ||: ``Huit'' II :|| & ||: ``Huit'' II :|| & \\
     \midrule
     2:24-2:44 & & •//• x2 & •//• x2 & •//• x2 & \\
     \midrule
     2:45-3:05 & & •//• x2 & •//• x2 & •//• x2 & \textit{Spider-man} I \\
     \midrule
     3:06-3:50 & & ||: \textit{Spider-Man} :|| x8 & MPC & MPC & \textit{Spider-man} II \\
     \midrule
     3:51-3:55 & & •//• x1* & •//• x1* & •//• x1* & ``Your attempts\textellipsis'' \\
     \midrule
     3:56-4:18 & & •//• x4  & •//• x4 & •//• x4 & \textit{Spider-man} III \\
     \bottomrule
\end{tabular}

\hfill{*choked on last two beats}
    \caption{Roadmap to MF DOOM and Madlib's ``One Beer''}
    \label{tab:1}
\end{table}

\normalsize Outlined in Table~\ref{tab:1}, Madlib constructs his beat from discrete sections of ``Huit Octobre'' that contrast in groove and harmony, providing sectional clarity beneath DOOM's through-composed verses. The first section features a synth and bass in octaves on the same melody accompanied by drums emphasizing upbeats on the hi-hat. The second section's straight feel juxtaposes the first's triplet-eighth swing. Pitched up a semitone to better match the harmony of the first, it features a four-chord loop with a prominent vocal countermelody. The basic beats formed by these two samples are rendered in Figures~\ref{fig:1.2}~and~\ref{fig:1.3}.

\begin{figure}[h]
    \centering
    \includegraphics[width=\textwidth]{images/figures/chp 02/Figure-02.1-One-Beer-BB-I.pdf}
    \caption{The basic beat in the first section of ``One Beer.''}
    \label{fig:1.2}
\end{figure}

Two dimensions of each section elucidate Madlib's intended contrast. Harmonically, the shift in samples brings about different function. The first sample features a single, \emph{repetitive} harmony that arpeggiates a global I$^{b7}$.\footnote{\cite{kyleadamsHarmonicSyntacticMotivic2020}. The three categories Adams provides for harmony in hip-hop are repetitive, oscillating, and expansional, all of which I touch on throughout the course of this paper.} Harmonic stasis gives way to \emph{oscillating} F-sharp and G extended tertian harmony in the second sample, which functions as an activation for a sixteen-bar would-be verse unit before arriving returning to stasis for an eight-bar would-be hook.

\begin{figure}[h]
    \centering
    \includegraphics[width=\textwidth]{images/figures/chp 02/Figure-02.2-One-Beer-BB-II.pdf}
    \caption{The basic beat in the second section of ``One Beer.''}
    \label{fig:1.3}
\end{figure}

In conjunction with harmony, the timbre, dynamics, and rhythmic stress of the drum patterns in each sample reflect the contrasting functions of each section the beat projects. Schloss notes that producers often choose their samples based on the aesthetic delight they experience concerning timbre, and that drum sounds are sought after with preference.\footnote{\cite{josephgschlossMakingBeatsArt2004}, 141-42.} In addition to the timbre of the drums in ``Huit Octobre,'' Madlib's two sections contrast in rhythmic content and function. When the track arrives at the first sample, the relative stasis is reinforced by the accompanimental kit pattern, focusing the listener on the synth line. In the second section, the drums activate along with the harmony using distinctive fills and striking timbres throughout the four-bar loop. This culminates in the prominence of the two-beat long triplet eighth fill in m. 2 of the loop.

Madlib's samples also communicate information about the theme and message of ``One Beer.'' Combined with DOOM's \emph{allosonic}\footnote{\cite{justinawilliamsRhyminStealinMusical2013}, 3. Whereas most hip-hop sampling would be an \emph{autosonic} duplication, DOOM intones Porter's words in a parodical \textit{sprechstimme} that sounds as if it were sung by a drunk fool getting a kick out of brew.} interpolation of Cole Porter's ``I Get a Kick Out of You'' and dialogue from the 1981 episode of \textit{Spider-man}, ``Dr. Doom, Master of the World,'' Madlib draws from a ``rich assortment of multimedia borrowings, references, and parodies that operate in hip-hop music as a whole'' all within the course of one song.\footnote{\cite{joannademersSampling1970sHipHop2003}: 42.} The result of this is two-fold: (1) the piece coalesces as a work of timbral and stylistic heterogeneity,\footnote{\cite{ollywilsonHeterogeneousSoundIdeal1992}: 329. Timbral stratification is core to Wilson's heterogeneous sound ideal.} and the thematic focus of the track (DOOM, rap's supervillian, joking around about alcohol), is reinforced by the sound sources of the beat.

\subsection*{\centering Kendrick Lamar's ``Rigamortis''}

\begin{figure}[h]
    \centering
    \includegraphics[width=\textwidth]{images/figures/chp 02/Figure-02.3-Rigamortis-BB.pdf}
    \caption{The basic beat of ``Rigamortis.''}
    \label{fig:2.1}
\end{figure}

The section of music Willie B samples from ``The Thorn'' features Jones' combo regrouping a simple quadruple meter into 3+3+2 beat divisions, underscoring the saxophone melody. Figure~\ref{fig:2.1} shows the sampled layers in its top three staves composited with Willie B's added layers in the staves below. To combine these elements, he filters out the low end, pitches up the sample by nine semitones, and adds a drum loop, sub bass, synth, and other post-production effects. ``Rigamortis'' also begins with a version of the sample that repeats only its first two measures before layering in the other textures.

The beat's repetitive harmony and form help ground the listener within Lamar's complex vocal delivery. Although the return of the ``He dead!'' section functions like a hook, Lamar plays on listener's expectations for this call-and-response to return intact; instead, its first instance adds lyrics, it returns early as a fragmented interruption of the first verse, and it transitions seamlessly into the second verse via shared rhyme scheme. Because Lamar's verse has the ``tendency to fill up all the musical space'' within the mix,  the beat is relegated to a more simplistic, repetitive role.\footnote{\cite{ollywilsonHeterogeneousSoundIdeal1992}, 328.}

The grounded, repetitive simplicity of the beat does not mean the beat is unchanging. Willie B makes frequent use of sample choking, evening muting the sample completely for parts of Lamar's final verse. He also obfuscates the texture by employing what I call \emph{sample slippage}, wherein the micro-rhythmic space between the lead sample and drum loop ebbs and flows due to slightly varied loop lengths and expressively delayed re-triggering.\footnote{The phenomenon of sample slippage dovetails with Anne Danielsen's work on the Beat Bin and rhythmic tolerance (see \cite{annedanielsenHereThereEverywhere2016}: 29\textit{ff.})} The affect of such shifting creates a listening experience akin to phasing, as samples (albeit sonically discrete ones) move in and out of sync with each other.  While Figure~\ref{fig:2.1} and Table~{\ref{tab:2}} show all of the sample loops beginning and ending in alignment, in reality, the beginnings and ends of loops are messy, and each float in and out of time with each other throughout the track.

\begin{table}[h!]
    \centering
    \tiny
\begin{tabular}{c|c|c|c|c|c|c}
     Timecode & Kendrick & Lead Sample & Drums & Bass & Synth & SFX \\
     \toprule 
      0:00-0:10 & ``Alright, here we go\textellipsis''* & ||: ``Thorn'' I :|| x4 & & & & \\
     \midrule
      0:11-0:15  & ``Got me breathin'\textellipsis''* & •//• x2  & & & & \\
     \midrule
     0:16-0:26 & ``(bas)tard, I'm Marilyn\textellipsis'' & ||: ``Thorn'' II :|| & & & & \\
     \midrule
     0:27-0:32 & ``this is rigor\textellipsis'' & •//• x1 & & & & Filter Sweep** \\
     \midrule 
     0:33-0:42 & ``orbit, you an\textellipsis'' & •//• x2 & ||: 2-bar boom-bap† :|| & & & \\
     \midrule
     0:43-0:53 & ``(for)gin' all my\textellipsis'' & •//• x2† & ||: 2-bar boom-bap :|| & Low A Drone & & \\
     \midrule
     0:53-0:58 & ``(suit and) tie  are\textellipsis'' & •//• x1† & •//• x1 & & & ``Hey!'' Off-beats \\
     \midrule
     0:59-1:04 & ``(He) dead! Amen! \textellipsis'' & •//• x1† & •//• x1 & & & •//• \\       
    \midrule
     1:05-1:15 & ``(Ferra)gami\textellipsis'' & •//• x2† & •//• x2 & •//• & & \\
     \midrule
     1:16-1:25 & ``(Wrest)ling? That's\textellipsis'' & •//• x2† & •//• x2 & & & •//•* \\ 
     \midrule
     1:26-1:30 & ``(He) dead! Yup yup!\textellipsis'' & •//• x1† & •//• x1 & •//• & G-A, D-A & \\
     \midrule
     1:31-1:41 & ``Got me breathin'\textellipsis'' & •//• x2† & •//• x2 & •//• & •//• & Delay Throw** \\
     \midrule
     1:43-1:47 & ``Got me breathin'\textellipsis'' & •//• x1† & 2-bar boom-bap & & & \\
     \midrule
     1:48-1:57 & ``I rapped 'em\textellipsis'' & •//• x2† & •//• x2 & •//• & & \\
     \midrule
     1:58-2:08 & ``my casualty, and it's\textellipsis'' & •//• x2† & •//• x2 & & & ``Hey!'' + Dly** \\
     \midrule
     2:09-2:19 & ``And I go visit\textellipsis'' & •//• x2† & 32nd-note fill** & A subs & & ``Hey!'' Off-beats \\
     \midrule
     2:20-2:30 & ``(men)tion, how the far\textellipsis'' & •//• x2† & ||: 2-bar boom-bap :|| & & •//• & •//• \\
     \midrule
     2:31-2:35 & ``(He) dead! Yup yup!\textellipsis'' &  •//• x1† & •//• x1 & & \\  
     \midrule
     2:36-2:41 & ``Got me breathin'\textellipsis'' & •//• x1† & •//•x1 & & &\\
     \midrule
     2:42-2:47 & ``(He) dead! Yup yup!\textellipsis'' & •//• x1† & •//•x1† & & \\ 
     \bottomrule
\end{tabular}

\hfill{*enters at the 2nd sample repetition}

\hfill{**enters on the last two beats}

\hfill{†sample is choked, shifted, or otherwise altered}
    \caption{Roadmap to Kendrick Lamar and Willie B's ``Rigamortis.''}
    \label{tab:2}
\end{table}

\normalsize This devotion to messiness manifests heterogeneity within the limitations of a four-measure loop. The concepts of downbeat, meter, and formal structure are all fraught within ``Rigamortis,'' in a manner that echoes the technological limitations of rap's advent as a genre. Because of this messiness, ``Rigamortis'' sounds how one might expect a mixtape from Compton to sound, reinforcing Lamar's identity with the underground and his roots.

\clearpage
\section{Live-Tracked Case Studies}
The final two case studies investage Milo's 2015 `Rabblerouse,'' produced by Kenny Segal, and Noname's 2018 ``Blaxploitation,'' produced by Phoelix. Neither Segal nor Phoelix use a single lead sample to construct their beat; instead, they live-track their loops with instruments at their disposal. Both producers remain conceptually adherent to the basic beat, and both introduce variety using similar techniques for manipulation to sample-based approaches. Each producer varies the form of their beat against a conceptually mainstream structure, using vocal samples and musical styles to communicate alterity.

\subsection*{\centering Milo's ``Rabblerouse''}

On ``Rabblerouse,'' Milo delivers a single 24-bar verse that operates as an introductory song fragment to the concept album it resides upon. Segal constructs an unstable beat, propelling the listener toward a resolution that arrives with the second track, ``Souvenir.'' The beat ends with a lone vocal sample of the character Yoshimitsu from \textit{Soul Caliber 2} to connect the two tracks. The sample also connects thematically to the penultimate track ``Napping Under the Echo Tree,'' when Milo refers to himself as the ``Yoshimitsu of Boyle Heights.''

\begin{figure}[h]
    \centering
    \includegraphics[width=\textwidth]{images/figures/chp 02/Figure-02.4-Rabblerouse-BB.pdf}
    \caption{The basic beat of ``Rabblerouse.''}
    \label{fig:3.1}
\end{figure}

Segal manifests instability in the beat for ``Rabblerouse'' on several structural levels. First, the loop repeats an irregular six-bar chord progression on a Fender Rhodes. The pattern, though technically \emph{expansional}, sounds unresolved.\footnote{\cite{kyleadamsHarmonicSyntacticMotivic2020}. Adams does not make it a requisite condition of the \emph{expansional} harmonic category to function as a complete phrase, although he notes that it commonly will.} Figure~\ref{fig:3.1} indicates my harmonic reading: the chords function in E Dorian, but the progression leaves out a resolution to F-sharp minor that would close the loop.

``Rabblerouse'' also feels unstable because Milo's verse enters after four bars, and though this is conventional, the six-bar structure of the basic beat offsets the meters being projected by rapper and producer. Segal acccounts for this metric dissonance by repeating the loop six times, shortening the penultimate repetition to four bars to realign the verse's end with the next downbeat. Thereafter, he employs a \emph{sample glitch} where the downbeat is re-triggered four times in a row, extending the E$^{sus4}$ harmony. As Table~\ref{tab:3} illustrates, the unresolved sonority heard from the beginning dissipates as the Yoshimitsu sample is triggered.

\begin{table}
\centering
\tiny
\begin{tabular}{c|c|c|c|c|c|c|c} 
     Timecode & Milo & Electric Drums & Rhodes & Bass I & Synth & Bass II & Vocal Sample \\
     \toprule 
     0:00-0:11 & ``They couldn't\textellipsis''* & 6-bar halftime & 3-chord loop & & & & \\
     \midrule
     0:12-0:23 & ``(merci)ful, I'm\textellipsis'' & •//• & •//• & ||: E-B :|| x3 & fourths & & \\
     \midrule
     0:24-0:35 & ``I might\textellipsis'' & •//• & •//• & •//• & •//• & & \\
     \midrule
     0:36-47 & ``evening, I\textellipsis'' & •//• & •//• & •//• & •//• & Improv** & \\ 
     \midrule
     0:48-0:55 & ``We all\textellipsis''& 4-bar halftime† & 2 chords† & ||: E-B :|| x2† & •//• & & \\
     \midrule
     0:56-1:02 & & 4-bar halftime† & 1 chord† & •//• x1† & Org samp? & Improv & \\
     \midrule
     1:06-1:10 & & & & & & & Yoshimitsu Sample \\
     \bottomrule
\end{tabular}

\hfill{*Entrance at bar 4 with C\#$^{7{sus4}}$}

\hfill{**Entrance anticipates downbeat}

\hfill{†Sample is choked, glitched, or otherwise altered}
    \caption{Roadmap to Milo and Kenny Segal's ``Rabblerouse''}
    \label{tab:3}
\end{table}

\normalsize Segal's fragmentary aesthetic on ``Rabblerouse'' sounds as a mode of alterity because it is uncommon for a hip-hop beat not to function as a closed loop. The metric dissonance projected from the beginning does not resolve by the track's end, necessarily drawing a listener in to the text function and pointing towards the remaining tracks on the album. This beat is unstable alone, but functions cohesively with the sonic palette of Milo's LP \textit{So The Flies Don't Come} as a whole. This technique is predicated upon an interest in lyricism and conceptual unity propagated within the underground hip-hop scene.

\subsection*{\centering Noname's ``Blaxploitation''}

\begin{figure}[h]
    \centering
    \includegraphics[width=\textwidth]{images/figures/chp 02/Figure-02.5-Blaxploitation-BB.pdf}
    \caption{The basic beat of ``Blaxploitation.''}
    \label{fig:4.1}
\end{figure}

Phoelix's production on ``Blaxploitation'' centers around an angular, funk bassline that forms a ``functional circuit'' in C-sharp minor.\footnote{\cite{kyleadamsHarmonicSyntacticMotivic2020}.} The drums and bass sound heterogeneously in spite of the relative sparsity of the orchestration due to the frequency of attacks within a short time frame.\footnote{One dimension of the heterogeneous sound ideal is a ``high density of musical events within a relatively short musical time frame''(See \cite{ollywilsonHeterogeneousSoundIdeal1992}, 329.)} Figure~\ref{fig:4.1} shows the three primary elements of the basic beat, as well as the sympathetic accentual patterns between the drums and bass.

\begin{table}[h!]
    \centering
    \tiny
\begin{tabular}{c|c|c|c|c|c|c|c}
     Timecode & Noname & Bass & Drums & Synth & Organ & BGVs & Vocal Samples \\
     \toprule 
     0:00-0:07 & & & & & & & \textit{Dolemite} I \\
     \midrule
     0:08-00:17 & & 3-chord loop & 4-bar funk & C\#-D\# ost & & & \\
     \midrule
     0:18-0:26 & ``Penny Proud\textellipsis'' & •//• & •//• & & & & \\
     \midrule 
     0:27-0:45 & ``Mmm, yummy\textellipsis'' & ||: 3-chord loop :|| & ||: 4-bar funk :|| & ||: C\#-D\# ost :|| & & & \\
     \midrule
     0:46-1:04 & & •//• & •//• & •//• & & Countermel. & \textit{Dolemite} II \\
     \midrule
     1:05-1:23 & ``Anti-political\textellipsis'' & •//• & •//• & •//• & Improv & & \\
     \midrule
     1:24-1:32 & ``Traded hoodie\textellipsis'' & 3-chord loop & 4-bar funk & C\#-D\# ost & & \\
     \midrule
     1:33-1:51 & & ||: 3-chord loop :|| & ||: 4-bar funk :|| & ||: C\#-D\# ost :|| & & Countermel. & \textit{TSWSBTD} \\
     \midrule
     1:52-2:12 & & 3-chord loop & 4-bar funk & C\#-D\# ost & & \\
     \bottomrule
\end{tabular}
    \caption{Roadmap to Noname and Phoelix's ``Blaxploitation.''}
    \label{tab:4}
\end{table}
As the track's title suggests, the beat of ``Blaxploitation'' is steeped in the sound of the 1970s, grooving in an allosonic nod to the Motown-inspired soundtracks of the film genre. As shown in Table~\ref{tab:4}, the track also autosonically samples dialgoue from two blaxploitation-era films – \textit{Dolemite} (1975) and \textit{The Spook Who Sat by the Door} (1973) – in lieu of hooks. Both of these production decisions instill an air of political consciousness to the track, in keeping with Noname's underground image. 

Joanna Demers notes that the practice of coding the revolutionary politics of Black Americans through a blaxploitation sound is common to hip-hop as an art form. She writes that historically, rappers have ``[monolithically interpreted blaxploitation] films as unified both politically and morally\textellipsis The hip-hop movement neatly compressed [a] more pessimistic view of racial relations under the aegis of Black Power.''\footnote{\cite{joannademersSampling1970sHipHop2003}, 50.} Noname and Phoelix's track participates in this lineage of sounded political consciousness through both autosonic and allosonic techniques.

\section{Looking Forward}
This paper has highlighted some of the ways in which heterogeneity, cohesion, development, and repetition can be mapped into the architecture of underground hip-hop beats. Producers code these phenomena into their music in myriad ways: elements of production that unify the sounded underground are alterations of verse and hook lengths and forms, manipulations of recorded material at the sample level, and additions of musical and textual signifiers alongside the emcee's flow. Via transcriptions that demonstrate both a beat's capacity for development and its repetitive structural layers, I concluded that producers and their beats play a coequal part to the rapper in demonstrating the alterity of underground identity. Future iterations of this project, however, will necessarily turn toward the role of the rapper in sounding the underground.