\onehalfspacing
\section{A Working Definition of Flow}
In comparison to its conceptual counterpart ``the beat'', a ``flow'' resists a clear definition in
academic and technical discourse. While there seems to be some agreement that the beat is comprised
of musically-distinct layers looping throughout a piece, definitions of flow range from anywhere as
specific as ``simply the rhythms and rhymes [a hip-hop song] contains''\footnote{
    \autocite[63]{pauledwardsHowRapArt2009}. While Edwards' statement comes within the context of an
    instruction on rapping (and therefore his reductiveness may be pedagogically useful), this attitude
    toward flow is implied by music academics who transcribe flow primarily on one line of staff notation;
    such transcriptions implicitly show flow as something existing within a fixed (or indiscernible) pitch
    space, as well as primarily existing with a musical meter.} 
to as wide-ranging as ``all of the rhythmical and articulative features of an emcee's delivery of 
the lyrics.''\footnote{
    \cite{kyleadamsMetricalTechniquesFlow2009}.} 
Before I name and discuss a few of the stylistic hallmarks of flow in underground hip-hop, it will be
worthwhile for me to construct a definition that ascertains what musical elements comprise it.

Consider, once more, Open Mike Eagle's characterization of his own relationship to MF DOOM's flow:
that Eagle has to be careful with it because of how much time he has spent `in DOOM's mind.'\footnote{
    \cite{estellecaswellRappingDeconstructedBest2016}. For my earlier discussion of this passage and its
    relationship to meditative listening, see Section~\ref{listeningasmeditation}.} 
The fact that Eagle can characterize his relationship to the style of DOOM's flow demonstrates that
flow encompasses more than rhythm and rhyme alone; by themselves these elements do not account for flow
as being tied to an emcee's specific or generic style. Flow, then, should be conceived more broadly than
the manifestation of rhythms through the vocal delivery of rhymes.

At the same time, however, emcees frequently describe flow as text in relation to the music, drawing
contrasts with the non-musically bounded epithet ``poetry.'' The emcee Rakim's oft-cited definition
of rap (that it is ``rhythm and poetry'') creates a distinction between texts that can be rapped and
those that are conceived within the poetic medium. Clarifying this distinction, the emcee Myka 9 of
Freestyle Fellowship offers, ``sometimes I might write a poem, a spoken-word poem, but then morph
that into a rap rhythmically.''\footnote{
    \autocite[63]{pauledwardsHowRapArt2009}.}
Myka 9's insight helps to construct a continuum for rapped text: it exists on a spectrum from
texts that are spoken in musically-bounded ways and non-musically-bounded ways.\footnote{
    Interestingly, rapped verses can be and often are delivered on varying degrees of this
    spectrum. Mitchell Ohriner notes two distinct modes of delivery, which he calls speech-rhythmic
    and music-rhythmic, based on the degree of non-alignment between the musical meter and the 
    degree to which syllable onsets correspond to metrical positions (see 
    \cite{mitchellohrinerLyricRhythmNonalignment2019}.}

This distinction between the construction of the rap as text and its performance as music is
foundational to the method by which I interrogate the concept of flow in this chapter. I believe
the techniques underground emcees employ when rapping can be neatly divided and then analyzed based
on this division, into categories I refer to as \emph{structural} and \emph{performance} techniques.
Structural techniques deal primarily with rap as text, considering its syntactic elements with primacy
over their manifestation as a stream in the musical object. By contrast, performance techniques deal
primarily with rap as music, considering the ways in which an emcee makes manifest the structures
they contrive when composing texts. Moreover, performance techniques encapsulate the decisions made
when treating the voice as an object within a digital recording. With these distinct categories,
I aim to examine how, as Mitchell Ohriner claims, flow ``encompasses phrasing, rhythm, meter, accent,
patterning, and groove, not to mention the relations among these parameters.''\footnote{
    \autocite[28]{mitchellohrinerFlowRhythmicVoice2019}.}

\section{Thesis Statement \& Definitions}
In this chapter, I examine a few notable techniques that underground emcees employ in structuring
and performing their flow, arguing that, like producers, their choice to use these techniques reaches
listeners as underground. The four techniques I define below do not form an exhaustive list, but rather
are salient and illustrative in how they relate to my conception of the subgenre. My terms are also 
divided into the subcategories of structural and performance techniques, a distinction I draw in my 
understanding of how emcees work with the text that makes up their flow. When the underground emcee 
inhabits the role of writer, they experiment with structuring techniques including \emph{pivot rhyme} 
and \emph{closing fragmentation} to craft their verses. Stepping into the recording booth or onto the 
stage, the emcee shifts into their role as rapper and therefore draws on performance techniques such 
as \emph{mimesis} and \emph{processing} to shape their vocal delivery. Each of these techniques serves
as the focus one of the close-readings that follow, so I will clarify their functions before I examine
repertory.

As a device for constructing a verse, a pivot rhyme allows an emcee to execute a shift in end rhyme;
it occurs when the concluding words of the previous bar conjure up an idea that is semantically linked 
to the bar's topic but does not rhyme. In this instance, the rapper chooses to displace the lyric past
the next bar line, allowing that word or phrase to serve as the primary rhyming sound within a new repetition
of the beat pattern. Often the emcee uses this instance as a way to play with audience expectation, taking
the listener's focus on this final rhyme to shift towards a topic semantically distinct from the previous bar.

Underground emcees employ closing fragmentation as another method of structuring a verse, particularly
to mark the end of some sort of formal section. To accomplish this, emcees will deliver a ``full'' textual
phrase, usually structured as a setup and punchline, then signal a close through breaking up and repeating 
the component elements of that textual phrase in a more improvisatory and loosely-organized style of delivery.
Fragmentation and repetition here do not function as they do in the construction of a hook; rather, they
demonstrate to the listener that some larger textual unit\textemdash a phrase, a verse, the song 
itself\textemdash is ending.

With both performance techniques, emcees articulate the role of their voice as a layer \emph{amongst} the rest 
of the mix, rather than the principal element within it. In particular, the use of mimesis, or stylizing vocal 
delivery to mimic other elements in the beat layer, draws attention to the other musical elements in an attempt
to foreground their musicality. This promotes the other musical layer to a co-soloist role, rather than 
functioning as a background loop or layer, allowing the listener to focus on it as more than just accompanimental
structure beneath the vocal flow.

Processing more generally refers to the manipulation of digital audio after its recorded (especially through
the use of EQ, compression, and reverb software or hardware), but here I use it to refer to methods of altering
the vocal signal as a method of electronic composition. In particular, rappers use processing effects like delay, 
distortion, and pitch transposition to alter the audio of their voice during or after tracking it to treat it 
as a musically manipulable element; the vocals here function as \emph{sound} inasmuch as they do text.\footnote{
    My definition of processing here is a broader version of the definition of ``glitch'' in Chapter 2 
    (see p.~\pageref{glitch}.)}
In this way, processing treats the voice as if it were any other musical layer and, as a result, any layer can
be listened with as much consideration as the voice.


\newpage
\section{Structuring Techniques}
\phantomsection
\subsection*{\centering Madvillian -- ``Great Day''}
\addcontentsline{toc}{subsection}{Madvillain's ``Great Day''}

\begin{itemize}
    \item Transcription of flow and beat alignment for setup and punchline for ``What you might put on 
    movie food\textellipsis Uh, what is jalopenos?'' and ``One thing this party could use is more\textellipsis
    booze''
    \item DOOM uses a \emph{linkage rhyme} to humorous effect in both instances: the first refers to 
    the characterization of his own flow as ``buttery'', the second to rap's stereotypical lyrical content,
    ``bitches''
    \item In both instances, the different word becomes a new device for ordering the remainder of the 
    verse: ``jalapenos'' / ``holla at your seniors'' / ``hasish fienda'' / ``grass is greener'' and 
    ``booze'' / ``shoes'' / ``don't use''
\end{itemize}

\phantomsection
\subsection*{\centering Armand Hammer (ft. R.A.P. Ferreira) -- ``Dead Cars''}
\addcontentsline{toc}{subsection}{Armand Hammer and R.A.P. Ferreira's ``Dead Cars''}

\section{Performance Techniques}
\phantomsection
\subsection*{\centering R.A.P. Ferreira -- ``NONCIPHER''}
\addcontentsline{toc}{subsection}{R.A.P. Ferreira's ``NONCIPHER''}
    \begin{itemize}
        \item Transcription of ``framing region'' and Verse 2, Vocal and Sax Lines
        \item Rory draws the ear to the saxophone work, live-tracked by Aaron shaw
        \item Demonstrating his own musical ear; showing that flow has a melodic dimension
    \end{itemize}
  
\phantomsection
\subsection*{\centering Moor Mother and billy woods (ft. Elucid)  -- ``Tiberius''}
\addcontentsline{toc}{subsection}{Moor Mother, billy woods, and Elucid's ``Tiberius''}