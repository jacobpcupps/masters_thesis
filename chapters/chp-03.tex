%\onehalfspacing
\section{Thesis Statement}
    \begin{itemize}
        \item Emcees can inhabit either the role of vocalist and lyricist
        \item Underground emcees play with techniques in each role
        \item Structuring Techniques: \emph{linkage rhymes}, \emph{non-hook fragmentation*}
        \item Performance Techniques: \emph{mimesis}, \emph{processing}
    \end{itemize}

\section{Structuring Techniques}

\phantomsection
\subsection*{\centering Madvillian -- ``Great Day''}
\addcontentsline{toc}{subsection}{Madvillain's ``Great Day''}

\begin{itemize}
    \item Transcription of flow and beat alignment for setup and punchline for ``What you might put on 
    movie food\textellipsis Uh, what is jalopenos?'' and ``One thing this party could use is more\textellipsis
    booze''
    \item DOOM uses a \emph{linkage rhyme} to humorous effect in both instances: the first refers to 
    the characterization of his own flow as ``buttery'', the second to rap's stereotypical lyrical content,
    ``bitches''
    \item In both instances, the different word becomes a new device for ordering the remainder of the 
    verse: ``jalapenos'' / ``holla at your seniors'' / ``hasish fienda'' / ``grass is greener'' and 
    ``booze'' / ``shoes'' / ``don't use''
\end{itemize}

\phantomsection
\subsection*{\centering Armand Hammer (ft. R.A.P. Ferreira) -- ``Dead Cars''}
\addcontentsline{toc}{subsection}{Armand Hammer and R.A.P. Ferreira's ``Dead Cars''}

\section{Performance Techniques}
    \begin{itemize}
        \item Underground emcees stress the role of performative role of their voices as a layer
        \emph{amongst} the sum total of the rap song. 
        \item They emphasize this through mimesis, vocally mimicking other musical elements in a way
        that calls attention to them, and also through processing, the digital manipulation of their 
        vocal audio as if it were any other layer of the mix
        \item These techniques compound with the structuring techniques
    \end{itemize}

\phantomsection
\subsection*{\centering R.A.P. Ferreira -- ``NONCIPHER''}
\addcontentsline{toc}{subsection}{R.A.P. Ferreira's ``NONCIPHER''}
    \begin{itemize}
        \item Transcription of ``framing region'' and Verse 2, Vocal and Sax Lines
        \item Rory draws the ear to the saxophone work, live-tracked by Aaron shaw
        \item Demonstrating his own musical ear; showing that flow has a melodic dimension
    \end{itemize}
  
\phantomsection
\subsection*{\centering Moor Mother and billy woods (ft. Elucid)  -- ``Tiberius''}
\addcontentsline{toc}{subsection}{Moor Mother, billy woods, and Elucid's ``Tiberius''}