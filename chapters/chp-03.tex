\onehalfspacing

%\newpage
\section{Thesis Statement \& Definitions}
Like producers, underground emcees construct identity beyond the semantic meaning of their lyrics
through a set of distinctive techniques, a few of which I name and define in the paragraphs below. 
I divide these techniques into subcategories based on the role that serves as the emcee's primary 
focus. When the underground emcee inhabits the role of writer, they play with structuring techniques 
including  \emph{pivot rhyme} and \emph{closing fragmentation} to craft their verses. Stepping into
the booth or onto the stage, the emcee shifts into their role as the rapper and therefore draws on
performance techniques such as \emph{mimesis} and \emph{processing} to shape their vocal delivery. 
Each of these techniques serves as the focus one of the close-readings that follow, so I will clarify
their function before I examine repertory.

As a device for constructing a verse, a pivot rhyme allows an emcee to navigate a shift in end rhyme;
it occurs when the conclusion of the previous bar conjures up an idea that is semantically linked to
the topic but does not rhyme. In this instance, the rapper chooses to displace the lyric passed the next
barline, allowing that word or phrase to serve as the new basis for a rhyme within a new repetition of
the beat pattern. Often the emcee uses this instance as a way to play with audience expectation, taking
the listener's focus on this final rhyme to shift toward a topic that seems disparate or unexpected.

Underground emcees employ closing fragmentation as another method of structuring a verse, particularly
to mark the end of some sort of formal section. To accomplish this, emcees will deliver a ``full'' textual
phrase, usually structured as a setup and punchline, then signal close through breaking up and repeating the
component elements of that textual phrase in a more improvisatory and loosely-organized style of delivery.
Rather than granting that textual idea hook-like function, the use of fragmentation signals the close of
a larger textual unit: a phrase, a verse, or even the song itself.

    \begin{itemize}
        \item Underground emcees stress the role of performative role of their voices as a layer
        \emph{amongst} the sum total of the rap song. 
        \item They emphasize this through mimesis, vocally mimicking other musical elements in a way
        that calls attention to them, and also through processing, the digital manipulation of their 
        vocal audio as if it were any other layer of the mix
        \item These techniques compound with the structuring techniques
    \end{itemize}


%\newpage
\section{Structuring Techniques}
\phantomsection
\subsection*{\centering Madvillian -- ``Great Day''}
\addcontentsline{toc}{subsection}{Madvillain's ``Great Day''}

\begin{itemize}
    \item Transcription of flow and beat alignment for setup and punchline for ``What you might put on 
    movie food\textellipsis Uh, what is jalopenos?'' and ``One thing this party could use is more\textellipsis
    booze''
    \item DOOM uses a \emph{linkage rhyme} to humorous effect in both instances: the first refers to 
    the characterization of his own flow as ``buttery'', the second to rap's stereotypical lyrical content,
    ``bitches''
    \item In both instances, the different word becomes a new device for ordering the remainder of the 
    verse: ``jalapenos'' / ``holla at your seniors'' / ``hasish fienda'' / ``grass is greener'' and 
    ``booze'' / ``shoes'' / ``don't use''
\end{itemize}

\phantomsection
\subsection*{\centering Armand Hammer (ft. R.A.P. Ferreira) -- ``Dead Cars''}
\addcontentsline{toc}{subsection}{Armand Hammer and R.A.P. Ferreira's ``Dead Cars''}

\section{Performance Techniques}
\phantomsection
\subsection*{\centering R.A.P. Ferreira -- ``NONCIPHER''}
\addcontentsline{toc}{subsection}{R.A.P. Ferreira's ``NONCIPHER''}
    \begin{itemize}
        \item Transcription of ``framing region'' and Verse 2, Vocal and Sax Lines
        \item Rory draws the ear to the saxophone work, live-tracked by Aaron shaw
        \item Demonstrating his own musical ear; showing that flow has a melodic dimension
    \end{itemize}
  
\phantomsection
\subsection*{\centering Moor Mother and billy woods (ft. Elucid)  -- ``Tiberius''}
\addcontentsline{toc}{subsection}{Moor Mother, billy woods, and Elucid's ``Tiberius''}