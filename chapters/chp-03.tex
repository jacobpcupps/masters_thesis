\onehalfspacing
%\newpage
\section{Thesis Statement \& Definitions}
Like producers, underground emcees construct identity beyond the semantic meaning of their lyrics
through a set of distinctive techniques, a few of which I name and define in the paragraphs below. 
I divide these techniques into subcategories based on the role that the emcee primarily inhabits. 
When the underground emcee inhabits the role of writer, they experiment with structuring techniques 
including  \emph{pivot rhyme} and \emph{closing fragmentation} to craft their verses. Stepping into 
the recording booth or onto the stage, the emcee shifts into their role as rapper and therefore 
draws on performance techniques such as \emph{mimesis} and \emph{processing} to shape their vocal delivery. 
Each of these techniques serves as the focus one of the close-readings that follow, so I will clarify
their functions before I examine repertory.

As a device for constructing a verse, a pivot rhyme allows an emcee to execute a shift in end rhyme;
it occurs when the concluding words of the previous bar conjure up an idea that is semantically linked 
to the bar's topic but does not rhyme. In this instance, the rapper chooses to displace the lyric past
the next bar line, allowing that word or phrase to serve as the primary rhyming sound within a new repetition
of the beat pattern. Often the emcee uses this instance as a way to play with audience expectation, taking
the listener's focus on this final rhyme to shift towards a topic semantically distinct from the previous bar.

Underground emcees employ closing fragmentation as another method of structuring a verse, particularly
to mark the end of some sort of formal section. To accomplish this, emcees will deliver a ``full'' textual
phrase, usually structured as a setup and punchline, then signal a close through breaking up and repeating 
the component elements of that textual phrase in a more improvisatory and loosely-organized style of delivery.
Fragmentation and repetition here do not function as they do in the construction of a hook; rather, they
demonstrate to the listener that some larger textual unit\textemdash a phrase, a verse, the song 
itself\textemdash is ending.

With both performance techniques, emcees articulate the role of their voice as a layer \emph{amongst} the rest 
of the mix, rather than the principal element within it. In particular, the use of mimesis, or stylizing vocal 
delivery to mimic other elements in the beat layer, draws attention to the other musical elements in an attempt
to foreground their musicality. This promotes the other musical layer to a co-soloist role, rather than 
functioning as a background loop or layer, allowing the listener to focus on it as more than just accompanimental
structure beneath the vocal flow.

Processing more generally refers to the manipulation of digital audio after its recorded (especially through
the use of EQ, compression, and reverb software or hardware), but here I use it to refer to methods of altering
the vocal signal as a method of electronic composition. In particular, rappers use processing effects like delay, 
distortion, and pitch transposition to alter the audio of their voice during or after tracking it to treat it 
as a musically manipulable element; the vocals here function as \emph{sound} inasmuch as they do text.\footnote{
    My definition of processing here is a broader version of the definition of ``glitch'' in Chapter 2 
    (see p.~\pageref{glitch}.)}
In this way, processing treats the voice as if it were any other musical layer and, as a result, any layer can
be listened with as much consideration as the voice.


\newpage
\section{Structuring Techniques}
\phantomsection
\subsection*{\centering Madvillian -- ``Great Day''}
\addcontentsline{toc}{subsection}{Madvillain's ``Great Day''}

\begin{itemize}
    \item Transcription of flow and beat alignment for setup and punchline for ``What you might put on 
    movie food\textellipsis Uh, what is jalopenos?'' and ``One thing this party could use is more\textellipsis
    booze''
    \item DOOM uses a \emph{linkage rhyme} to humorous effect in both instances: the first refers to 
    the characterization of his own flow as ``buttery'', the second to rap's stereotypical lyrical content,
    ``bitches''
    \item In both instances, the different word becomes a new device for ordering the remainder of the 
    verse: ``jalapenos'' / ``holla at your seniors'' / ``hasish fienda'' / ``grass is greener'' and 
    ``booze'' / ``shoes'' / ``don't use''
\end{itemize}

\phantomsection
\subsection*{\centering Armand Hammer (ft. R.A.P. Ferreira) -- ``Dead Cars''}
\addcontentsline{toc}{subsection}{Armand Hammer and R.A.P. Ferreira's ``Dead Cars''}

\section{Performance Techniques}
\phantomsection
\subsection*{\centering R.A.P. Ferreira -- ``NONCIPHER''}
\addcontentsline{toc}{subsection}{R.A.P. Ferreira's ``NONCIPHER''}
    \begin{itemize}
        \item Transcription of ``framing region'' and Verse 2, Vocal and Sax Lines
        \item Rory draws the ear to the saxophone work, live-tracked by Aaron shaw
        \item Demonstrating his own musical ear; showing that flow has a melodic dimension
    \end{itemize}
  
\phantomsection
\subsection*{\centering Moor Mother and billy woods (ft. Elucid)  -- ``Tiberius''}
\addcontentsline{toc}{subsection}{Moor Mother, billy woods, and Elucid's ``Tiberius''}