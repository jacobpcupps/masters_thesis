\documentclass[12pt]{report}

%---------Packages----------%

\usepackage[utf8]{inputenc}
\usepackage[T1]{fontenc}
\usepackage[english]{babel}
\usepackage[letterpaper, margin=1in]{geometry}
\usepackage[normalem]{ulem}
\usepackage{lmodern, csquotes, graphicx, lipsum, booktabs, amsmath, setspace, rotating}

%---------TitleSec----------%

\usepackage[explicit]{titlesec}
\titleformat{\chapter}[display]{\bfseries\centering}{\Huge Chapter \thechapter:}{1em}{\Huge #1}

%---------FancyHdr----------%

%\usepackage{fancyhdr}
%\pagestyle{fancy}
%\fancyhead{}
%\fancyhead[L]{\nouppercase{\leftmark}}
%\fancyhead[R]{Sound of the Underground}
%\fancyfoot{}
%\fancyfoot[L]{Jacob P. Cupps}
%\fancyfoot[R]{\thepage}
%\renewcommand{\headrulewidth}{0.4pt}
%\renewcommand{\footrulewidth}{0.4pt}
%\renewcommand{\chaptermark}[1]{\markboth{#1}{}}

%---------BibLaTeX----------%

\usepackage[notes, backend=biber]{biblatex-chicago}
\bibliography{references.bib}

%---------Epigraph----------%

\usepackage{epigraph}
\AtBeginDocument{\renewcommand {\epigraphflush}{center}}
\setlength \epigraphwidth {0.6\linewidth}
\setlength\epigraphrule{0pt}

%---------Verse------------%

\usepackage{verse}
\renewcommand*{\theHpoemline}{\arabic{verse@envctr}.\arabic{poemline}}

%---Hyperref, after Verse--%

\usepackage[hidelinks]{hyperref}

%----------DOCUMENT---------%

\begin{document}
\pagenumbering{roman}
\thispagestyle{empty}
\begin{titlepage}
    \begin{center}
        \small Washington University in St. Louis \\
        School of Arts \& Sciences \\
        Department of Music \\
        \vspace*{1.25in}
        \Large ``The Sound of the Underground'': \\
        \vspace{0.1cm}
        \large Distinguishing Alternative Identity in \\ 
        Rap's Contemporary Moment  \\
        \vspace{0.5cm}
        \normalsize Jacob P. Cupps \\
        \vfill
        \includegraphics[width=0.3\textwidth]{images/misc/header_logo.png} \\
        \vspace{0.8cm}
        \small A thesis presented to \\ The Graduate School \\ of Washington University in \\ partial fulfillment of the \\ requirements for the degree \\ of Master of Arts \\
        \vspace{0.8cm}
        \normalsize May 2022 \\
        St. Louis, Missouri
    \end{center}
\end{titlepage}
\frenchspacing %\doublespacing

%-------Front Matter--------%

\setcounter{page}{2}
  %----_TOC, Lists-----%
    \tableofcontents
    \listoffigures
    \addcontentsline{toc}{chapter}{List of Figures}
    \listoftables
    \addcontentsline{toc}{chapter}{List of Tables}

  %--Acknowledgements--%
    \chapter*{Acknowledgements}
    \addcontentsline{toc}{chapter}{Acknowledgements}

This project is indebted to an always-growing number of people who influence, encourage, 
and inspire me. First, I would like to thank my advisor for this project, Paul Steinbeck, 
as well as my second readers, Lauren Eldridge Stewart and Christopher Stark. The classes 
I've taken as well as the time I've spent in discussion with each of you have helped shape
this project in distinct and insightful ways. My thanks also goes out to other academic 
mentors I've connected with at Washington University in St. Louis. I extend my thanks to 
Paula Harper and Alexander Stefaniak, both of whom read and provided feedback on papers 
and abstracts that informed my research early on in its development.
 
I adamantly believe this project would not be what it is without the camaraderie of the 
graduate students in Washington University's department of music. I am thankful to all my 
colleagues for the many hours of discussion and collaboration inside and outside of class,
in addition to their commitment toward making our department an inclusive and amenable space.
My special thanks go out to my Eden Attar and Varun Chandrasekhar; I appreciate both of you
for the time spent writing with me, listening to the music I studied, and giving me space to
workshop my ideas.

My friends and family have created an integral community of support around me throughout my 
completion of this project and my time in academia in general. There are more folks to name 
than just the ones here, but a few who have been indispensable to me are my partner Haley 
Jo Mack and my friend Emory Parker; you both have been there for me as readers, confidants, 
and\textemdash more than anything\textemdash a support system. I am similarly thankful to my 
brother and father, Justin and Brian Cupps: you two keep me grounded and have helped me
immensely throughout my time in this master's program. I would like to dedicate this work to
the four of you, for, truly, I am able to do what I do because of how you all have shaped me.



\vspace{0.2cm}
\hfill{Jacob P. Cupps}

\noindent \textit{Washington University in St. Louis}

\noindent \textit{May 2022}

  %-----Abstract-----%
    \chapter*{Abstract}
    
    \begin{center}
        \large ``The Sound of the Underground'': \\
        \large Distinguishing Alternative Identity in Rap's Contemporary Moment  \\
        \normalsize by \\
        Jacob P. Cupps \\
        Master of Arts in Music Theory \\
        
        Washington University in St. Louis, 2022 \\
    \end{center}
    
    \addcontentsline{toc}{chapter}{Abstract}

    \normalsize
    This thesis focuses on transcribing and analyzing a repertory of underground rap songs, examining
    how the quality of undergroundness manifests within beats and flows. I define underground as a
    subgenre of hip-hop that is unified by a culture of meditative listening, and I opt for it over
    related terms because it is style agnostic. Therefore, I am able to investigate how a small set
    of techniques manifest across a wide variety of approaches to making underground hip-hop. I argue
    that these techniques are a method of sounding the underground, communicating meaning to the 
    listener at the level of the song.

    After an introductory Chapter 1, the subsequent chapters examine techniques used in underground
    production and underground emceeing respectively. Chapter 2 argues that producers employ techniques
    I term resampling/recomposing, choking, glitching, and slipping to create beats that use non-standard
    song forms and prioritize an aesthetic of internal variety and contrast; this heterogeneous sound
    promotes producers to co-authors in the construction of musical identity. Chapter 3 posits that emcees
    use techniques I term pivot rhyme, closing fragmentation, mimesis, and processing to engage listener's
    expectations and challenge the perception that flow is the most hierarchically significant element in a
    rap song. This demonstrates an emcee's reciprocity in the construction of their own identity, inviting
    listeners to deeply engage with how the music informs the resulting musical persona.


%---------------------------%
\chapter{Locating the Underground in Styles, Histories, and Practices}
\pagenumbering{arabic}
\onehalfspacing

\epigraph{{``Listening correctly can cause introspecting.''}}{R.A.P. Ferreira}

\section{Open Mike Eagle and Nocando's ``Unapologetic''}

Before landing several projects on best-of-the-decade lists from outlets like \textit{Pitchfork},
\textit{Brooklyn Vegan}, and \textit{Hip-Hop Golden Age}, the L.A.-based rapper Open Mike Eagle 
started off the decade with a declaration. His first studio LP, \textit{Unapologetic Art Rap}, coined
the subgeneric term ``art rap,'' which Eagle defines as ``a shorthand for leftfield and avant-garde rap 
music'' that he ``has spent his career redefining and expanding.''\footnote{
    \cite{openmikeeagle}.}
And not just Eagle himself, but so too the world of American hip-hop fandom around him: in 2020, the 
former Editor-in-Chief of \textit{Rap Genius} Shawn Setaro claimed that: ``[Art Rap] is a movement that
has become one of the most vital things happening in hip-hop today. Boundaries are stretched constantly,
in a way that recalls the innovations that took place in rap's Golden Age, without aping that era's 
sonics.''\footnote{\cite{shawnsetaroWhyYouShould2020}.}

If Setaro's claim holds merit, then we should interrogate how the music being received as art rap comes
to be known in that way: not just in its idiomatic ``sonics'' but also its themes, techniques, and 
approaches. Max Bell attempted this by digging through Bandcamp's ``art rap'' genre tag for 
\textit{Bandcamp  Daily.} After purporting then qualifying that art rap ``is antithetical to 
terrestrial radio station playlists,'' he notes a few unifying stylistic traits ``including but not limited
to left-field, forward thinking production, unconventional song structures and cadences, songs written 
from the perspectives of fictional characters, explicit and protracted engagement with social and political
issues, and absurdist metaphors and similes.''\footnote{
    \cite{maxbellWalkAvantGardeWorld2017}.} 
Bell's attempt at a definition is somewhat helpful, but its breadth offers only a partial picture, especially
due to the qualifications it must make. It seems, more than a decade later, the hip-hop community is still 
asking the question posed in Eagle's seventh track: ``WTF is Art Rap?''

Eagle and the guest emcee Nocando offer a glimpse at a definition within the lyrical content of the following
track, ``Unapologetic.'' As art-rappers, Nocando claims that they are small-time enough to identify with 
an underground scene:
\settowidth{\versewidth}{I dress nice but I ain't no goddamned sneaker freak.}
    \begin{verse}[\versewidth]
        I'm underground, man, like I'm beneath the streets, \\ 
        I dress nice but I ain't no goddamned sneaker freak.\footnote{
            \cite{openmikeeagle2010}.}
    \end{verse}
At the same time, the two are working at a level above the rappers who are friends with the local 
``scenesters'':
\settowidth{\versewidth}{There's a reason I'm performing after him.}
    \begin{verse}[\versewidth]
        Now everybody's got a rapper friend, \\ 
        But my name's higher on the flyer,\\
        There's a reason I'm performing after him.
    \end{verse}
On Eagle's end, he envisions art rap as dedicated to a particular audience:
\settowidth{\versewidth}{Who make sketches and daydream the whole class.}
    \begin{verse}[\versewidth]
        This to all the adolescent negro lads, \\
        Who make sketches and daydream the whole class. \\
        In '96, they would have been De La Soul Fans, \\
        (In 2010), it's My Chemical Romance.
    \end{verse}
At the same time, he is troubled by a whiter (and perhaps less-initiated) audience his art rap 
parties have attracted:
\settowidth{\versewidth}{Came to watch us battle and cackle like Fran Drescher.}
    \begin{verse}[\versewidth]
        Scene's built on the backs of black rappers, \\
        (Somehow) when I'm asking for cash you can't answer. \\ 
        Descendants of the folks that shackled my ancestors, \\ 
        Came to watch us battle and cackle like Fran Drescher.
    \end{verse}
The above quatrain also shows Eagle's concern with financial success. Notably, however, Eagle has a clear 
conception of a type of rap music that has sold itself over to the entertainment industry:
\settowidth{\versewidth}{Cause my little brother never heard of Little Brother \textellipsis}
    \begin{verse}[\versewidth]
        Cause my little brother never heard of Little Brother \textellipsis \\ 
        The only [rappers] he can discover \\
        Are the ones who please Viacom's executive n----- lovers.
    \end{verse}

Even from a brief lyrical analysis, ``Unapologetic'' creates a complex picture of the (black, masculine) identities
of the emcees on the track. They clearly work to textually communicate this, but text alone does not construct identity
in rap: musical content plays a role in this. This chapter works considers ``Unapologetic'' as a case study for the music
the thesis overall intends to address. Although (as my title suggests) I will eventually come to contextualize the track
within a subgenre I call underground, I first consider its relationship to some categories in rap which have received
attention in music scholarship in the past: namely, Amanda Sewell's conception of the ``nerdcore'' sugenre, and Adam Krims
and Justin Williams ``jazz/bohemian'' rap constructions. I come to reject these designations for ``Unapologetic'' because
of the limitations of contextualizing it within a single sound or style, instead looking towards listening practices in
hip-hop to come to a definition of underground that will serve as a guide for what this thesis explores.


\section{Nerdcore, Jazz-Rap, and the ``Golden Age''} \label{nerdcorejazzrapgoldenage}

For me as a listener, ``Unapologetic'' engenders two connections to the nerdcore subgenre. First, its beat 
layer is primarily constructed from two synthesizer loops. Although synthesizer does not explicitly
correlate to ``nerdiness'' in most contexts, when deployed within a hip-hop beat, it draws my attention
towards the technical prowess of the producer. This hearing also helps contextualize concerns of intelligence
made by both emcees. Nocando puts his own intelligence on display moving through a wide variety of allusions, 
including the Hanna-Barbera cartoon \textit{Tom and Jerry}, Joshua Marston's spanish-language film 
\textit{María, llena eres de gracia}, and even a rewritten version of Psalm 23.\footnote{
    According to Rory Ferreira, Nocando's re-contextualizaiton of Biblical language, here, might also 
    allude to Friedrich Nietzsche's \textit{Twilight of the Idols}, where he also plays with language
    from the same Psalm (Genius Annotation?).} 
Though less topically diffuse, Eagle's verses focus on intelligence by painting a picture of the listener he
attempts to reach: the ``adolescent Negro lads / who make sketches and daydream the whole class.''

The emcees' focus on intelligence through these contrasting approaches relates to the generic construction of
nerdcore. According to Amanda Sewell, ``[n]erdcore hip-hop happens when technologically savvy, verbally precocious,
and social marginalized people begin to make hip-hop using their skill sets and experiences.''\footnote{
    \autocite[223]{amandasewellNerdcoreHiphop2015}.}
It's also notable that
   
   \begin{itemize}
    \item Sewell's conception of nerdcore:
            \begin{itemize}
                \item MC Frontalot ``[As a nerd, you] don't fit into the traditional expectations of 
                masculinity, especially as a teenager'' (226). Compare to OME's ``adolescent negro lads who 
                daydream the whole class''
            \end{itemize}
        \item OME's ``Art Rap Party'' liner notes video:
            \begin{itemize}
                \item ``My idea of an art rap party, which is somewhat imaginary and hasn't all the way
                happened yet, is the kind of function where, like, intelligent folks go to dig on some
                intelligent music, \textellipsis it's not a dance party, you know what I mean?''
                \item ``  It's the kind of party where information's exchanged between people[;] 
                information's exchanged between performer and audience, and back and forth'' 
                \item ``Of course it's gonna be a little bit nerdy[.] \textellipsis We could talk about 
                anime, \textellipsis political stuff, exploring realms conversationally that aren't in the 
                typical rap song''
            \end{itemize}
    \end{itemize}

Adam Krims constructs a variety of subgeneric designations for rap that he argues started to concretize
around 1994, each harnessing distinct musical styles to address particular topics. In particular, he 
notes that the genres he designates ``jazz/bohemian'' rap and ``reality'' rap articulated themselves 
as alternatives  to ``gangsta'' rap based on the perception that it was the genre most aligned with
a commercial mainstream.\footnote{
    \autocite[64--65]{adamkrimsRapMusicPoetics2000}. Krims offers two case studies\textemdash The Roots' 
    ``What They Do'' and RZA's introduction on Wu-Tang Clan's \textit{Wu-Tang Forever}\textemdash as 
    instances of each genre directly addressing what it conceived to be the commercial mainstream at
    the time.}
In much the same way, Eagle's criticism of the types of artists that please Viacom's executives and his 
estimation of his own work as a ``Black man's art'' allow him to construct the same dichotomy; he and
other rappers like him are bohemians within the music industry, and you, the listener who can appreciate
his style, are a connoisseur of ``fine art'' rap music.

    \begin{itemize}
        \item OME's definitions (and genre designation) construct art rap as a type of high art taken with 
        respect to the gamut of hip-hop in the 2010s
        \item Another way in which hip-hop musicians have constructed their sound to be heard as a high art
        within  hip-hop communities is through sonic links to jazz and related Black sonic profiles (Table 1.1)
            \begin{itemize}
                \item As early as 2010, artists including Kendrick Lamar and billy woods are sampling 
                from  mainstream jazz in the seventies and eighties
                \item As the decade progressed, other artists including R.A.P. Ferreira (fka Milo, aka 
                Scallops Hotel)  and Armand Hammer (the rap duo consisting of woods and ELUCID) begin to 
                draw on more eclectic styles of jazz, especially Sun Ra
                \item Soul and blues samples proliferate throughout the decade as well
            \end{itemize}
        \item Williams (2015) argues that, during the golden era of hip-hop, artists used jazz and other 
        Black sonic profiles in order to construct their work as a ``high art'' substyle within the genre
            \begin{itemize}
                \item Jazz rap came into formation as an ``alernative'' to gansta and pop rap, and jazz in 
                the decade  preceding had become an institutional and ``serious'' art form (48--52).
                \item Certain musical elements function as ``jazz codes'' which, through performance and 
                timbral qualities, reach audience members as having a jazz \emph{feel}, even if they are 
                not drawn from jazz records, strictly speaking (55).
                \item Groups like A Tribe Called Quest and Digable Planets drew on these codes to (explictly 
                or subtextually) reach listeners as artists with an identity that was different than those of 
                their contemporaries working in pop/gangsta rap styles. (58)
            \end{itemize}        
        \item While I'm not convinced that jazz rap's renaissance in the 2010s was explicitly motivated to construct 
        jazz rap as a type of high art uniformly, I think the same urge to \emph{differentiate} one's self to the 
        listener, to articulate identity in a way that goes against some construction of identity in the mainstream, 
        still exists as a driving force in the 2010s.
    \end{itemize}
    
The sonic profile of ``Unapologetic'' does not fit neatly into the generic category Krims and Williams
construct. While both acknowledge the possibility for a stylistic eclecticism beyond that reaches beyond
jazz sonically, the groups that receive their focus (A Tribe Called Quest, De La Soul, Digable Planets, 
Jungle Brothers, and The Roots) all rely to \emph{some} degree on the sound of jazz in their music. 
Without the sonic trappings of some jazz codes signify some relationship to jazz in sound, 
``Unapologetic'' cannot reach listeners as ``bohemian''\textemdash at least not in the same way 
the groups connected to the ``Golden Age'' did. I reject that ``Unapologetic'' can neatly
fit in within one sound or style, so I will instead look for some sort of practice in the music
which can be emulated.

Mike Eagle's ``Unapologetic''\textemdash his whole conception of art rap as a moment at the start
of the 2010s\textemdash articulates something that is at once new in hip-hop while relating to ideas,
sounds, and genres from throughout hip-hop's history. Although the track's ``tech-y'' sound and 
construction of ``intelligence'' intimates some connection to nerdcore, the designation falls short 
because of nerdcore's historically white positionality and lack of concern with its own construction 
as art. On the other hand, to situate ``Unapologetic'' within the lineage of jazz/bohemian rap does 
not adequately account for the  timbral qualities or ``codes'' of the track but gets closer to the 
topical focus and  positionality art rap affords itself with respect to the ``mainstream.''

\section{Hip-Hop Practice and Listening as Meditation} \label{listeningasmeditation}

In the previous section, I argue that ``Unapologetic'' is characterized not by one style such as
nerdcore or jazz rap but rather by a musical practice. This practice is one of listening: in 
particular, a type of sonic engagement I will term \emph{listening as meditation}. Historically,
this listening practice is abundant in all stages of the making of hip-hop\textemdash from its 
conception in the mind of the producer and emcee, to its performance live or in studio, and 
eventually its reception by the listener. In this section, I show that practitioners throughout
hip-hop's history and Eagle himself promote this style of listening, one that engenders a deep 
immersion in hip-hop as a  musical form.

The phrase listening as meditation serves as an inversion of the mind-body dualism invoked by 
earlier popular music discourses. To be sure, by invoking ``listening'' as a unifying thread 
amongst a wide variety of stylistic manifestations, I am inviting a tacit definition of the 
underground as ``hip-hop to be \emph{listened} to'' as distinct from ``to be danced to'' or 
``merely heard.''\footnote{
    My focus on listening runs the risk of positioning underground hip-hop as a ``transmission of mind--mind messages'' 
    between composer and listener (see~\autocite[20]{suzanneg.cusickFeministTheoryMusic1994}.) While this style of
    inquiry still characterizes a great deal of music-theoretical scholarship, my reservations arise from the potential
    for erasing the embodied experience of performers (who tend to be the composers in my repertory as well) and the
    listeners within the exchange that happens in underground hip-hop. I therefore intend with my term for this mode
    of listening (``as meditation'') to be interpreted as a unified mind-body practice.}
Part of the trouble with such a definition is potential its use as a justification  for excluding certain
musics from being objects of critical inquiry. Moreover, my choice to prioritize discussing 
this music as an engagement of one's musical intellect does not exclude its use as ``dance'' 
music, nor as a music that does not engender embodied responses. The metaphor of listening 
as meditation, then, proposes a unification of cognitive and embodied responses that characterize
engagement with underground hip-hop. In short, it is an argument that will, in the coming pages,
put this chapter's epigraph to the test (that ``Listening correctly can cause introspecting.'')

At each point in the process of making hip-hop, and indeed throughout different points in
its history as a music form, several practitioners have evinced that their listening exhibits
a meditative engagement with music. The rapper-producer Count Bass D suggests that this style
of listening guides his process for sample selection and composition particularly for the
introduction to his 2002 track ``Truth to Light'': 
    \begin{quote}
        Songs that really I like a whole lot, that I've liked over the years, kind of run through my head all
        the time and so they kind of creep into songs. \textellipsis Unless you know [Nice \& Smooth's ``Funky
        for You''], you don't know who I'm talking about or what I'm talking about, but to people who are in
        the know, it strengthens their faith that the things I'm talking about that they don't understand may
        have some relevance to them in time.\footnote{\autocite[100]{mickeyhessHipHopDead2007}.}
    \end{quote}
Count particularly focuses on familiarity, being ``in the know'' with the source material, as
a requirement for parsing his dialogue with the Nice \& Smooth at the beginning of the track, but
his discussion also reveals listening norms within the hip-hop community. First, Count as a producer
engaged with ``Funky for You'' regularly enough over a long period of time that the decision to
dialogue with the artists ad-libs on the original recording felt subliminal and ``[crept] into''
the track. Second, he expects that his audience will have engaged with the material on a similar
level, enough to ``know what [he's] talking about.'' Finally, and perhaps most interestingly,
Count conceives of a \emph{type} of listener, who may not know the source material but as a
result of their familiarity with the listening norms in the genre trusts will come to them
``in time'' as they listen further and more broadly beyond ``Truth to Light.'' Such a
listener is the \emph{meditative} listener and is the type of listener Count considers
his primary audience.

The meditative listening process for the emcee is similar to the producer's in that the emcee
listens with enough regularity and focus that the imitation of style can become subconscious,
to the point where it can prove problematic for their on writing process. Open Mike explains
that because of his listening interactions with the emcee MF DOOM's flow, ``he has to be 
careful with [DOOM's] flow, because [he] can almost get into [DOOM's] mind in terms of how he 
writes.''\footnote{\cite{estellecaswellRappingDeconstructedBest2016}.} Open Mike characterizes
his depth of engagement with DOOM as something that could turn into a problem because it would
be easy for him to be write verse in the same exact manner that DOOM does and thus turn into
something more derivative than he would like. Again the practitioner positions their compositional
choices as a subconscious choice based on deep engagement with models for composing that show 
up in their listening. For Open Mike as an emcee, this listening has occurred with enough
frequency, depth, and regularity that he subsumed DOOM's style of flow into his own and needs
to consciously choose to offset his writing from this practice at times.

Finally, hip-hop practitioners show that listening as meditation informs their interpretation of
performance by other practitioners; they envision the artists they interact with as meditative
listeners, who then serve as guides for further meditative listening. The Bronx DJ Grand Wizzard
Theodore describes, when hearing a record through DJ Kool Herc's soundsystem, ``[i]t made you
listen to a record and made you appreciate the record even more. He would play a record that 
you listened to every day and you would be like `Wow, that record has \emph{bells} in it?' It's 
like you heard instruments in the record that you never thought the record even had.''\footnote{
\autocite[139]{christabronGlassHiphopProduction2015}.} Theodore's description shows that even
within the context of a block party, the type of listening hip-hop engenders brings about new
depths to the music beyond the standard engagement.

\section{Definitions and Limitations of Underground}

In the above sections, I have examined stylistic and thematic traits in two musicological 
constructions of subgenre\textemdash nerdcore and jazz rap\textemdash considering Open Mike
Eagle's ``Unapologetic'' as an example that does not neatly fit in either category as previous
scholars have defined them. Additionally, I considered how Eagle participates in a historical
practice shared by a number of hip-hop musicians: a practice where musicians are listeners first,
listeners who engage with the music at a meditative level. In this section, I argue that the 
designation \emph{underground} in hip-hop is defined by meditative listening. Fans of underground
hip-hop are encouraged to engage with the subgenre as meditative listeners because they see 
underground artists perform meditative listening. Then, as a result of the perceived closeness to
the artist that characterizes the underground, listeners choose to engage with the music in a 
similarly meditative fashion.

Before I arrive at this definition, I need to address three conceptions of the underground which I
believe to be faulty and therefore do not wish to import in my definition. In particular, a definition
of underground cannot meaningfully be constructed from discourses of artistry, authenticity, or 
commerciality. These three metrics permeate popular discussions of what makes music ``underground,'' 
so addressing them out front will prove useful in elucidating what I wish to focus on: namely, 
listening practice and methods of constructing the music.

Defining underground hip-hop as an ``art'' genre invokes discourses around what makes a piece of
music beautiful, a metric that is at once too broad to define and too based in subjective experience
to communicate effectively. Although rappers like Eagle have constructed their personas in this way,
and in fact fans buy into such constructions, I find it telling that such discourses are often
short-lived. As early as 2013, Eagle raps:

\settowidth{\versewidth}{I used to throw these sensitive parties for art rap,}
    \begin{verse}[\versewidth]
        I used to throw these sensitive parties for art rap, \\
        No regrets, but I was foolish to start that. \\
        Sophisticated fuckers left a bitch with a bar tab, \\
        And now we just throw pizza parties.\footnote{
        \cite{milo2013}.}
    \end{verse}
Eagle's indictment of the ``art rap'' scene came relatively quickly on the heels of his construction of
identity as an art-rapper, in a feature only a few projects later than he unapologetically declared
it. Though his lyrics focus on the pretension and bourgeois tastes of the fans it garnered him, I also 
wonder if writing ``artful'' rap lost its meaning. As a listener, I find it hard to describe whether
something reaches me artful; for instance, I do not, as Justin Williams suggests, hear a piece based around
a jazz sample as more ``artistic'' than, say, something constructed from DAW-based synthesizers simply
because of the importance jazz is given in contemporary American culture.\footnote{
    For my discussion on Williams' argument that jazz rap reaches listeners as a ``high art,'' see
    section~\ref{nerdcorejazzrapgoldenage}.} 
Rap music's artistry seems less quantifiable (and more personally defined) than its ``undergroundness.''

Invoking authenticity to define underground also proves to be a slippery metric, in perhaps a similar way
to artistry. This definition finds purchase in popular discourse as well because of artists who tie it
closely to the early, pre-recorded era of hip-hop. Fab 5 Freddy summarized this mentality when discussing 
his role as the writer for the 1982 film \textit{Wild Style}. The writers set the story before the 1979
release of Sugarhill Gang's single ``Rapper's Delight'' because ``[they] wanted to go back to a few years 
earlier\textellipsis \emph{when hip-hop was completely underground, when the form was raw and pure.}''\footnote{
    \autocite[23]{justinawilliamsRhyminStealinMusical2013}. Emphasis my own. The release
    of ``Rapper's Delight'' is often thought of as the moment when hip-hop went commercial.}
Fab 5 Freddy's association of ``undergroundness'' with ``rawness'' or ``purity'' illustrates one type
of a way that listeners, critics, and fans alike make sense of hip-hop, but once again, I wonder if the
term has any bearing on the sound of underground hip-hop. To me, discourses of authenticity seem to be
based on subjective, evaluative judgements a listener makes.

Finally, a definition of underground based in commerciality may seem valuable at a surface level, but modern
methods for discovering and participating in cultures of listening problematize such a conception. Commerciality
may very well be the metric for judging the authenticity of a rap song: an artists' commercial success can be tied to
whether or not a listener receives them as underground. However, commercial success is not the sole quality that
constructs ``undergroundness.'' As Anthony Kwame Harrison argues in his ethnographic foray into the San Francisco
underground hip-hop scene, ``the label `underground' can be applied to everything from Grammy-nominated artists like
Common and the Roots to groups like The Latter, who once boasted(!) of having sold only two copies of one of their
CDs.''\footnote{
    \autocite[9]{anthonykwameharrisonHipHopUnderground2009}.}
His quotation about the designation from Kegs One, the artist and proprietor of the hip-hop shop Below the Surface, 
elucidates how gradations of commerciality manifest within underground hip-hop:

    \begin{quote}
        I have a few different people that are into the more `commercial underground' stuff, but then 95 percent of my
        customers are here for the  literal four-track tapes. You know, dirty-sounding, low-budget, in-the-room, in-the-closet
        recorded tapes.\footnote{
        Quoted in \autocite[10]{anthonykwameharrisonHipHopUnderground2009}.}
    \end{quote}
Harrison assesses a variety of commercial levels in hip-hop, all the way from The Latter's two records sold, to 
the Bay-area artists who are about to touch the surface underground, on up musicians like The Roots, who have been 
Jimmy Fallon's in-house band since 2009. For acts to have varying degrees of success and still be considered 
underground points to some metric beyond commerciality being invoked to define the quality of ``undergroundness.''

The definition Harrison constructs is based on an ethos he observes in the Bay-area underground, and his observations
support serve as a basis for my own definition based in listening practices. In particular, he notes ``the idea of a
blurring, thinning, almost imperceptible line that separates artists and fans'' and ``an intimacy of the fan bases
[underground artists] tend to attract.''\footnote{
    \autocite[10--11]{anthonykwameharrisonHipHopUnderground2009}.}
Although, like Harrison, I believe this ethos decreases in palpability as an artists commercial success and notoriety
increases, these artists are still stylistic participants in underground hip-hop because this ethos can be perceived
in the music. In particular, I argue, this intimacy can be perceived in the way an artists models meditative listening
for fans; in effect, they see the artist acting as a listener\textemdash as a fan themselves\textemdash and experience
a type of intimacy in that recognition.

This perception of intimacy and the acts of meditative listening it engenders are core to understanding how a listener
experiences underground hip-hop. Underground is not a style of hip-hop codified by one sonic practice, nor by one thematic
focus, but rather by a multiplicity of both, guided by the ways in which the musicians demonstrate their relation to a
network of practices and focuses. Underground hip-hop artists that bring together a multiplicity of experiences in their
work model a listening practice for the fan, who may at first intentionally re-navigate these listening networks for themselves
or may in time, as Chuck D suggests, come to trust that these networks exist and will come to them through their own paths of
musical experience.


\section{Outline of the Thesis}

Working from the definition of underground as a practice of meditative listening shared between fans and artists 
participating in the subgenre, this thesis focuses on transcribing and analyzing a repertory of underground rap songs, 
attempting to uncover how the quality of ``undergroundness'' manifests within beats and flows. Even though ``underground''
cannot be clarify how  the music is received as underground, especially in instances where ``undergroundness'' is less 
textually explicit than in Eagle's ``Unapologetic.'' In short, I am interested in how musical text in my repertory 
``sound'' the underground.

My approach to reading musical texts agentially is bound up in Loren Kajikawa's notion of ``sounding'' in rap music. 
Particularly, Kajikawa argues that the rap song is a medium for transmitting identities and meaning between the artist
and the listener.\footnote{
    \autocite[2]{lorenkajikawaSoundingRaceRap2015}.}
Where his scholarship focuses on the encoding of racial and gender identities, mine will focus on a subsequent matter:
how an artist navigates mainstream expectations about these identities, and in so doing asks you to listen in a new, more
meditative way as a result of their navigation.

The two primary chapters in this project will interrogate the overlapping yet distinct methods by which hip-hop
musicians sound the underground. My first chapter focuses on beatmaking, arguing that producers create underground
hip-hop beats by avoiding fixed loops and regular sample lengths.\footnote{
    Justin A. Williams argues that all of rap's subgenres deconstruct this notion (see \cite{justinawilliamsBeatsFlowsResponse2009}).
    Williams' critique of the musical-theoretical focus on repetition in hip-hop is foundational to my approach; however, my work
    in this chapter deals more with underground producers' deliberate choice to disrupt repetition beyond layering elements in and
    out of the music.} 
In particular, I examine how producers introduce variety primarily through digital editing techniques that mimic 
the live, improvisatory roots of the genre within a broadly repetitive musical texture. I employ two distinct methods of
transcription based on other scholars methods for beat transcription\textemdash namely, Kajikawa's breakbeat 
transcriptions\footnote{
    \textit{Cf.} \autocite[29--30 and 36--37]{lorenkajikawaSoundingRaceRap2015}, 29-30  and 36-37.} 
and Williams' basic beat transcriptions.\footnote{
    \textit{Cf.} \autocite[61ff]{justinawilliamsRhyminStealinMusical2013}.} 
From my transcriptions, I work out four terms to methods producers use to affect variety in the musical texture: 
resampling/recomposing, choking, glitching, and slipping. While this list of methods does not exhaust the ways in which
producers sound the underground, it demonstrates the hip-hop beat as a space for co-creation of variety with the rapper
and thus alternative identity within hip-hop.

Building off this notion, my second chapter the comparable role underground emcees play in composing their verses in rap
songs. Although lyrics offer the clearest method for transmitting identity, this chapter focuses on emcees' non-textual 
methods of ``[amplifying] and in some cases [transforming] the information that listeners receive'' through texted and visual
communication.\footnote{
    \autocite[12]{lorenkajikawaSoundingRaceRap2015}.} 
As in my previous chapter, I employ a few methods of transcription to examine different musical elements. When working primarily
the way text creates metrical patterns, rhyme schemes, and linguistic-syntactic structures, I use Mitchell Ohriner's system for
visualizing rap flows, which is intentionally abstracted from Western musical notation.\footnote{
    For a sample and detailed overview of Ohriner's notational method, see \autocite[xxvii--xl]{mitchellohrinerFlowRhythmicVoice2019}.}
When dealing with an emcees' approach to performance rather than structure, I opt for staff notation, mindful of the its limited
ability to denote the complexities of a vocal rap performance. 
    

\chapter{Beat Construction and Methods of Alteration in the Underground}
\section{Variety and Reptetition}
In response to Kyle Adams' seminal \textit{Music Theory Online} article ``Aspects of the Music/Text Relationship in Rap,'' Justin Williams litigated an important phenomenon in rap music production that changed alongside the move from the turntable to the recording studio. He writes:

\begin{quote}
    \small ``In terms of rap music recordings, the idea of a completely fixed loop is largely fictitious. There may be a set of layers which we could term the `basic beat' which repeats intact for certain durations of time, but one would be hard-pressed to find an entire musical complement that stays the same throughout. Rap music’s layers will more often than not fluctuate throughout a given song, with sonic additions and subtractions, manipulations of digital samples, and even sharp changes in aspects of the basic beat.''\footnote{\cite{justinawilliamsBeatsFlowsResponse2009}. It is worth noting that Adams does not necessarily dispute the occurrence of development within an accompaniment as a phenomenon, but he does maintain that the unchanging elements of the beat function as ``primary accompanimental layers'' (See \cite{kyleadamsPeopleInstinctiveAssumptions2009}.)}
\end{quote}

\noindent \normalsize As both a fan of and critical listener to rap music, I experience repetition and variation significantly within musical lexicon of the rap instrumental. Adams notes that even as the locus of rap music-making moved into the studio, producers remained adherent to the break-beat based origins of the genre, and so structural elements of the hip-hop beat tend to repeat within a four to eight bar, often simple quadruple metrical space. At the same time, Williams' observations point towards producers' observable preference for variation within repetition, made feasible by newer technologies for sampling, manipulating, and composing with pre-recorded digital materials.

The answer to whether the hip-hop beat is primarily repetitive or primarily developmental is indeterminate because both dimensions exert influence over the process of creating it. Instead, I want to interrogate how producers use these dimensions to aesthetic or rhetorical ends: to invoke Loren Kajikawa's notion of sounded phenomena in hip-hop, how ``rap artists produce (and listeners interpret) musical meanings at the level of the song.''\footnote{\cite{lorenkajikawaSoundingRaceRap2015}, 2.} While Kajikawa's focus on racial identity unpacks the most important element of a producer's  curatorial work, another conscious choice a producer makes is to identify with or against the hip-hop mainstream. Producers use the beat as a means of coding themes and identities complementary to the rapper who declares allegiance to the underground within their flow. Such a reading, I believe, resonates with the central argument of Kajikawa's \textit{Sounding Race} because ``rap has cultivated a mainstream audience\textellipsis by promoting highly visible (and often controversial) representations of black masculine identity''\footnote{\cite{lorenkajikawaSoundingRaceRap2015}, 5.}.

As with any sub-generic distinction predicated on narratives of authenticity, I use the term \emph{underground} trepidatiously, knowing full well it has as many meanings as it has users. What I call the underground signifies a space where deviation from the mainstream is expected. This is not a value judgement, nor an all-encompassing definition of either mainstream or underground. My definition is necessarily vague because these are elusive terms, but they hold importance because they represent \textit{de facto} ``imagined communities'' with which rappers, producers, and listeners identify.\footnote{The concept of the imagined community of hip-hop has been broadly traced by Williams via Joseph G. Schloss: the term comes from Benedict Anderson's conceptualization of how a community 
Williams argues for the importance of recognizing hip-hop as an imagined community, a term coined by Benedict Anderson for conceptualizing nationalism, because such a community must exist to provide a cohesive set of strategies by which producers and listeners interpret musical objects (see \cite{justinawilliamsRhyminStealinMusical2013}: 13-19.) 

Joseph G. Schloss also develops the notion of an imagined hip-hop community to categorize the network of hip-hop producers who serve as his ethnographic subjects, and to explain their conception of the audience for their music (see \cite{josephgschlossMakingBeatsArt2004}: 4-5.)}

In this chapter, I argue that producers work with rappers to sound the hip-hop underground by using variety within their beat architecture, punctuating alternative identity in sampling; I base this argument in transcriptions that illustrate developmental and repeating elements within the musical texture. My case studies show that underground producers tend to deviate from expectations concerning form in their sectional divisions of the beat. Highlighting sample-based and non-sample-based approaches to beat-making, I trace methods of sample and loop manipulation such as choking, glitching, and slippage as producers' means of variation. Lastly, I contend that underground hip-hop is sounded by an overarching aesthetic of disunity, linking the tradition to Olly Wilson's heterogeneous sound ideal of African-American music. I briefly conclude by discussing the importance of the rapped text as the most variable element within the texture and consider its importance in future projects I will undertake concerning the hip-hop underground.

\section{Methods of Transcription and Analysis}
I use transcription in this chapter\textemdash indeed, overall in this project\textemdash despite knowing that it introduces a level of abstraction from both the musical practice and perceptual experience of my hip-hop repertory. Scholars such as Joseph G. Schloss have meaningfully analyzed hip-hop production while eschewing transcription altogether on ethical and aesthetic grounds.\footnote{\cite{josephgschlossMakingBeatsArt2004}, 13-15.} Others, like Kajikawa and Adam Krims, have employed methods of transcription that move away from standard notation in Western Classical tradition.\footnote{\textit{Cf.} \cite{lorenkajikawaSoundingRaceRap2015}, 29-30 and 36-37; \cite{adamkrimsRapMusicPoetics2000}: 105-110.} Still others, including Adams and Robert Komaniecki, rely primarily on standard notation in order present their arguments within traditional spheres of music-theoretical discourse.\footnote{\textit{Cf. }\cite{kyleadamsMetricalTechniquesFlow2009}; \cite{robertkomanieckiAnalyzingCollaborativeFlow2017}.} Each approach holds its own merit, and each privileges a different audience: the creator, the listener, and the academic.

Although I understand Schloss and others' reticence to transcribe rap music, my choice to do grows out of the style of deep listening requisite to creating a transcription. Using transcription, I do not aim to apply ``the tools of notation and analysis developed for the study of Western Classical music\textellipsis uncritically to rap music.''\footnote{\cite{lorenkajikawaSoundingRaceRap2015}, 12.}. Instead, I offer them as a subjective realization of my ``living inside'' the musical object for a time.\footnote{\cite{peterwinklerWritingGhostNotes1997}: 200.} 

The two styles of transcription I employ in this chapter represent the tendency to create variety within repetition in distinct ways. First, I use a tabular “roadmap” style of transcription that overviews musical form. The roadmaps call to attention that which how producers might sound formal divisions, in addition marking the alterations the producer introduces in each second. Each roadmap also has a more expanded version that details the function and relationship of each instrumental layer, all of which can be found in the appendix beginning on p.~\pageref{appendix:fullroadmaps}. 

Second, I use standard notation to create a musical ``snapshot'' of the beat at distinct points in the musical texture. These ``snapshots'' allow me to discuss the function of particular musical layers, in addition to showing how repetitive textures might shift over the course of the beat. I use staff notation not as a reflection of producers' methods for visualizing the beat, but because staff notation is more widely legible in academic settings than tabulature and other methods of transcription.

Each style of transcription also allows me to draw analytical conclusions about compositional tendencies within the underground. Based on my roadmaps, I note a tendency to play with the expectations for form in mainsteam hip-hop. Ben Duinker, argues that what he calls \emph{verse-hook} form has increasingly become the preferred form of the genre in. accordance with hip-hop's assimilation into mainstream popular culture.\footnote{\cite{benduinkerSongFormMainstreaming2020}, 94. Duinker defines this form as the sixteen-measure verses and more tighlty-knit hooks, framed by short instrumental intro and outro sections.} Based on my repertory, I argue that underground producers tend to use the same section types Duinker identifies\textemdash \textit{NAME THEM HERE}\textemdash but play with the function of these section through sequencing them in ways that comment on or contrast the emcee's vocal delivery.

Based on my snapshots, I also argue that underground producers treat variety within formal sections as aesthetically preferential to repetition and introduce it primarily through means beyond layering. Layering, as Duinker notes, is a technique hip-hop producers use wherein loops of two, four, or eight measures are added or subtracted to a texture gradually.\footnote{\cite{benduinkerSongFormMainstreaming2020}, 96.} In addition to use of layering, underground producers favor drastic changes even within formal sections. Specifically, they tend to use methods that introduce these changes in ways that mimic the live, improvisatory roots of the genre found in DJing.

I apply four terms to these methods of alteration within this chapter: recomposing (or resampiling), choking, glitching, and slipping.
\clearpage

\singlespacing
\section{Sample-Based Case Studies}
My first two case studies are MF DOOM's 2003 ``One Beer, '' produced by Madlib, and Kendrick Lamar's 2011 ``Rigamortis,'' produced by Willie B. These are both sample-based on account of their construction around one ``lead'' sample. ``One Beer'' samples primarily from Cortex's 1975 ``Huit Octobre 1971,'' and Willie B samples Willie Jones III's 2010 ``The Thorn.'' Both producers introduce variety to their limited compositional palettes through techniques of sample manipulation and layering while also circumventing formal expectations for mainstream hip-hop.

\subsection*{\centering MF DOOM's ``One Beer''}

    \begin{table}[ht]
        \centering
            \begin{tabular}{|c|c|c|c|l|}
                 \hline
                  Section & Timecode & Duration & Sample & Note \\ \hline
                  Intro & 0:00 & 8 bars & ``Huit Octobre'' I & Choked in Bar 8  \\ \hline
                  Verse 1A & 0:20 & 16 Bars & ``Huit Octobre'' II &  \\ \hline
                  Verse 1B & 1:02 & 16 Bars & ``Huit Octobre'' II & Choked in Bar 0.4 \\ \hline
                  \sout{Hook} & 1:43 & 8 Bars & ``Huit Octobre'' I & Choked in Bar 8.3-4 \\ \hline
                  Verse 2B & 2:03 & 8 Bars & ``Huit Octobre'' II & Choked in Bar 0.4, 8.4 \\ \hline
                  \sout{Verse 2C} & 2:24 & 16 Bars & ``Huit Octobre'' II & Choked in Bar 0.4 \\ \hline
                  Skit & 3:06 & 26 Bars & \textit{Spider-Man} & Resampling, drum improv \\ \hline
             \end{tabular}
        \caption{Condensed Roadmap to MF DOOM and Madlib's ``One Beer.''}
        \label{tab:onebeer}
    \end{table}

\normalsize Outlined in Table~\ref{tab:onebeer}, Madlib constructs his beat from discrete sections of ``Huit Octobre'' that contrast in groove and harmony, providing sectional clarity beneath DOOM's through-composed verses. The first section features a synth and bass in octaves on the same melody accompanied by drums emphasizing upbeats on the hi-hat. The second section's straight feel juxtaposes the first's triplet-eighth swing. Pitched up a semitone to better match the harmony of the first, it features a four-chord loop with a prominent vocal countermelody. The basic beats formed by these two samples are rendered in Figures~\ref{fig:onebeerintro}~and~\ref{fig:onebeermain}.

    \begin{figure}[ht]
        \centering
        \includegraphics[width=\textwidth]{images/figures/chp 02/000019onebeerintro.pdf}
        \caption{Snapshot of Madlib's first sample in ``One Beer,'' 0:00-0:19.}
        \label{fig:onebeerintro}
    \end{figure}

Two dimensions of each section elucidate Madlib's intended contrast. Harmonically, the shift in samples brings about different function. The first sample features a single, \emph{repetitive} harmony that arpeggiates a global I$^{b7}$.\footnote{\cite{kyleadamsHarmonicSyntacticMotivic2020}. The three categories Adams provides for harmony in hip-hop are repetitive, oscillating, and expansional, all of which I touch on throughout the course of this paper.} Harmonic stasis gives way to \emph{oscillating} F-sharp and G extended tertian harmony in the second sample, which functions as an activation for a sixteen-bar would-be verse unit before arriving returning to stasis for an eight-bar would-be hook.

    \begin{figure}[ht]
        \centering
        \includegraphics[width=\textwidth]{images/figures/chp 02/02031onebeermain.pdf}
        \caption{Snapshot of Madlib's first second sample in ``One Beer,'' 0:20-0:31.}
        \label{fig:onebeermain}
    \end{figure}

In conjunction with harmony, the timbre, dynamics, and rhythmic stress of the drum patterns in each sample reflect the contrasting functions of each section the beat projects. Schloss notes that producers often choose their samples based on the aesthetic delight they experience concerning timbre, and that drum sounds are sought after with preference.\footnote{\cite{josephgschlossMakingBeatsArt2004}, 141-42.} In addition to the timbre of the drums in ``Huit Octobre,'' Madlib's two sections contrast in rhythmic content and function. When the track arrives at the first sample, the relative stasis is reinforced by the accompanimental kit pattern, focusing the listener on the synth line. In the second section, the drums activate along with the harmony using distinctive fills and striking timbres throughout the four-bar loop. This culminates in the prominence of the two-beat long triplet eighth fill in m. 2 of the loop.

Madlib's samples also communicate information about the theme and message of ``One Beer.'' Combined with DOOM's \emph{allosonic}\footnote{\cite{justinawilliamsRhyminStealinMusical2013}, 3. Whereas most hip-hop sampling would be an \emph{autosonic} duplication, DOOM intones Porter's words in a parodical \textit{sprechstimme} that sounds as if it were sung by a drunk fool getting a kick out of brew.} interpolation of Cole Porter's ``I Get a Kick Out of You'' and dialogue from the 1981 episode of \textit{Spider-man}, ``Dr. Doom, Master of the World,'' Madlib draws from a ``rich assortment of multimedia borrowings, references, and parodies that operate in hip-hop music as a whole'' all within the course of one song.\footnote{\cite{joannademersSampling1970sHipHop2003}: 42.} The result of this is two-fold: (1) the piece coalesces as a work of timbral and stylistic heterogeneity,\footnote{\cite{ollywilsonHeterogeneousSoundIdeal1992}: 329. Timbral stratification is core to Wilson's heterogeneous sound ideal.} and the thematic focus of the track (DOOM, rap's supervillian, joking around about alcohol), is reinforced by the sound sources of the beat.

\subsection*{\centering Kendrick Lamar's ``Rigamortis''}

    \begin{table}[ht]
        \centering
            \begin{tabular}{|c|c|c|c|l|}
                \hline
                Section  & Timecode & Duration & Sample        & Note \\ \hline
                Intro    & 0:00     & 4 bars   & ``The Thorn'' & \\ \hline
                Hook     & 0:10     & 6 bars   & ``The Thorn'' & Full sample plays \\ \hline
                Verse 1A & 0:27     & 12 bars  & ``The Thorn'' & \\ \hline
                Verse 1B & 0:59     & 10 bars  & ``The Thorn'' & Improvisatory sample choking \\ \hline
                Hook     & 1:26     & 6 bars   & ``The Thorn'' & Lead sample slips backwards \\ \hline
                Verse 2A & 1:43     & 6 bars   & ``The Thorn'' & \\ \hline
                Verse 2B & 2:04     & 8 bars   & ``The Thorn'' & Alternating sample choking \\ \hline
                Hook     & 2:31     & 6 bars   & ``The Thorn'' & Improvisatory sample choking\\ \hline
            \end{tabular}
        \caption{Condensed roadmap to Kendrick Lamar and Willie B's ``Rigamortis.''}
        \label{tab:rigamortis}
    \end{table}

The section of music Willie B samples from ``The Thorn'' features Jones' combo regrouping a simple quadruple meter into 3+3+2 beat divisions, underscoring the saxophone melody. Figure~\ref{fig:2.1} shows the sampled layers in its top three staves composited with Willie B's added layers in the staves below. To combine these elements, he filters out the low end, pitches up the sample by nine semitones, and adds a drum loop, sub bass, synth, and other post-production effects. ``Rigamortis'' also begins with a version of the sample that repeats only its first two measures before layering in the other textures.

    \begin{figure}[ht]
        \centering
        \includegraphics[width=\textwidth]{images/figures/chp 02/013020thethornfull.pdf}
        \caption{Snapshot of the sampled portion of ``The Thorn,'' 0:13-0:20.}
        \label{fig:2.1}
    \end{figure}

The beat's repetitive harmony and form help ground the listener within Lamar's complex vocal delivery. Although the return of the ``He dead!'' section functions like a hook, Lamar plays on listener's expectations for this call-and-response to return intact; instead, its first instance adds lyrics, it returns early as a fragmented interruption of the first verse, and it transitions seamlessly into the second verse via shared rhyme scheme. Because Lamar's verse has the ``tendency to fill up all the musical space'' within the mix,  the beat is relegated to a more simplistic, repetitive role.\footnote{\cite{ollywilsonHeterogeneousSoundIdeal1992}, 328.}

The grounded, repetitive simplicity of the beat does not mean the beat is unchanging. Willie B makes frequent use of sample choking, evening muting the sample completely for parts of Lamar's final verse. He also obfuscates the texture by employing what I call \emph{sample slippage}, wherein the micro-rhythmic space between the lead sample and drum loop ebbs and flows due to slightly varied loop lengths and expressively delayed re-triggering.\footnote{The phenomenon of sample slippage dovetails with Anne Danielsen's work on the Beat Bin and rhythmic tolerance (see \cite{annedanielsenHereThereEverywhere2016}: 29\textit{ff.})} The affect of such shifting creates a listening experience akin to phasing, as samples (albeit sonically discrete ones) move in and out of sync with each other.  While Figure~\ref{fig:2.1} and Table~\ref{tab:rigamortis} show all of the sample loops beginning and ending in alignment, in reality, the beginnings and ends of loops are messy, and each float in and out of time with each other throughout the track.

This devotion to messiness manifests heterogeneity within the limitations of a four-measure loop. The concepts of downbeat, meter, and formal structure are all fraught within ``Rigamortis,'' in a manner that echoes the technological limitations of rap's advent as a genre. Because of this messiness, ``Rigamortis'' sounds how one might expect a mixtape from Compton to sound, reinforcing Lamar's identity with the underground and his roots.

%\subsection*{\centering Milo and Self Jupiter's ``Ornette's Swan Song''}

\section{Live-Tracked Case Studies}
The final two case studies investage Milo's 2015 `Rabblerouse,'' produced by Kenny Segal, and Noname's 2018 ``Blaxploitation,'' produced by Phoelix. Neither Segal nor Phoelix use a single lead sample to construct their beat; instead, they live-track their loops with instruments at their disposal. Both producers remain conceptually adherent to the basic beat, and both introduce variety using similar techniques for manipulation to sample-based approaches. Each producer varies the form of their beat against a conceptually mainstream structure, using vocal samples and musical styles to communicate alterity.

\subsection*{\centering Milo's ``Rabblerouse''}

On ``Rabblerouse,'' Milo delivers a single 24-bar verse that operates as an introductory song fragment to the concept album it resides upon. Segal constructs an unstable beat, propelling the listener toward a resolution that arrives with the second track, ``Souvenir.'' The beat ends with a lone vocal sample of the character Yoshimitsu from \textit{Soul Caliber 2} to connect the two tracks. The sample also connects thematically to the penultimate track ``Napping Under the Echo Tree,'' when Milo refers to himself as the ``Yoshimitsu of Boyle Heights.''

    \begin{figure}[ht]
        \centering
        \includegraphics[width=\textwidth]{images/figures/chp 02/012023rabblefirstglitch.pdf}
        \caption{Snapshot of the first drum glitch in ``Rabblerouse,'' 0:12-0:23.}
        \label{fig:rabblefirstglitch}
    \end{figure}

Segal manifests instability in the beat for ``Rabblerouse'' on several structural levels. First, the loop repeats an irregular six-bar chord progression on a Fender Rhodes. The pattern, though technically \emph{expansional}, sounds unresolved.\footnote{\cite{kyleadamsHarmonicSyntacticMotivic2020}. Adams does not make it a requisite condition of the \emph{expansional} harmonic category to function as a complete phrase, although he notes that it commonly will.} Figure~\ref{fig:rabblefirstglitch} indicates my harmonic reading: the chords function in E Dorian, but the progression leaves out a resolution to F-sharp minor that would close the loop.

``Rabblerouse'' also feels unstable because Milo's verse enters after four bars, and though this is conventional, the six-bar structure of the basic beat offsets the meters being projected by rapper and producer. Segal acccounts for this metric dissonance by repeating the loop six times, shortening the penultimate repetition to four bars to realign the verse's end with the next downbeat. Thereafter, he employs a \emph{sample glitch} where the downbeat is re-triggered four times in a row, extending the E$^{sus4}$ harmony. As Table~\ref{tab:rabblerouse} illustrates, the unresolved sonority heard from the beginning dissipates as the Yoshimitsu sample is triggered.


\begin{table}[ht]
    \centering
        \begin{tabular}{|c|c|c|c|l|}
             \hline
            Section & Timecode & Duration & Sample                  & Note \\ \hline
            Intro   & 0:00     & 4 Bars   &                         & Rhodes extends over boundary \\ \hline
            Verse   & 0:08     & 24 Bars  &                         & Rhodes slips backwards \\ \hline
                    & 0:14     &          &                         & Drum glitch shifts downbeat \\ \hline
                    & 0:39     &          &                         & Drum glitch and bass recomp. \\ \hline
            Outro   & 0:56     & 4 Bars   &                         & Glich, recomp., and choking \\ \hline
                    & 1:03     &          & \textit{Soul Caliber} 2 & \\ \hline
        \end{tabular}
    \caption{Condensed roadmap to Milo and Kenny Segal's ``Rabblerouse''}
    \label{tab:rabblerouse}
\end{table}

\normalsize Segal's fragmentary aesthetic on ``Rabblerouse'' sounds as a mode of alterity because it is uncommon for a hip-hop beat not to function as a closed loop. The metric dissonance projected from the beginning does not resolve by the track's end, necessarily drawing a listener in to the text function and pointing towards the remaining tracks on the album. This beat is unstable alone, but functions cohesively with the sonic palette of Milo's LP \textit{So The Flies Don't Come} as a whole. This technique is predicated upon an interest in lyricism and conceptual unity propagated within the underground hip-hop scene.

\subsection*{\centering Noname's ``Blaxploitation''}

\begin{figure}[ht]
    \centering
    \includegraphics[width=\textwidth]{images/figures/chp 02/006016blaxintro.pdf}
    \caption{Snapshot of the intro to``Blaxploitation,'' 0:06-0:16.}
    \label{fig:blaxploitationintro}
\end{figure}

Phoelix's production on ``Blaxploitation'' centers around an angular, funk bassline that forms a ``functional circuit'' in C-sharp minor.\footnote{\cite{kyleadamsHarmonicSyntacticMotivic2020}.} The drums and bass sound heterogeneously in spite of the relative sparsity of the orchestration due to the frequency of attacks within a short time frame.\footnote{One dimension of the heterogeneous sound ideal is a ``high density of musical events within a relatively short musical time frame''(See \cite{ollywilsonHeterogeneousSoundIdeal1992}, 329.)} Figure~\ref{fig:blaxploitationintro} shows the three primary elements of the basic beat, as well as the sympathetic accentual patterns between the drums and bass.

\begin{table}[ht]
    \centering
    \begin{tabular}{|c|c|c|c|l|}
         \hline
        Section     & Timecode & Duration    & Sample               & Note \\ \hline
        Intro       & 0:00     &             & \textit{Dolemite} I  & \\ \hline
                    & 0:08     & 4 Bars      &                      & \\ \hline
        Verse I     & 0:18     & 12 Bars     &                      & \\ \hline
        \sout{Hook} & 0:46     & 8 Bars      & \textit{Dolemite} II & BGV recomposition \\ \hline
        Verse II    & 1:05     & 12 Bars     &                      & Organ recomposition \\ \hline
        \sout{Hook} & 1:33     & 8 Bars      & \textit{TSWSBTD}     & \\ \hline
        Outro       & 1:52     & 4 Bars      &                      & \\ \hline
    \end{tabular}
    \caption{Condensed roadmap to Noname and Phoelix's ``Blaxploitation.''}
    \label{tab:blaxploitation}
\end{table}

As the track's title suggests, the beat of ``Blaxploitation'' is steeped in the sound of the 1970s, grooving in an allosonic nod to the Motown-inspired soundtracks of the film genre. As shown in Table~\ref{tab:blaxploitation}, the track also autosonically samples dialgoue from two blaxploitation-era films – \textit{Dolemite} (1975) and \textit{The Spook Who Sat by the Door} (1973) – in lieu of hooks. Both of these production decisions instill an air of political consciousness to the track, in keeping with Noname's underground image. 

Joanna Demers notes that the practice of coding the revolutionary politics of Black Americans through a blaxploitation sound is common to hip-hop as an art form. She writes that historically, rappers have ``[monolithically interpreted blaxploitation] films as unified both politically and morally\textellipsis The hip-hop movement neatly compressed [a] more pessimistic view of racial relations under the aegis of Black Power.''\footnote{\cite{joannademersSampling1970sHipHop2003}, 50.} Noname and Phoelix's track participates in this lineage of sounded political consciousness through both autosonic and allosonic techniques.

\section{The Co-Creation of Alternative Identity}
This paper has highlighted some of the ways in which heterogeneity, cohesion, development, and repetition can be mapped into the architecture of underground hip-hop beats. Producers code these phenomena into their music in myriad ways: elements of production that unify the sounded underground are alterations of verse and hook lengths and forms, manipulations of recorded material at the sample level, and additions of musical and textual signifiers alongside the emcee's flow. Via transcriptions that demonstrate both a beat's capacity for development and its repetitive structural layers, I concluded that producers and their beats play a coequal part to the rapper in demonstrating the alterity of underground identity. Future iterations of this project, however, will necessarily turn toward the role of the rapper in sounding the underground.

\chapter{Structural and Performance Tools for the Underground Emcee}
\onehalfspacing
\label{chapter3}
\section{A Working Definition of Flow}

In comparison to its conceptual counterpart \emph{the beat}, a ``flow'' resists a clear 
definition in academic and technical discourse. While there seems to be some agreement
that the beat is comprised of musically-distinct layers looping throughout a piece, 
definitions of flow range from anywhere as specific as ``simply the rhythms and rhymes
[a hip-hop song] contains''\footnote{
    \autocite[63]{pauledwardsHowRapArt2009}. While Edwards' statement comes within the 
    context of an instruction on rapping (and therefore his reductiveness may be 
    pedagogically useful), this attitude toward flow is implied by music academics who 
    transcribe flow primarily on one line of staff notation; such transcriptions implicitly
    show flow as something existing within a fixed (or indiscernible) pitch space, with 
    unambiguous, highly discernible rhythms.} 
to as wide-ranging as ``all of the rhythmical and articulative features of an emcee's 
delivery of the lyrics.''\footnote{
    \cite{kyleadamsMetricalTechniquesFlow2009}.} 
Before I name and discuss a few of the stylistic hallmarks of flow in underground hip-hop,
it will be worthwhile for me to construct a definition that ascertains what musical elements
comprise it.

Consider, once more, Open Mike Eagle's characterization of his own relationship to MF 
DOOM's flow: that Eagle has to be careful with it because of how much time he has spent
`in DOOM's mind.'\footnote{
    \cite{estellecaswellRappingDeconstructedBest2016}. For my earlier discussion of this
    passage and its relationship to meditative listening,  see Section~\ref{listeningasmeditation}.} 
The fact that Eagle can characterize his relationship to the style of DOOM's flow demonstrates
that flow encompasses more than rhythm and rhyme alone; by themselves these elements do not
account for flow as being tied to an emcee's specific or generic style. Flow, then, should
be conceived more broadly than the manifestation of rhythms through the vocal delivery of
rhymes.

At the same time, emcees frequently describe flow as text in relation to the music, drawing 
contrasts with the non-musically bounded epithet ``poetry.'' The emcee Rakim's oft-cited 
definition of rap (that it is ``rhythm and poetry'') creates a distinction between texts that
can be rapped and those that are conceived within the poetic medium. Clarifying this distinction,
the emcee Myka 9 of Freestyle Fellowship offers, ``sometimes I might write a poem, a spoken-word
poem, but then morph that into a rap rhythmically.''\footnote{
    \autocite[63]{pauledwardsHowRapArt2009}.}
Myka 9's insight helps to construct a continuum for rapped text: it exists on a spectrum from 
texts that are spoken in musically-bounded ways and non-musically-bounded ways.\footnote{
    Interestingly, rapped verses can be and often are delivered on varying degrees of this
    spectrum. Mitchell Ohriner notes two distinct modes of delivery, which he calls speech-rhythmic
    and music-rhythmic, based on the degree of non-alignment between the musical meter and the 
    degree to which syllable onsets correspond to metrical positions (see 
    \cite{mitchellohrinerLyricRhythmNonalignment2019}.}

This distinction between the rap's structure as text and its performance as music is 
foundational to the method by which I interrogate the concept of flow in this chapter. 
I believe the techniques underground emcees employ when rapping can be divided into 
categories I refer to as \emph{structural} and \emph{performance} techniques. Structural
techniques deal primarily with rap as text, considering its syntactic elements with 
primacy  over their manifestation as a stream in the musical object. By contrast, 
performance techniques focus on rap as music, considering the ways in which an emcee 
makes manifest the structures they contrive when composing texts. Performance techniques
also encapsulate the decisions made when treating the voice as an object within a 
digital recording. With these categories, I aim to examine how, as Mitchell Ohriner 
claims, flow ``encompasses phrasing, rhythm, meter, accent, patterning, and groove, not
to mention the relations among these parameters.''\footnote{
    \autocite[28]{mitchellohrinerFlowRhythmicVoice2019}.}

\section{Thesis Statement \& Definitions}

In this chapter, I examine a few notable techniques that underground emcees employ 
in structuring and performing their flow; I argue that, like producers, their choice
to use these techniques reaches listeners as underground. The four techniques I define
below do not form an exhaustive list, but rather are salient and illustrative in how 
they relate to my conception of the subgenre. My terms are also  divided into the 
subcategories of structural and performance techniques, a distinction I draw in my 
understanding of how emcees work with the text that makes up their flow. When the 
underground emcee inhabits the role of writer, they experiment with structuring 
techniques including \emph{pivot rhyme} and \emph{closing fragmentation} to craft 
their verses. Stepping into the recording booth or onto the stage, the emcee shifts
into their role as rapper and therefore draws on performance techniques such as 
\emph{mimesis} and \emph{processing} to shape their vocal delivery. Each of these 
techniques serves as the focus of one of the close-readings that follow, so I will
clarify their functions before provide examples.

As a device for constructing a verse, a pivot rhyme allows an emcee to execute a shift 
in end rhyme; it occurs when the concluding words of the previous bar conjure up an idea
that is semantically linked  to the bar's topic but does not rhyme. In this instance, the
rapper chooses to displace the lyric past the next bar line, allowing that word or phrase 
to serve as the primary rhyming sound within a new repetition of the beat pattern. Often 
the emcee uses this instance as a way to play with audience expectation, taking the listener's
focus on this final rhyme to shift towards a topic semantically distinct from the previous bar.

Underground emcees employ closing fragmentation as another method of structuring a verse, 
particularly to mark the end of some sort of formal section. To accomplish this, emcees will
deliver a ``full'' textual phrase, usually structured as a setup and punchline, then signal 
a close through breaking up and repeating the component elements of that textual phrase in a 
more improvisatory and loosely-organized style of delivery. Fragmentation and repetition here 
do not function as they do in the construction of a hook; rather, they demonstrate to the 
listener that some larger textual unit\textemdash a phrase, a verse, the song itself\textemdash
is ending.

With both performance techniques, emcees articulate the role of their voice as a layer
\emph{amongst} the rest of the mix, rather than the principal element within it. In particular,
the use of mimesis, or stylizing vocal delivery to mimic other elements in the beat layer, 
draws attention to the other musical elements in an attempt to foreground their musicality.
This promotes the other musical layer to a co-soloist role, rather than functioning as a
background loop or layer, allowing the listener to focus on it as more than just 
accompanimental structure beneath the vocal flow.

Processing more generally refers to the manipulation of digital audio after it is recorded
(especially through the use of EQ, compression, and reverb software or hardware), but here 
I use it to refer to methods of altering the vocal signal as a form of electronic composition.
In particular, rappers use processing effects like delay, distortion, and pitch transposition
to alter the audio of their voice during or after tracking it to treat it as a musically 
manipulable element; the vocals here function as \emph{sound} as much as they do 
text.\footnote{
    My definition of processing here is a broader version of the definition of ``glitch'' 
    in Chapter 2 (see p.~\pageref{glitch}.)}
In this way, processing treats the voice as if it were any other musical layer and, as a 
result, any layer can be listened with as much consideration as the voice.

%clearpage
\section{Structuring Techniques}
The transcription of flow makes clear the limitations of staff notation, perhaps more than
transcription of any other musical element. Throughout this project, my dedication to 
transcription in standard notation arises from my desire to emulate what Kofi Agawu 
conceptualizes as a ``post  colonial transcription,'' one based in an ``ideology of 
sameness so that\textellipsis we can gain a better view of difference.''\footnote{
    \autocite[67]{kofiagawuInventionAfricanRhythm2003}. Agawu's chapter traces a history of 
    non-African scholars transcribing Northern Ewe drumming in a way that both essentializes
    and exoticizes African music as primarily rhythmic and therefore fundamentally different
    from Western approaches to music making. Although I do not wish to employ colonization 
    uncritically as a metaphor for music theory's relationship to hip-hop, I do believe that
    the same essentializing and exoticizing tendencies would manifest if I were to completely
    avoid standard notation for this repertory.}
However, this approach exists at odds with another point noted by Agawu: no singular mode
of representation can sufficiently convey the totality of a musical listening 
experience.\footnote{
    \autocite[187]{kofiagawuAfricanRhythmNorthern1995}.}
In dealing with flow as a textual structure independent from its musical manifestation, 
I will opt to use another mode of representation: namely, Ohriner's $modulo$ 16 grid 
transcriptions.\footnote{
    For a detailed overview of the system and a few sample transcriptions, see 
    \autocite[xxviii--xl, 7--9]{mitchellohrinerFlowRhythmicVoice2019}.}

Ohriner's method of transcription simplifies certain elements that become problematic
when transcribing flow in standard notation. First, his 16-point grid for a bar avoids
proliferate use of eighth, sixteenth, and thirty-second notes, not to mention syncopation
between them. Subsequently, his method also simplifies the naming structure by labeling 
each metric position with a number from 0--15; one can more succinctly communicate a 
syllable landing on ``the third sixteenth note of beat 3'' as position 10, for instance. 
Finally, Ohriner's system does not force a transcriber to choose fixed metrical positions
in the same way standard notation and adaptations of TUBS for flow do; if a syllable onset
occurs slightly before the beat, this can be accounted for simply by moving the corresponding
circle. Although vocal groove and non-alignment with the meter is not a focus in this chapter,
the ability for a system to adapt between quantized and non-quantized rhythms is foundational
to being able to transcribe flow.

\phantomsection
\subsection*{\centering Madvillain's ``Great Day''}
\addcontentsline{toc}{subsection}{Madvillain's ``Great Day''}

Madvillain, the moniker under which DOOM and Madlib released all of their collaborations
except for ``One Beer,'' looms large in the world of underground hip-hop. One unrelenting 
focus of the accolades for  their 2004 double LP \textit{Madvillainy} is DOOM's flow: in 
the words of Ta-Nehisi Coates, ``[\textit{Madvillainy}'s] singular sound came mostly from
[DOOM's] raspy baritone rendering a sort of nerdcore poetry.''\footnote{
    \cite{ta-nehisicoatesMaskDoomNonconformist2009}.}

This claim to DOOM's uniqueness, eccentricity, and artistry finds purchase beyond the 
mythologizing in which Coates and much of the rest of the hip-hop community partake. 
Kyle Adams, for instance, notes that on ``All Caps'' both the melodic samples (various 
portions of the main theme for the NBC crime drama \textit{Ironside}) and DOOM's flow 
``[seem] to float free of the meter, being only weakly tethered to
it by the drum sample.''\footnote{
    \cite{kyleadamsMetricalTechniquesFlow2009}. What Adams notes about the 
    \textit{Ironside} sample likely bolsters my claim that sample slippage is a 
    frequently employed technique in underground hip-hop production, but it is not 
    uniformly employed across Madlib's sampling practice (more on this below).}
Adams arrives at this characterization of DOOM's flow in examining the structure of 
its syntax: particularly, how rhymes in ``All Caps'' rarely fall in regular metrical 
places and that syntactical units (or phrases) often cross metrical boundaries. DOOM's
novel use of this irregularity, per Adams, contributes to the perception of DOOM as an
underground rap artist.

While metrical ambiguity and enjambment may play a role in DOOM's sounding as underground
on ``All Caps,'' I am hesitant to accept this as the whole picture of DOOM's alternative 
identity. A counterexample to the techniques Adams accounts for on ``All Caps'' emerges from
the track that immediately follows it on  \textit{Madvillainy}\textemdash ``Great Day.'' The 
beat's primary melodic sample comes from ``How Do You Believe,'' an instrumental funk track 
by Stevie Wonder, released in 1968 under the alias Evits Rednow. Madlib closey aligns the 
sampled elements (electric piano, harmonica, bass, and auxiliary percussion) closely with 
the drum break he uses, ushering in a hypermetric downbeat with a C-sharp minor gospel lick
every fourth bar in the A section.\footnote{
    This type of metric regularity also pervades ``One Beer'', where the whole
    instrumental before the skit is sampled from one track by the funk band Cortex 
    (see Section~\ref{samplebased}.)}

Figure~\ref{fig:doomfirstpiv} shows how DOOM, as might be expected, flows with a syntactical
structure of a  similar regularity. Entering after the downbeat where the sample loops, he 
raps: ``Mad plays the bass  like the race card, / Villain on the case to break shards and 
leave her face scarred.'' Over these two  bars, DOOM structures a setup and punchline within
the confines of the barline, in addition to landing the two-syllable rhyme that closes each bar
on positions 12 and 14. The internal rhyme in the second bar (``break shards'') occurs in a 
metrically weaker position (6 to 8, as opposed to 4 to 6), but falls within what Ohriner refers
to as the same durational segment (in this case, a value of 2 on the grid). Rather than 
interrupting the sense of regularity in the flow, I argue the internal rhyme in this bar create
s an anticipation for the rhythmically ``restored'' position of the end rhyme.
%\clearpage

    \begin{figure}[!ht]
        \centering
        \includegraphics{images/figures/chp 03/059115greatdayfirstpivot.pdf}
        \caption{First pivot in Madvillain's ``Great Day,'' 0:59--1:15.}
        \label{fig:doomfirstpiv}
    \end{figure}

Over the next two bars, DOOM's flow increases in syntatic and rhythmic complexity, anticipating
the four-bar loop of the sample. Bars 25 and 26 increase the number of syllable onsets from 18 
to 22, as well as the  frequency, position, and syllable count of the rhymes. In the setup bar, 
DOOM creates an internal  rhyme with ``Groovy dude'' and ``prove to  be rude,'' each fitting 
within a durational segment of 2 from positions 2--4 and 10--12, respectively. In the punchline
, ``movie food'' once again restores the metric position of the end rhyme, preceded by two softer,
internal rhymes (``stuff is like'' with ``what you might'' falling between 2--4 and 6--8). Despite 
the increase in internal complexity, the  syntactic structure of these bars maintains regularity:
DOOM fits a sentence structured ``$x$, but $y$'' within the same time span as the previous setup 
and punchline.

My focus on the regular structure in the first four bars is important for establishing a connection
to the next four, two of which I transcribed as a part of Figure~\ref{fig:doomfirstpiv}. I perceive 
a semantic link in the content of bars 26 and 27 that allows DOOM to pivot to a new rhyme scheme but
in the process invites me as a listener to anticipate something different. It works as the setup for
a pivot rhyme but functions in a slightly different manner. My expectation at the end  of bar 26 is 
that DOOM is going to rap about butter: not only because I put butter on ``movie food,'' but also 
because he has done so twice already on the album up to this point.\footnote{
    DOOM refers to his ``buttery flow'' on ``Raid'' and Madlib's beats as ``so butter'' on ``All 
    Caps.''}
DOOM, of course, thwarts these expectations, positing ``uh\textellipsis what is jalapenos?'' as 
if  it were a botched  \textit{Jeopardy!} answer. He draws out a humorous semantic link between
these two bars, but rather than opting to construct a pivot rhyme with the word``butter'', he 
opts for jalapenos as an end rhyme for the following quatrain. With its final two syllables 
falling on positions 12 and 13, jalapenos ushers in a set of highly regular lines, each ending
with ``holla at ya  seniors,'' ``hashish fienda,'' and ``grass is greener'' in the same metric
positions.

    \begin{figure}[!p]
        \centering
        \includegraphics{images/figures/chp 03/121137greatdaysecondpivot.pdf}
        \caption{Second pivot in Madvillain's ``Great Day,'' 1:21--1:37.}
        \label{fig:doomsecondpiv}
    \end{figure}

DOOM's avoided pivot rhyme and the regularity of the subsequent quatrain provide crucial 
context for a similar moment between bars 34 and 35, an instance where I do apply the term
pivot rhyme. Figure~\ref{fig:doomsecondpiv} overviews a quatrain with a similarly regular
structure. ``Wishes''  falls on positions 11 and 12, displaced 1 position forward from the
ends of the two following  syntactic  closes: ``mad glitches'' and ``jaw twitches.'' The 
closely related structuring of the  middle bars of the quatrain set the listener up to hear:
``One thing this party could use is more\textellipsis booze.''And he makes that ellipsis 
audible! DOOM plays on listener expectations not only concerning rhyme structure, but also
their expectations concerning rap's ``topical stereotype:'' the humor in this instance
comes less from the fact that he implies the word ``bitches'' but that you expect that he
will say it.\footnote{
    My analysis of this moment varies slightly from Estelle Caswell's, who cites a 
    \textit{Pitchfork} reviewer's quotation that the ``hiliarity'' of this moment ensues
    from DOOM's non-sequitur (see \cite{estellecaswellRappingDeconstructedBest2016}).}
In effect, DOOM points the mic towards you at the end of this bar, asking you to examine
your listening expectations.

DOOM's patterning of flow, his regulating of syntactic structure, and his harnessing of
listener's expectations concerning rhyme lead into this moment; he crosses a barline, 
intentionally, yet again giving the listener a moment to expect one close to the phrase, 
before veering off in another direction. \textbf{Two more sentences here?}

%\clearpage
\phantomsection
\subsection*{\centering Armand Hammer and R.A.P. Ferreira's ``Dead Cars''}
\addcontentsline{toc}{subsection}{Armand Hammer and R.A.P. Ferreira's ``Dead Cars''}

Armand Hammer\textemdash the name given to the collaboration between billy woods and
ELUCID\textemdash is a flagship act on the roster of the New York based underground 
label Backwoodz Studioz.\footnote{
    \textbf{cite reviews?}}
The track ``Dead Cars'' from the 2019 LP \textit{Shrines} features production from 
Kenny Segal, as well as a feature verse from R.A.P. Ferreira\textemdash two premiere 
artists from Ruby Yacht, an underground hip-hop label for which Ferreira serves as 
the owner-operator. In short, as a track, it features some of the underground artists
who exist at the center of my conception of the genre, whose techniques as performers
deeply inform my construction of underground methodology.

The track features several short verse-parts delivered by each of the emcees. Counting
at 64 bpm, ELUCID, Ferreira, woods  rap for eight bars at a time, comparable to how
jazz musicians may trade fours on a jazz tune. Ferreira's feature verse enters after
ELUCID's second eight-bar section, accompanied by a drastic change in the musical 
texture. As Segal layers out the drum loop, he recomposes the orginal synth-texture
chord loop: a Fender Rhodes strikes chords on the downbeats and samples reversed audio
to construct a melodic line in between downbeats. Figure~\ref{fig:roryclosingfrag} 
transcribes the last six bars of Ferreira's verse, two bars prior to the drum's
re-entry.

    \begin{figure}[!htp]
        \centering
        \includegraphics{images/figures/chp 03/144206deadcarsendfrag.pdf}
        .\caption{Closing fragmentation in ``R.A.P. Ferreira's ``Dead Cars,'' 1:44--2:06.}
        \label{fig:roryclosingfrag}
    \end{figure}

Without a drum loop's articulation of a clear metrical hierarchy, Ferreira lets his text
float around freely in micro-rhythmic space; however, Ferreira uses rhyming syllables as
an anchor for the more lax placement of the intervening syllables. He places the end of 
his rhyming in slight anticipation of positions 4 and 12, where the snare drum often hits
in a drum loop.\footnote{
    According to Ohriner, orientation around these two beats is foundational to the 
    construction of the boom-bap texture pervasive in hip-hop beats (see his discussion
    of A\$AP Rocky's ``Purple Swag:  Chapter 2'' in 
    \autocite[18]{mitchellohrinerFlowRhythmicVoice2019}). My decision to mark out what 
    seems to be a slower, half-time tempo in ``Dead Cars'' is based on this orientation
    normalizes my hearing of the vocal sections as verse-parts rather than full verses.}
This rhyme structure may seem commonplace, but his choice to continue doing so without 
the drum loop strictly articulating them is important for orienting a listener in the 
sparse musical texture.

When, in bar 31, the drums do re-enter, Ferreira stops rapping new text. Syntactically,
his remaining four bars only rely on what I hear as two units: ``Bronze Kafka metamorph, 
Black Orpheus set the course / in the Backwoodz, jiggin' with no  remorse.''\footnote{
    My hearing of the first bar as a full syntactic unit is in part due to his rapping 
    of it within the space of one bar, but it has some semantic justification. I hear
    ``Bronze Kafka'' and ``Black Orpheus'' as two epithets the emcee gives 
    himself\textemdash the first a play on the author's name; the second, a nod to the
    1959 Brazilian film of the same name.}
The second of the two, a shoutout to the label to which his host emcees are signed, serves
as the textual focus of his shift into a more improvisatory form of vocal delivery. Maintaining
the rhythmic position of the final rhyming word ``remorse,'' Ferreira shifts the text around
in his three repetitions of the phrase: the verb-form \textit{jiggin'} becomes the adjective
\textit{jiggy} as Ferreira shifts the metric placement of the first syllable to on the beat
in bar 32, position 1 and behind the beat in position 8. Before cycling back to a near direct
repetition in bar 33, he slightly accelerates ``Backwoodz'' from its original position, though
still completing his statement as a pickup to the downbeat of the next bar.

For a listener familiar with Ferreira's style, the mode of delivery into which he shifts
functions as signposting that the verse is coming to a close. When he shifts into a textual
delivery that is more fragmented, repeated, and rhythmically varied (in this instance, while
delivering his shout out to the other emcees for hosting him), he is fragmented and repeated
text to tell an audience that he is finished rapping; therefore, I refer to this structuring 
device as \emph{closing fragmentation}.

Ferreira's technique relates to methods of delivering text that are used to signal a change
in formal sections theorized by Ben Duinker. In particular, Duinker identifies fragmentation
and repetition as signs that the rapper is no longer delivering a verse but instead either a
hook or a looser organized vocal-section.\footnote{
    \autocite[98--101]{benduinkerSongFormMainstreaming2020}. Duinker's definition of hook is 
    discussed on p.~\pageref{duinkerhookdef}. His ``looser-organized'' sections function like
    a catch all category; he includes ``ad-hoc, ametric vocals, skits, or\textellipsis [rapping]
    that doesn't function like a verse or hook'' as typical in this section-type.}
Constructed from examining form in mainstream hip-hop, Duinker's categories break down due to
Ferreira's use of the technique within the verse as a formal unit. The rhyme scheme matches of
the fragmented units matches that of the verse immediately preceding it (so too the metric 
placement of its rhymes). And though the phrase's semantic content may be construed as a kind
of ``hype'' vocal that Duinker identifies, Ferreira imbues them with a structural function that
cannot be located outside the verse: signalling its close. Closing fragmentation, then, become
s a valuable way to describe an emcee's use of fragmentation and repetition more with a new 
categorical purpose. 

In my listening, this technique most prominently occurs in Ferreira's catalog. His influence, 
however, can perhaps be traced in the use of this technique by other emcees in his satellite:
in particular, other members of the Ruby Yacht crew including Pink Navel and S.AL have taken 
to the technique when constructing their own rhymes. Perhaps most telling that this technique
is more than just Ferreira's  idiosyncrasy: each of the verse-parts that precede Ferreira's on
``Dead Cars'' end with some form of closing fragmentation as well.

%\clearpage
\section{Performance Techniques}
\phantomsection

An emcee's performance of their rhymes demands a slightly different approach to the transcription
of a musical object, one which can account more clearly for pitched elements and their interrelation
in a musical texture. Ohriner's monograph on flow does not uniformly employ a method of transcription
that accounts for verses in pitch-space, later works of his develop a contour-graph method of showing
changes of pitch broadly over the range of G2--G4 with gridlines.\footnote{
    \textbf{Ohriner, 2019, analysing pitch of the rapping voice, 418--419.}}

My approach to representing pitch marks a point of departure from Ohriner's, likely due to conflicting
readings of the imperatives about transcription discussed in the prior section. He takes Agawu's
admonitions to use transcription as a means of illuminating musical practice as a drive to look beyond
the confines of standard notation. On the contrary, I adamantly believe Agawu suggests music theorists 
should use standard notation \textit{creatively} to articulate a complex picture of its sameness and 
difference from familiar repertory within our sphere of discourse. It is for this reason that my 
transcriptions in this section harness standard notation diacritically, transcribing rap flows pitched
in relation to themselves and other musical elements.\footnote{
    Such an approach to notation has precedence in \textbf{Robert Komaniecki, Vocal Pitch in Rap Flow} and 
    \textbf{Martin Connor, Musical Artistry of Rap}.}

\subsection*{\centering R.A.P. Ferreira's ``NONCIPHER''}
\addcontentsline{toc}{subsection}{R.A.P. Ferreira's ``NONCIPHER''}

If one were to look through R.A.P. Ferreira's catalog on any major streaming platform, they would
guess that the 2020 LP \textit{Purple Moonlight Pages} is an artistic debut, but this is not case.
In reality, Ferreira's career spans the entire preceding decade with a full catalog of releases under
various monikers, the most prominent of these being Milo. He ties the decision to switch to his own
name closely to the concept of identity: ``my name is rory allen phillip ferreira. and after 9 years
pro rapping i'm confident enuff to put it on what i make. i never knew who milo was.''\footnote{
    \textbf{Tweet on May 31, 2019.}}
Ferreira's statement\textemdash that he now raps in a way deserving of his given name rather than 
a chosen alias\textemdash speaks to the intentional curation of identity that takes place within 
underground hip-hop and therefore demonstrates that his musical style as ``R.A.P. Ferreira'' is 
something worth investigating.

Along with his persistent use of closing fragmentation noted above, one element of his music that
manifests in musical performance is his vocal mimesis of elements of the musical texture. 
Figure~\ref{fig:rorymimesis} examines instances in the second verse of ``NONCIPHER'' where Ferreira
delivers his verse in a dialogue with the alto saxophone, live-tracked by Aaron Shaw.
    \begin{figure}[!t]
        \centering
        \includegraphics{images/figures/chp 03/120137nonciphermimesis.pdf}
        \caption{Mimesis in R.A.P. Ferreira's ``NONCIPHER,'' 1:20--1:37.}
        \label{fig:rorymimesis}
    \end{figure}
Immediately following a verse that ends in closing fragmentation, Ferreira and Shaw both anticipate
m.~30 with syncopated entrances to the second verse. Although Ferreira cuts across the saxophone's 
phrase beginnings and endings later, his first clause of text (``Dared to peak through the viewfinder'')
aligns with Shaw's performance in both rhythmic placement and pitch contour. Shaw's lick also informs
his delivery of the subsequent clause (``up close with a rude reminder''). As Shaw presses on, 
Ferreira approaches the downbeat of m.~30 in close approximation of Shaw's performance, singing the 
phrase ``shufflin' chicken bones'' with the same melodic inflection as Shaw, adding a shake to his 
delivery at the phrase's end.

After a brief departure from the mimicking the saxphone, Ferreira once again locks in with Shaw at
the onset of m.~34. Traversing into a higher register to mimic the saxophone's G5, he raps ``[hidden]
motives turn to rigor mortis'' in a related but staggered descent from Shaw's outlining of a C minor
triad. The lick outlining G to C (m.~34, beat 2) delivered by Shaw gives Ferreira another contour
to mimic within the beats that immediately follow: ``rigid hortense'' ends with a shadow vowel that
inflects upwards, and ``vivid reminders'' places its final syllable in a higher register than those
which immediately precede it. In general, Ferreira uses mimesis at this verse's beginning to align 
his vocal performance to  the saxophone line.

The rhetorical value of this performance technique is two-fold as it relates to Ferreira's identity 
as an underground emcee. That he draws us into the saxophone melody as a demonstration of his own 
musicality, marking his voice as a musical participant in the whole texture. The ``beat'' in this 
verse section is not simply something happening around or behind Ferreira's flow; instead, he puts 
himself on display \emph{as} a listener, encouraging a similar engagement \emph{from} the listener.
Moreover, this reorders the hierarchy of the rap song as a musical object: the emcee's performance
does not supersede any instrumental stream simply because of its use of text. Instead, the text is
fused to the music surrounding it.

%\clearpage
\phantomsection
\subsection*{\centering Moor Mother,  billy woods, and ELUCID's ``Tiberius''}
\addcontentsline{toc}{subsection}{Moor Mother, billy woods, and ELUCID's ``Tiberius''}

An underground emcee's fusion of text to music also informs my processing as a performance technique. 
Underground emcees are often responsible not only the construction and performance of their own text 
but often the means by which their voice is captured in the recording process. Although a quality 
microphone and some compression, EQ, and reverb does the trick for many emcees, others such as Moor 
Mother explore a wider range of possibility for what exactly vocal tracking of a rap verse entails.
Her verse on the track ``Tiberius'' from the collaborative LP \textit{Brass} (2020) with billy woods
illustrates her more maximal approach to fitting her voice in the musical texture.

Moor Mother's use of multitracking and effects processing demands a wider conception of the idea of a
unitary ``vocal stream'' occurring the beat. Figure~\ref{fig:moormotherprocess} details the components
of her vocal tracking in the first fifteen bars of her verse. Entering on the heels of ELUCID's final 
bar, she repeats the text he just delivered (``It's me dummy!'') as a pivot into her own verse. The 
channel on which this verse enters (Moor Mother II in the figure) proves to be a secondary, ad-lib 
track, as a more-prominent vocal  enters in the next bar (Moor Mother I in the figure). This vocal 
channel bifurcates into a lower, primary layer of vocals and one created with a transposition effect,
sounding around a fifth higher than the primary layer, represented by diamond-head notation in 
the figure.

\begin{figure}[!p]
    \centering
    \includegraphics[width=\textwidth]{images/figures/chp 03/107136tiberiusprocessing.pdf}
    \caption{Vocal processing in Moor Mother and billy woods' ``Tiberius,'' 1:07--1:37.}
    \label{fig:moormotherprocess}
\end{figure}

Moor Mother uses these two vocal channels to deliver the first twelve bars of her verse: a series
of questions rapped in a triplet flow pattern that slightly anticipates the beat. The hurried, 
harsh delivery of her verse crests in mm. 39--41, where she shifts the delivery far enough forward
in the grove that the first eighth-note of her tuplet on the text ``catalogue of death and dismay''
syncopates before the downbeat. In the bar before this shift, she adds an echo of the triplet ad-lib
on ``how long `til you break,'' panned hard left in the mix and processed with a low-pass EQ filter
to diminish its prominence in the mix; the echoes shown in Figure~\ref{fig:moormotherprocess} on
an ossia staff below Moor Mother II.

In the following three measures, Moor Mother's use of processing helps to articulate an oncoming 
boundary within her verse\textemdash a transition to a flow with new rhythmic gestures and articulations.
Repeating text in a new rhythmic position, she arrives at the question ``how much can you bear?'' 
in slight anticipation of the downbeat of m.~44. Another echo repeats the text ``bear'' mid-measure,
this time in both vocal channels. Trailing off during the final question ``How long is the (wait),''
she delivers a guttural ``yeah!'' exclamation on the following downbeat.

%\clearpage
A sonic marker of the transition to the rest of the verse, this moment is marked out with effects 
processing throughout the texture: the synth, which had dropped out upon the textual repeat, now
re-enters with a wash of noise while Moor Mother's textual exclamation is extended through the 
use of a glitch-like delay. The delay used in this instance increases the frequency of signal 
repetitions at the same time that it extends the feedback time for the loop, letting the sound
of the ``Yeah!" more quickly double up on itself and extend across the remainder of the bar. This
timbre is also heightened in the mix, as the delayed signal ``ping-pongs'' back and forth from 
the right to left audio channels.

At this point, the function of the exclamation is no longer textual in the way any of the questions
preceding it have been. Through the use of glitch, the audio of Moor Mother's voice carries less 
semantic weight and functions more abstractly as a sound, receding into the mix as she begins the
next section of her verse. But at the same time, Moor Mother's treatment of the audio of her voice(s)
throughout the excerpt demonstrates that her vocals have more than one function. Certainly, they carry
the text, and the text carries semantic value, perhaps even a poetic meaning for the listener. But the
use of several distinct and overlapping vocal signals, each with its own performative subtleties 
stratifies the concept of the ``emcee's voice'' within the musical texture. Her particular approach
heightens the voice to become \emph{synthetic}; each layer's treatment could be likened to several
oscillators used to create a synthesizer patch. Rather than approaching vocal treatment as 
contrapuntally distinct lines, Moor Mother's audio processing creates heterogeneity within a single
``line'' of music, and likens its role to any other multitracked musical element in the mix.

%---------------------------%

\singlespacing
\printbibliography
\addcontentsline{toc}{chapter}{Bibliography}
\nocite{*}

%---------------------------%
\chapter*{Appendix}
\addcontentsline{toc}{chapter}{Appendix}
\onehalfspacing
\appendix \label{appendix:fullroadmaps}
\renewcommand{\thetable}{A.\arabic{table}}
\setcounter{table}{0}

This appendix contains five tables, the first of which is referenced in Chapter~\ref{chapter1} and
the remaining four in Chapter~\ref{chapter2}. The first contains a short list of underground hip-hop
tracks throughout the 2010s that make sonic references which function as jazz codes. Organized by
release date of the hip-hop track, the table indicates the type of reference (sample or interpolation)
as well as the artist and title for both the hip-hop track and its referent.

The tables from the second chapter are expanded ``roadmap'' transcriptions of each of the four case
studies the chapter presents. They differ slightly from the in-chapter tables in their organization:
rather than tracking form by bars, they track it by repetitions of samples and loops. Each musical
participant gets its own column; samples are named by their sources, while non-sampled loops are
given a descriptive word, phrase, or musical term. When. samples repeat multiple times within one
time code designation, I use a notation that mimics repeat signs in standard notation with a multiplier
of the number of repeats if there is more than one (||: $x$ :|| x3, x4, etc.). If that element repeats
as is in another section, I use another type of repeat sign to designate that style of repeat (•//• x3,
x4, etc.). The goal of these roadmap transcriptions is to notate the overall musical texture with a
production-forward approach. If a reader would like to compare between this style of notation and the
formal designations I opt for in chapter, they need only to match up the time codes given in each
table, spotting in to see the musical components within my given formal designation.

\begin{sidewaystable}[p]
    \centering
    \small
    \begin{tabular}{|c|c|c|c|}
         \hline
        Year & Track & Referent & Type \\ \hline
        2010 & Kendrick Lamar -- ``Rigamortus'' & Willie Jones III - ``The Thorn'' & Sample \\ \hline 
        2012 & billy woods (ft. Elucid) -- ``Sour Grapes'' & Miles Davis -- ``Pharoah's Dance'' & Sample \\ \hline
        2012 & billy woods -- ``Body of Work'' & Nina Simone -- ``Work Song'' & Sample \\ \hline
        2012 & billy woods -- ``Crocodile Tears'' & Muddy Waters -- ``Champagne \& Reefer'' & Interpolation \\ \hline
        2012 & billy woods -- ``DCMA'' & Junior Murvin -- ``Police and Thieves'' & Interpolation \\ \hline
        2012 & Kendrick Lamar -- ``Sing About Me, I'm Dying of Thirst'' & Grant Green -- ``Maybe Tomorrow'' & Sample \\ \hline
        2013 & Milo (ft. Busdriver) -- ``The Gus Haynes Cribbage League'' & Quincy Jones (ft. James Ingram) -- ``Just Once'' & Sample \\ \hline
        2015 & Kendrick Lamar -- ``King Kunta'' & James Brown -- ``The Payback'' & Interpolation \\ \hline
        2015 & Milo (ft. Hemlock Ernst) -- ``Souvenir'' & Shuggie Otis -- ``Rainy Day'' & Sample \\ \hline
        2016 & Scallops Hotel -- ``Niopo Tree Stipend'' & Ella Jenkins -- ``Moon Don't Go'' & Sample \\ \hline
        2016 & Scallops Hotel (ft. SB the Moor) -- ``Lanquidity'' & Sun Ra -- ``Lanqudiity'' & Sample \\ \hline
        2017 & Kendrick Lamar ``XXX.'' & James Brown -- ``Get Up Offa That Thing'' & Sample \\ \hline
        2017 & Milo -- ``Call + Form (Picture)'' & Eddie Munji III -- ``Doon Po Sa Amin'' & Sample \\ \hline
        2017 & Milo (ft. Elucid) -- ``Landscaping'' & Sun Ra -- ``Quiet Ecstasy'' & Sample \\ \hline
        2017 & Scallops Hotel -- ``Ain't No Hustle Where I Live'' & Stanley Cowell -- ``Here I Am'' & Sample \\ \hline
        2017 & Scallops Hotel -- ``A Beat for My Lil Boy'' & Sun Ra -- ``Where There Is No Sun'' & Sample \\ \hline
        2018 & Armmand Hammer -- ``VX'' & Prince Far I -- ``Throw Away Your Gun'' & Sample \\ \hline
        2018 & Armmand Hammer -- ``No Days Off'' & Sun Ra Arkestra -- ``The All of Everything'' & Sample \\ \hline
        2018 & Milo -- ``Tiptoe'' & Hank Crawford -- ``Teach Me Tonight'' & Sample \\ \hline
        2019 & billy woods -- ``Fnu Lnu'' & Hank Crawford -- ``Wildflower'' & Sample \\ \hline
    \end{tabular}
    \caption{References to jazz, soul, and funk pieces in 2010s underground hip-hop.}
    \label{tab:jazz_references}
\end{sidewaystable}


\begin{sidewaystable}[p]
    \small
    \centering
\begin{tabular}{|c|c|c|c|c|c|}
     \hline
     Timecode  & DOOM & Lead Sample & Bass & Drums & Vocal Sample \\ \hline
     0:00-0:14 & ``I get no kick\textellipsis'' & ||: ``Huit'' I :|| x4 & ||: ``Huit'' I :|| x4 & ||: ``Huit'' I :|| x4 &  \\ \hline
     0:15-0:17 & ``I get a kick outta\textellipsis'' & •//• x1 & •//• x1 & •//• x1 & \\ \hline
     0:18-0:19 & ``brew!'' & & & & \\ \hline
     0:20-0:40 & ``There's only one beer\textellipsis'' & ||: ``Huit'' II :|| & ||: ``Huit'' II :|| & ||: ``Huit'' II :|| &  \\ \hline
     0:41-1:01 & ``Told him tell 'em\textellipsis'' & •//• x2 & •//• x2 & •//• x2 & \\ \hline
     1:02-1:21 & ``He went to go laugh\textellipsis'' & •//• x2 & •//• x2 & •//• x2 & \\ \hline
     1:22-1:42 & ``(skeezer) eye, and squeeze\textellipsis'' & •//• x2 & •//• x2 & •//• x2 & \\ \hline
     1:43-1:57 & ``Looser than a pair\textellipsis'' & ||: ``Huit'' I :|| x4 & ||: ``Huit'' I :|| x4 & ||: ``Huit'' I :|| x4 & \\ \hline
     1:58-1:59 & ``Few could do it\textellipsis'' & •//• x1 & •//• x1 & •//• x1 & \\ \hline
     2:00-2:02 & ``Take it from the dude\textellipsis'' & •//• x1* & •//• x1* & •//• x1* & \\ \hline
     2:03-2:23 & ``He plot shows like...\textellipsis'' & ||: ``Huit'' II :|| & ||: ``Huit'' II :|| & ||: ``Huit'' II :|| & \\ \hline
     2:24-2:44 & & •//• x2 & •//• x2 & •//• x2 & \\ \hline
     2:45-3:05 & & •//• x2 & •//• x2 & •//• x2 & \textit{Spider-man} I \\ \hline
     3:06-3:50 & & ||: \textit{Spider-Man} :|| x8 & SP-303 & SP-303 & \textit{Spider-man} II \\ \hline
     3:51-3:55 & & •//• x1* & •//• x1* & •//• x1* & ``Your attempts\textellipsis'' \\ \hline
     3:56-4:18 & & •//• x4  & •//• x4 & •//• x4 & \textit{Spider-man} III \\ \hline
\end{tabular}

\vspace{0.2cm}
\hspace{5.5in}{*choked on last two beats}
    \caption{Full roadmap to MF DOOM and Madlib's ``One Beer''}
    \label{tab:onebeerfull}
\end{sidewaystable}
\clearpage
\singlespacing

\begin{sidewaystable}
    \centering
    \begin{tabular}{|c|c|c|c|c|c|c|}
     \hline
      Timecode  & Kendrick & Lead Sample & Drums & Bass & Synth & SFX \\ \hline
      0:00-0:10 & ``Alright\textellipsis''* & ||: ``Thorn'' :|| x4 & & & & \\ \hline
      0:11-0:15 & ``Got me\textellipsis''* & •//• x2  & & & & \\ \hline
      0:16-0:26 & ``(bas)tard, I'm Marilyn\textellipsis'' & ||: ``Thorn'' II :|| & & & & \\ \hline
      0:27-0:32 & ``this is rigor mortis\textellipsis'' & •//• x1 & & & & Filter Sweep** \\ \hline
      0:33-0:42 & ``orbit, you an orphan\textellipsis'' & •//• x2 & ||: 2-bar† :|| & & & \\ \hline
      0:43-0:53 & ``(for)gin' all my signatures\textellipsis'' & •//• x2† & ||: 2-bar :|| & Drone & & \\ \hline
      0:53-0:58 & ``(suit and) tie  are suitable\textellipsis'' & •//• x1† & •//• x1 & & & ``Hey!''s \\ \hline
      0:59-1:04 & ``(He) dead! Amen! That's what\textellipsis'' & •//• x1† & •//• x1 & & & •//• \\ \hline
      1:05-1:15 & ``(Ferra)gami, so many\textellipsis'' & •//• x2† & •//• x2 & •//• & & \\ \hline
      1:16-1:25 & ``(Wrest)ling? That's irrelevant\textellipsis'' & •//• x2† & •//• x2 & & & •//•* \\ \hline
      1:26-1:30 & ``(He) dead! Yup yup! Amen!\textellipsis'' & •//• x1† & •//• x1 & •//• & G-A, D-A & \\ \hline
      1:31-1:41 & ``Got me breathin'\textellipsis'' & •//• x2† & •//• x2 & •//• & •//• & Delay Throw** \\ \hline
      1:43-1:47 & ``Got me breathin'\textellipsis'' & •//• x1† & 2-bar & & & \\ \hline
      1:48-1:57 & ``I rapped 'em and made 'em\textellipsis'' & •//• x2† & •//• x2 & •//• & & \\ \hline
      1:58-2:08 & ``my casualty, and it's casually\textellipsis'' & •//• x2† & •//• x2 & & & ``Hey!'' + Dly** \\ \hline
      2:09-2:19 & ``And I go visit\textellipsis'' & •//• x2† & fill** & Subs & & ``Hey!''s \\ \hline
      2:20-2:30 & ``(men)tion, how the far\textellipsis'' & •//• x2† & ||: 2-bar :|| & & •//• & •//• \\ \hline
      2:31-2:35 & ``(He) dead! Yup yup! Amen!\textellipsis'' &  •//• x1† & •//• x1 & & & \\ \hline
      2:36-2:41 & ``Got me breathin'\textellipsis'' & •//• x1† & •//•x1 & & & \\ \hline
      2:42-2:47 & ``(He) dead! Yup yup! Amen!\textellipsis'' & •//• x1† & •//•x1† & & & \\ \hline
\end{tabular}

\vspace{0.2cm}
\hfill{*enters at the 2nd sample repetition}

\hfill{**enters on the last two beats}

\hfill{†sample is choked, shifted, or otherwise altered}
    \caption{Full roadmap to Kendrick Lamar, Willie B, and Sounwave's ``Rigamortis''}
    \label{tab:rigamortusfull}
\end{sidewaystable}

\begin{sidewaystable}[t]
    \centering
    \small
    \begin{tabular}{|c|c|c|c|c|c|c|c|} 
        \hline
         Timecode & Milo & Drums & Rhodes & Bass I & Synth & Bass II & Vocal Sample \\ \hline
         0:00-0:11 & ``They couldn't\textellipsis''* & 6-bar halftime & 3-chord loop & & & & \\ \hline
         0:12-0:23 & ``(merci)ful, I'm\textellipsis'' & •//• & •//• & ||: E-B :|| x3 & fourths & & \\ \hline
         0:24-0:35 & ``I might\textellipsis'' & •//• & •//• & •//• & •//• & & \\ \hline
         0:36-0:47 & ``evening, I\textellipsis'' & •//• & •//• & •//• & •//• & Improv** & \\ \hline
         0:48-0:55 & ``We all\textellipsis''& 4-bar halftime† & 2 chords† & ||: E-B :|| x2† & •//• & & \\ \hline
         0:56-1:02 & & 4-bar halftime† & 1 chord† & •//• x1† & Org samp? & Improv & \\ \hline
         1:06-1:10 & & & & & & & \textit{Soul Caliber 2} \\ \hline
    \end{tabular}

\vspace{0.2cm}
\hfill{*Entrance at bar 4 with C\#$^{7{sus4}}$}

\hfill{**Entrance anticipates downbeat}

\hfill{†Sample is choked, glitched, or otherwise altered}
    \caption{Full roadmap to Milo and Kenny Segal's ``Rabblerouse''}
    \label{tab:rabblerousefull}
\end{sidewaystable}
\clearpage

\begin{sidewaystable}[t]
\begin{tabular}{|c|c|c|c|c|c|}
     \hline
     Timecode  & billy woods                       & Guitar                   & Omnichord                & Drums            & Synth  \\ \hline
     0:00-0:06 &                                   &     G5 F\#5 G5 F\#5      &                          &                  &        \\ \hline
     0:07-0:26 &                                   & ||: G5 F\#5 G5 F\#5 :||* &                          & Kick + Snare     &        \\ \hline
     0:26-0:39 & ``If I haven't\textellipsis''*    & •//•*                    &                          & •//•             & Lead*  \\ \hline
     0:40-0:52 & ``Pencil him in\textellipsis''*   &                          & ||: Gm F\#M Gm F\#M :||* & Snare**          &        \\ \hline
     0:53-1:06 & ``While Shameek\textellipsis''*   &                          & •//•*                    & •//•             & •//•*  \\ \hline
     1:07-1:19 & ``Shot the movie\textellipsis''*  & ||: G5 F\#5 G5 F\#5 :||* &                          & Full Kit*        & •//•*  \\ \hline
     1:20-1:32 & ``Whitey finally\textellipsis''*  & •//•*                    &                          & •//•*            &        \\ \hline
     1:33-1:39 & ``Dish served''\textellipsis''    &     G5 F\#5 G5 F\#5      &                          & •//•*            &        \\ \hline
     1:40-1:52 & ``Grip!''                         &                          &     Gm F\#M Gm F\#M*     & Snare            &        \\ \hline
     1:53-2:09 & ``Pace the palace\textellipsis''* & ||: G5 F\#5 G5 F\#5 :||* &                          & Full Kit*        & •//•*  \\ \hline
     2:10-2:19 & ``Not a blink!''                  &                          &     Gm F\#M Gm F\#M*     & Snare**          &        \\ \hline
     2:20-2:35 & ``My goals was\textellipsis''*    & ||: G5 F\#5 G5 F\#5 :||* &                          & Full Kit*        & •//•*† \\ \hline
     2:36-2:49 & ``Freaky shit\textellipsis''*     & •//•*                    &                          & •//•*            &        \\ \hline
     2:50-3:02 & ``No sweat\textellipsis''*        & •//•*                    &                          & •//•*            & •//•** \\ \hline
     3:03-3:13 & ``Unimpressed\textellipsis''*     & •//•*                    &                          & •//•*            &        \\ \hline
\end{tabular}

\vspace{0.2cm}
\hfill{*Enters in previous region with an anacrusis}

\hfill{** Layered in after downbeat}

\hfill{†loop is glitched}
    \caption{Full roadmap to billy woods and Kenny Segal's ``Checkpoints''}
    \label{tab:checkpointsfull}
\end{sidewaystable}


\end{document}
